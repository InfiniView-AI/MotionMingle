\documentclass{article}

\usepackage{float}
\restylefloat{table}

\usepackage{booktabs}

\title{Team Contributions: Rev 0\\\progname}

\author{\authname}

\date{}

%% Comments

\usepackage{color}

\newif\ifcomments\commentstrue %displays comments
%\newif\ifcomments\commentsfalse %so that comments do not display

\ifcomments
\newcommand{\authornote}[3]{\textcolor{#1}{[#3 ---#2]}}
\newcommand{\todo}[1]{\textcolor{red}{[TODO: #1]}}
\else
\newcommand{\authornote}[3]{}
\newcommand{\todo}[1]{}
\fi

\newcommand{\wss}[1]{\authornote{blue}{SS}{#1}} 
\newcommand{\plt}[1]{\authornote{magenta}{TPLT}{#1}} %For explanation of the template
\newcommand{\an}[1]{\authornote{cyan}{Author}{#1}}

%% Common Parts

\newcommand{\progname}{SFWRENG 4G06 Capstone Design Project} % PUT YOUR PROGRAM NAME HERE
\newcommand{\projname}{MotionMingle} % project name
\newcommand{\authname}{Team \#18, InfiniView-AI
\\ Anhao Jiao
\\ Kehao Huang
\\ Qianlin Chen
\\ Shu Qi
\\ Xunzhou Ye
} % AUTHOR NAMES

\usepackage{hyperref}
    \hypersetup{colorlinks=true, linkcolor=blue, citecolor=blue, filecolor=blue,
                urlcolor=blue, unicode=false}
    \urlstyle{same}


\begin{document}

\maketitle

\section{Demo Plans}

The team will showcase a working prototype of the Motion Mingle video
conferencing platform to demonstrate the basic functionalities and user
experience. The presented prototype is intended to include the following
functionalities and behaviours:
\begin{itemize}
\item A user interface for practitioner clients and instructor clients.
\item The ability to broadcast the annotated video stream from the instructor to
  multiple practitioner clients.
\item The ability to configure the desired annotation from each practitioner
  client.
\item A SFU server that processes the video stream from the instructor, renders
  visual effects through machine learning pipelines, and broadcasts annotated
  video streams to practitioner clients according to their configuration.
\end{itemize}

\section{Meeting Attendance}

\begin{table}[H]
  \centering
  \begin{tabular}{ll}
    \toprule
    \textbf{Student}   & \textbf{Meetings} \\
    \midrule
    Total        & 26          \\
    Qi Shu       & 26          \\
    Xunzhou Ye   & 26          \\
    Qianlin Chen & 26          \\
    Kehao Huang  & 26          \\
    Anhao Jiao   & 26          \\
    \bottomrule
  \end{tabular}
\end{table}

\section{Lecture Attendance}

\begin{table}[H]
  \centering
  \begin{tabular}{ll}
    \toprule
    \textbf{Student}   & \textbf{Lectures} \\
    \midrule
    Total        & 2           \\
    Qi Shu       & 2           \\
    Xunzhou Ye   & 2           \\
    Qianlin Chen & 2           \\
    Kehao Huang  & 2           \\
    Anhao Jiao   & 2           \\
    \bottomrule
  \end{tabular}
\end{table}


\section{Commits}

The number of commits in the main branch:
\begin{table}[H]
  \centering
  \begin{tabular}{lll}
    \toprule
    \textbf{Student}   & \textbf{Commits} & \textbf{Percent} \\
    \midrule
    Total        & 4          & 100\%      \\
    Anhao Jiao   & 4          & 100\%      \\
    Kehao Huang  & 4          & 100\%      \\
    Qianlin Chen & 4          & 100\%      \\
    Qi Shu       & 4          & 100\%      \\
    Xunzhou Ye   & 4          & 100\%      \\
    \bottomrule
  \end{tabular}
\end{table}

The number of commits in the main branch of our project is not a accurate
indicator of the team member contributions. As we squash all the commits in pull
requests.

The number of commits in unmerged branches:
\begin{table}[H]
  \centering
  \begin{tabular}{lll}
    \toprule
    \textbf{Student}   & \textbf{Commits} & \textbf{Percent} \\
    \midrule
    Total        & 19         & 100\%      \\
    Anhao Jiao   & 2          & 10.5\%     \\
    Kehao Huang  & 2          & 10.5\%     \\
    Qianlin Chen & 3          & 15.8\%     \\
    Qi Shu       & 16         & 84.2\%     \\
    Xunzhou Ye   & 4          & 21.0\%     \\
    \bottomrule
  \end{tabular}
\end{table}

Works in our project is distributed as we discussed. Certain parts of the
project may require more work than others, Members who work on those parts will
have more commits. As well as research contributions can not be reflected by the
number of commits.


\section{Issue Tracker}

\begin{table}[H]
  \centering
  \begin{tabular}{lll}
    \toprule
    \textbf{Student}   & \textbf{Authored (O+C)} & \textbf{Assigned (C only)} \\
    \midrule
    Qi Shu       & 27                & 21                   \\
    Xunzhou Ye   & 21                & 21                   \\
    Qianlin Chen & 1                 & 21                   \\
    Kehao Huang  & 1                 & 21                   \\
    Anhao Jiao   & 1                 & 21                   \\
    \bottomrule
  \end{tabular}
\end{table}


Qi Shu and Xunzhou Ye, in their roles as team leaders, bear the responsibility
of documenting lecture notes and strategizing for project deliverables. In cases
of their unavailability, other team members will assume these responsibilities.

\section{CICD}

In our project, the CI/CD process is implemented using GitHub Actions, as
defined by the workflow \texttt{Action-Test-Demo}. This workflow is triggered on
pushes to the \texttt{main} and \texttt{develop} branches. The CI/CD pipeline is
centered around ensuring coding standards, running unit tests, and conducting
comprehensive regression tests for each pull request. The workflow is named
dynamically using the GitHub actor's name, indicating who initiated the run,
with a playful rocket emoji to signify progress.

The job \texttt{APP-Test} runs on the latest Ubuntu runner and tests the application
in the \texttt{src/client/src} directory. It's designed to support multiple
versions of Node.js, specifically 16.x and 18.x, ensuring compatibility and
robustness across different environments. The steps include checking out the
code, setting up the Node.js environment, installing dependencies with
\texttt{npm install}, and then executing tests using \texttt{npm test}.

This CI/CD process emphasizes quality and collaboration by requiring approvals
from at least two team members for review, and necessitates the successful
execution of all defined GitHub actions before merging into the main branch.
This approach ensures high standards in code quality and functionality,
fostering a reliable and efficient development workflow.

\end{document}
\documentclass[12pt, titlepage]{article}

\usepackage{booktabs}
\usepackage{tabularx}
\newcolumntype{L}[1]{>{\raggedright\let\newline\\\arraybackslash\hspace{0pt}}p{#1}}
\usepackage{hyperref}
\hypersetup{
    colorlinks,
    citecolor=blue,
    filecolor=black,
    linkcolor=red,
    urlcolor=blue
}
\usepackage[round]{natbib}
\usepackage[table]{xcolor}
\usepackage[shortlabels]{enumitem}
\usepackage{float}
\usepackage{geometry}
\usepackage{pdflscape}
\usepackage{siunitx}
\usepackage[normalem]{ulem}
\newcommand{\rt}[1]{\textcolor{red}{#1}}

%% Comments

\usepackage{color}

\newif\ifcomments\commentstrue %displays comments
%\newif\ifcomments\commentsfalse %so that comments do not display

\ifcomments
\newcommand{\authornote}[3]{\textcolor{#1}{[#3 ---#2]}}
\newcommand{\todo}[1]{\textcolor{red}{[TODO: #1]}}
\else
\newcommand{\authornote}[3]{}
\newcommand{\todo}[1]{}
\fi

\newcommand{\wss}[1]{\authornote{blue}{SS}{#1}} 
\newcommand{\plt}[1]{\authornote{magenta}{TPLT}{#1}} %For explanation of the template
\newcommand{\an}[1]{\authornote{cyan}{Author}{#1}}

%% Common Parts

\newcommand{\progname}{SFWRENG 4G06 Capstone Design Project} % PUT YOUR PROGRAM NAME HERE
\newcommand{\projname}{MotionMingle} % project name
\newcommand{\authname}{Team \#18, InfiniView-AI
\\ Anhao Jiao
\\ Kehao Huang
\\ Qianlin Chen
\\ Shu Qi
\\ Xunzhou Ye
} % AUTHOR NAMES

\usepackage{hyperref}
    \hypersetup{colorlinks=true, linkcolor=blue, citecolor=blue, filecolor=blue,
                urlcolor=blue, unicode=false}
    \urlstyle{same}


\begin{document}

\title{Project Title: System Verification and Validation Plan for \progname{}}
\author{\authname}
\date{\today}

\maketitle

\pagenumbering{roman}

\section*{Revision History}

\begin{tabularx}{\textwidth}{llX}
\toprule {\bf Date} & {\bf Developer(s)} & {\bf Change} \\
\midrule
November 3, 2023 & KH, QS, AJ, QC, XY & Initial Draft \\
March 6, 2024 & KH, QS, AJ, QC, XY & Rev 0.1 \\
March 27, 2024 & KH, QS, AJ, QC, XY & Rev 1 \\
\bottomrule
\end{tabularx}

\newpage

\tableofcontents

\listoftables

\newpage

\section{Symbols, Abbreviations, and Acronyms}

\subsection{Symbols}
Table \ref{tab:abbrv} lists all the symbols, abbreviations, and acronyms used
throughout this document.
\begin{table}[H]
  \centering
  \begin{tabular}{L{0.25\linewidth}L{0.8\linewidth}} \toprule
    \textbf{Symbol}        & \textbf{Description}                                                                                                                                                                     \\ \midrule
    Instructor       & A person who teaches a Tai Chi class through an online conference system.                                                                                                          \\
    Performance test & Testing the performance(includes different metrics) of a system.                                                                                                                   \\ 
    Pipeline         & A sequence of processing elements connected in series, which are responsible for generating annotations.                                                                           \\
    POC              & proof of concept.                                                                                                                                                                  \\
    Practitioner     & A person who learns Tai Chi through an online conference system.                                                                                                                   \\
    SFU              & Selective Forwarding Unit, a component in real-time communication systems like WebRTC that routes and selectively forwards audio and video streams from one participant to others. \\
    SRS              & The Software Requirements Specification document.                                                                                                                                  \\
    Stress test      & Testing the limits (often the amount of data) of a system.                                                                                                                         \\
    TA               & Teaching Assistant.                                                                                                                                                                \\
    Tai Chi          & A classical Chinese martial art system practiced for health promotion and rehabilitation.                                                                                          \\
    Type: Functional & Functional test: An input/output blackbox type test.                                                                                                                               \\
    Type: Dynamic    & Dynamic test: Test type that requires code/program to be executed.                                                                                                                 \\
    Type: Manual     & Manual test: Test type that requires the user to do the test without using automated tools.                                                                                        \\
    Type: Automated  & Automated test: Test type containing code/program that can be executed.                                                                                                            \\
    UI               & User Interface.                                                                                                                                                                    \\
    V\&V Plan        & The verification and validation plan document(this document).                                                                                                                      \\ \bottomrule
  \end{tabular}
  \caption{List of symbols, abbreviations, and acronyms}
  \label{tab:abbrv}
\end{table}

\subsection{Symbolic Constants}\label{sec:symbolic-constants}
\begin{table}[H]
  \centering
  \begin{tabularx}{\linewidth}{lX} 
    \toprule
    \textbf{Symbolic Constant} & \textbf{Value} \\
    \midrule
    \rt{MAX\_RESPONSE\_TIME} & \rt{2 seconds}\\
    \rt{TEST\_ITERATIONS} & \rt{100}\\
    \rt{TEST\_ACCOUNTS\_NUMBERS} & \rt{5}\\
    \rt{MIN\_NUM\_USER\_SUPPORTED} & \rt{5}\\
    \rt{TEST\_DISCONNECT\_DURATION} & \rt{30 seconds}\\

    \rt{MAX\_DELAY}  & \rt{\SI{500}{\milli\second}} \\
    \rt{MAX\_TEMP}  & \rt{\SI{65}{\degreeCelsius}} \\
    \rt{MIN\_RES}  & \rt{720p} \\
    \rt{TARGET\_FACTOR}  & \rt{\SI{85}{\percent}} \\

    \rt{MAX\_ACCEPTABLE\_DELAY}  & \rt{2 seconds} \\
    \rt{MAX\_ERROR\_RATE}  & \rt{\SI{1}{\percent}} \\
    \rt{MIN\_OPERATIONAL\_PERIOD}  & \rt{8 hours} \\
    \bottomrule
  \end{tabularx}
  \caption{List of Symbolic Constants}
  \label{tab:symbolic-constants}
\end{table}

\newpage
\pagenumbering{arabic}

\section{General Information}

This section introduces some general information about the present document,
including a summary, objectives of the document, and related documentation\sout{s} that
were mentioned or referred to in the present document.

\subsection{Summary}

This document describes the testing, validation, and verification procedures
that will be implemented for the “Real-time Digital Annotation Solution for
Online Exercise Lessons” project. The general functionality of this project is
to set up a live stream call streaming online TaiChi exercising class, while
practitioners can configure the annotation they like to be added \sout{on} to the
instructor’s video stream. The test cases specified in this document were
conceived prior to the majority of the implementation and are intended to be
used by the project team for future reference during project development,
testing, and maintenance.

\subsection{Objectives}

\sout{The objectives of the validation and verification document for this project \sout{is}\rt{are}
to serve as a comprehensive guideline for validating project requirements,
ensuring the correctness of code, and establishing metrics to enhance user
satisfaction. As all developers working on this project also have other life
priorities, we don’t have the resources to test out all aspects of the project.
Some parts of the project will be considered out of scope. For instance, we will
not be testing the machine learning model utilized in the project, and we assume
the machine learning model has already been verified by its implementation team.
\sout{And} Our tests specified in this document will prioritize the functionalities and
connections between components.}\\
\rt{The objectives of the validation and verification document for this project are to provide 
a comprehensive guide for validating project requirements, ensuring the correctness of code, 
and establishing metrics to enhance user satisfaction. Given the scope of the project and the 
time constraints, certain aspects will be prioritized to ensure effective use of resources. For 
example, this document will not cover the testing of the machine learning model utilized in the 
project, as it is presumed to have undergone verification by its dedicated implementation team. 
The focus of the testing will be on critical functionalities and the interconnectivity between 
components, to ensure a robust and seamless user experience.}


\subsection{Relevant Documentation}

Relevant documentation\sout{s} include:
\begin{description}
\item[\href{https://github.com/InfiniView-AI/MotionMingle/blob/main/docs/SRS/SRS.pdf}{SRS}]
  The SRS document is a related document as most of the requirements specified
  in this document come from the SRS document.
\item[\href{https://github.com/InfiniView-AI/MotionMingle/blob/main/docs/DevelopmentPlan/DevelopmentPlan.pdf}{Development Plan}]
  The Development Plan document is a related document as it specifies the
  technologies that will be used in the project, including testing tools.
\item[\href{https://github.com/InfiniView-AI/MotionMingle/blob/main/docs/HazardAnalysis/HazardAnalysis.pdf}{Hazard Analysis}]
  The Hazard Analysis document is a related document as some of the requirements
  come from the Hazard Analysis document.
\item[\href{https://github.com/InfiniView-AI/MotionMingle/blob/main/docs/Design/SoftArchitecture/MG.pdf}{\rt{Module Guide}}]
  \rt{The Module Guide is essential as it will outline the high-level architecture of the system, providing a roadmap for development and facilitating understanding of the system's structure and modularity.}
\item[\href{https://github.com/InfiniView-AI/MotionMingle/blob/main/docs/Design/SoftDetailedDes/MIS.pdf}{\rt{Module Interface Specification}}]
  \rt{The Module Interface Specification is pivotal for detailing the interactions between various system components, ensuring compatibility and defining the communication protocols necessary for module integration.}
\end{description}

\section{Plan}

This section introduces the verification and validation team with their
associated responsibilities, and provides verification plans for SRS, system
design, V\&V plan, and system implementation. In addition, it introduces several
automated testing and verification tools that are planned to be utilized by the
testing team during the verification and validation process.

\subsection{Verification and Validation Team}

The verification and validation process is a collective responsibility within
the team, with each member actively participating.

\begin{description}
\item[Kehao Huang, Qi Shu] Responsible for Integration of Machine Learning Models
  and Application Back-end Testing. \rt{This involves unit testing to validate each 
  individual component of the machine learning models for correctness, efficiency, 
  and reliability. Integration testing will follow, which checks the interaction 
  between the machine learning models and the back-end systems to ensure data flows 
  correctly and that the models perform as expected within the application environment. 
  They might also use validation testing to confirm that the models meet the initial 
  requirements and perform accurately with real-world data.}
\item[Qianlin Chen, Xunzhou Ye, Anhao Jiao] Responsible for Full Stack Development
  Testing. \rt{For the front-end, they will use black box testing, which tests the 
  user interface without knowledge of the underlying code, to ensure the user experience 
  is as designed and is bug-free. They will conduct white box testing on the back-end to 
  verify the internal workings of the application's logic. System testing will also be 
  part of their responsibility to check the complete and integrated software product 
  functions in line with the requirements. Finally, they'll perform end-to-end testing 
  to simulate real user scenarios and make sure the application behaves as expected in 
  production.}
\end{description}

\subsection{SRS Verification Plan}

\sout{Our SRS (Software Requirements Specification) will be subject to verification
through a flexible and collaborative feedback process involving our classmates
and Teaching Assistants (TAs). Given that the individuals involved in reviewing
our VnV (Verification and Validation) plan and our TAs possess in-depth
knowledge of our SRS, we have designed a verification plan that encourages them
to explore our application with a degree of improvisation. The primary goal is
to ensure comprehensive coverage of every functionality specified in our SRS.

Key aspects of the verification approach include flexible feedback, exploratory
assessment, and functional coverage. Reviewers will be encouraged to explore the
application using their own methods and approaches, allowing for creative and
innovative testing techniques. The review process will actively seek out and
identify possible behavioural bugs and anomalies within the application. This
approach leverages the diverse perspectives and problem-solving abilities of our
reviewers, helping us ensure that our SRS is thorough, accurate, and aligned
with the project's objectives, ultimately contributing to the success of our
project.}\\
\rt{The Software Requirements Specification (SRS) is crucial for defining the project 
scope and guiding its implementation. To ensure the SRS meets the project's standards 
and objectives, a rigorous verification plan involving iterative feedback from peers 
and Teaching Assistants (TAs) will be employed. This process begins with an initial 
review where the SRS will be scrutinized for clarity, completeness, and alignment with 
the project's goals. Feedback will be systematically gathered and categorized to 
identify areas needing refinement. Additionally, exploratory assessment techniques 
will be utilized, enabling reviewers to interact with the application's prototype 
or review design mockups, thereby ensuring requirements are realistically translated 
into application features.}\\

\rt{Subsequent to the exploratory phase, functional testing based on the SRS will be 
performed to validate the behavior of the application. This testing will be geared 
toward verifying that each function of the application performs according to the 
specified requirements. The feedback gathered will be carefully integrated into the 
SRS, with the development team revising any sections as necessary to improve clarity 
and accuracy. The final step of the verification plan involves a comprehensive review 
to confirm that all feedback has been effectively addressed, solidifying the SRS as 
a true reflection of the intended system. By leveraging the diverse insights from 
our reviewers, the SRS will be refined to a state that facilitates a smooth transition 
into the development phase, setting a strong foundation for the project's success.}

\subsection{Design Verification Plan}

The design verification plan outlines the strategies and procedures that will be
taken by the team to verify the correctness and reliability of the design of our
Tai Chi video conferencing application. This plan will be treated as a guideline
during the testing phase to ensure that the design meets the intended
requirements and are able to mitigate potential hazards that the team has
discovered. The following procedures will be taken by the testing team during
the verification process:

\begin{description}
\item[Document Review] The system’s design documentation and related material will
  be reviewed by each member of the testing team after the first draft is
  produced. During the document review process, the testing team will
  \begin{enumerate}
  \item ensure that the design of the system aligns with all the functional and
    non-functional requirements, and
  \item assess if the document accurately describes the intended functionality and
    behaviour of the system. Any design elements that deviate from specified
    requirements should be recorded and reported for further discussion.
  \end{enumerate}
\item[Prototype Testing] After the scheduled POC demo, there should be a completed
  prototype for each key component of the system. The prototypes will be
  assembled to conduct system testing. During the prototype testing process, the
  testing team will verify the usability of the user interface, system
  performance under different loads, and handling of failure conditions.
\end{description}

\subsection{Verification and Validation Plan Verification Plan}

The verification and validation plan verification plan provides a guideline for
the testing team to ensure the quality, accuracy and completeness of the
verification and validation plan document itself. The following procedures will
be taken by the testing team during the verification process:
\begin{description}
\item[Document Review] The V\&V plan document will be reviewed by each member in
  the testing team after the initial draft is finished. The testing team will
  ensure that the document addresses all the key aspects of the verification and
  validation process, aligns with the project objectives and goals, and
  maintains completeness and consistency.
\item[Cross-Reference with other Documents] After the initial draft of the V\&V
  plan document is completed, the document will be cross-referenced with other
  projects including problem statement, SRS and hazard analysis to verify
  consistency and alignment with the project’s overall objective.
\item[Stakeholder Review] The initial draft of the V\&V plan document will be
  presented to relevant stakeholders to gather feedback. The feedback will be
  reviewed and discussed by the testing team, and changes to the V\&V plan
  document will be made accordingly to ensure stakeholders’ satisfaction.
\end{description}

\subsection{Implementation Verification Plan}

\sout{For ensuring}\rt{To ensure} the accuracy and effectiveness of our code implementation, we will
employ a multi-faceted verification approach.
\begin{description}
\item[System Testing] The primary means of validating the functionality and
  efficiency of our software will be through system testing. This process will
  confirm that our software consistently executes all designated tasks as per
  the given specifications. Furthermore, Non-Functional Requirements (NFRs)
  validation will also be incorporated as part of this system testing. The tests
  will simulate real-world conditions and environments to ensure our application
  delivers optimal performance.
\item[Unit Testing] At a backend level, unit tests will be in place to evaluate
  the individual components or modules within our system. These tests, aided by
  appropriate stubs and drivers, will focus on the quality of each module,
  ensuring they operate as intended. For every function within a module, we will
  formulate distinct test cases to guarantee precision and reliability.
\item[Static Verification] Beyond dynamic testing, our implementation will be
  subjected to static verification methods. We propose a peer review system
  where fellow developers and the TA will conduct a thorough examination of our
  source code. This manual inspection serves a dual purpose:
  \begin{enumerate}
  \item Highlight any deviations from the accepted coding standards and
    conventions.
  \item Unearth potential logical inconsistencies or oversight\rt{s} in our
    coding that automated tests might miss.
  \end{enumerate}
\end{description}
By integrating these multifaceted testing and verification techniques, we aim to
ensure our code is both robust and aligned with \sout{the} both functional and
non-functional requirements. This plan will not only verify that the
implementation meets the designated specifications but will also foster code
quality, readability, and maintainability.

\subsection{Automated Testing and Verification Tools}

To ensure the robustness and stability of our real-time video conference
application, we will utilize a range of automated testing and verification tools
tailored to each specific component of our system.

\begin{table}[h]
  \centering
  \begin{tabularx}{1.0\linewidth}{ll} \toprule
    \textbf{Module/Component}            & \textbf{Testing \& Verification Tools}        \\ \midrule
    \multicolumn{2}{l}{\textbf{Front-End (Client-Side)}}                                    \\
    UI                             & React-testing-library                   \\
    WebRTC Client                  & KITE, testRTC                           \\
    Real-Time Communication Engine & RCTPeerConnection API, WebRTC Internals \\ \midrule
    \multicolumn{2}{l}{\textbf{Back-End (Server-Side)}}                                     \\
    Signaling Server               & Socket.io tester, Wireshark             \\
    SFU                            & Jitsi-meet-torture                      \\
    API Server                     & Jest, Cypress.io                        \\ \bottomrule
  \end{tabularx}
  \caption{List of testing tools}
  \label{tab:testingtools}
\end{table}

The tinter tool for both Front-End and Back-End will be ESlint in order to make
sure the coding standard is followed. Table \ref{tab:testingtools} listed the
relevant testing tools. For the client-side components:
\begin{description}
\item[UI] We will use React-testing-library for unit tests in the user interface,
  as the react-testing-library is the default and most used library for react
  testing.
\item[WebRTC Client] We will use KITE (Karoshi Interoperability Testing Engine)
  for compatibility and interoperability tests, while testRTC helps in
  monitoring, functional and load testing of the WebRTC services.
\item[Real-Time Communication Engine] The RTCPeerConnection API will be used to
  inspect the properties and performance of a WebRTC peer connection. The WebRTC
  \sout{I}\rt{i}nternals tool offers deeper insights into the client-side WebRTC status.
\end{description}

On the server-side:
\begin{description}
\item[Signaling Server] Socket.io tester will help in validating WebSocket
  communication, while Wireshark will allow for network protocol analysis
  ensuring data packets are transmitted correctly.
\item[SFU] The Jitsi-meet-torture will be employed for stress and performance
  testing. API Server: Jest will be employed for unit testing our server-side
  logic, and Cypress.io will be used for more complex integration tests.
\end{description}

Through this diverse toolkit, we aim to maintain the high reliability and
functionality standards of our real-time video conference application.


\subsection{Software Validation Plan}

In the development phase, sync-up and review sessions will be conducted with the
stakeholders on a regular basis. Intermediate results or evidence of progression
will be shared and discussed with supervising stakeholders. This is to ensure
that the requirement documents together with the development, under the
assumption that the development is verified to satisfy the requirements, are
aligned with the stakeholders’ expectations and needs.

Near the end of the capstone timeline as the product is being gradually
finalized within the refined scope of the capstone project, a software design
evaluation and validation study will be conducted with volunteers from each of
the intended user groups. Specifically, at least one Tai Chi instructor and at
least one Tai Chi practitioner or potential Tai Chi student should be invited to
evaluate the software product. The process will be broken down into two main
components. The first one \sout{being}\rt{is} an observation study. Study participants will be
asked to perform specific tasks as identified and described as use case
scenarios in SRS. For the instructor, tasks include:
\begin{itemize}
\item Launch the application and set up appropriate peripherals, such as webcams\sout{,} \rt{and}
  microphones, to create an environment ready for online instructional Tai Chi
  demonstrations.
\item Navigate through the application interfaces and initiate a video conferencing
  session.
\end{itemize}
For the practitioner, tasks include:
\begin{itemize}
\item Navigate through the application interfaces and join an existing video
  conferencing session. Attempt to follow along the instructor’s movements.
\end{itemize}

Developments will observe and note how well the participants can complete the
given tasks, making connections to the fit criteria\sout{s} of the non-functional
requirements. The second component will be a user satisfaction survey (refer to
Appendix \ref{sec:survey}) conducted after the observation study. Participants
will be asked to take a survey to rate their experience in using the
application, \sout{in the}\rt{on a} scale of 1 (unsatisfied) to 5 (satisfied). The targeted
satisfactory factor is set to TARGET\_FACTOR. The survey also includes questions
about difficulties the user has encountered, and potential future improvements
can be \sout{done}\rt{made} to the application.

For future developments beyond the capstone timeline, the development team will
continue to conduct review sessions and presentations with stakeholder
representatives, making sure that the development stays on track. For each major
revision or software release, product evaluations will be carried out with
targeted users. The feedback and user data gathered from the evaluation process
will be used to guide the development and refinement of the product.

\section{System Test Description}

This sections provides system tests for both functional requirements and
non-functional requirements.

\subsection{Tests for Functional Requirements}

\begin{enumerate}[FR-T1]
\item \label{FRT1}
  \begin{description}
  \item[Type] Functional, Dynamic, Automated
  \item[Initial State] Client application is running on the user's device, but the
    user didn’t do any operations yet.
  \item[Input/Condition] User clicks on the applicable
    identity(instructor/practitioner) button
  \item[Output/Result] The live stream video window pops out on the user's screen.
  \item[How Test Will Be Performed] Test will be done automatically by the Cypress
    testing framework in the CI pipeline. The test code will simulate a user's
    click on the screen and click on the applicable identity button, then assert
    the live stream video window showed up on the user's screen within the next
    \sout{2 seconds}\rt{MAX\_RESPONSE\_TIME}.
  \end{description}
\item \label{FRT2}
  \begin{description}
  \item[Type] Functional, Dynamic, Manual
  \item[Initial State] Application running on user’s computer, and the user has
    clicked on “the instructor identity button” to indicate they are a TaiChi
    instructor. A window asking for permission to use the camera on the
    instructor's device popped out.
  \item[Input/Condition] User allow/deny the webcam permission
  \item[Output/Result] The webcam on the instructor’s device is turned on
  \item[How Test Will Be Performed] Test will be done manually by developers
    running the prototype. The developer will make sure the on-device camera
    should be turned on after allowing the webcam permission, or the on-device
    camera should not be turned on after denying the webcam permission. The test
    should be done at least \sout{100}\rt{TEST\_ITERATIONS} times to ensure the webcam on the instructor’s
    device is turned on properly.
  \end{description}
\item \label{FRT3}
  \begin{description}
  \item[Type] Functional, Dynamic, Automatic
  \item[Initial State] \sout{b}\rt{B}oth client applications and the server are running.
  \item[Input/Condition] The user clicks on the applicable identity button to
    indicate they are an instructor or a practitioner.
  \item[Output/Result] A log message indicates connection between the user’s device
    and the server has been established.
  \item[How Test Will Be Performed] Test will be done automatically by the Cypress
    testing framework in the CI pipeline. The test code for the client
    application component will try to connect to a server and establish
    connection, then it will assert the log indicates a successful connection
    showed up in the console.
  \end{description}
\item \label{FRT4}
  \begin{description}
  \item[Type] Functional, Dynamic, Automated
  \item[Initial State] The live stream Window for practitioners.
  \item[Input/Condition] The user’s device has established a connection with the
    server as a practitioner device.
  \item[Output/Result] A request from the client device to the server for accessing
    the list of available annotation configuration\rt{s}.
  \item[How Test Will Be Performed] Test will be done automatically by the Cypress
    testing framework. The test code will first establish \rt{a} connection with a
    server as a practitioner device. Once the connection is established, the
    server will send a list of available annotations to the client device, and
    the client device should render the list into a selectable list of type\rt{s} of
    annotation on the user's screen. The test for the client application will
    assert the list of available annotations from the server is received and
    rendered as a selectable list.
  \end{description}
\item \label{FRT5}
  \begin{description}
  \item[Type] Functional, Dynamic, Automated
  \item[Initial State] The selectable list of the type of annotations is
    rendered on the user's screen.
  \item[Input/Condition] Practitioner’s selection on the list of types of
    annotations.
  \item[Output/Result] A request(that reflects user’s annotation selection) from
    the client device to the server for updating the annotation configuration,
    with a log indicating the request is sent.
  \item[How Test Will Be Performed] Test will be done automatically by the
    Cypress testing framework. The unit test code will first establish connection
    with a server as a practitioner device. The testing framework will then
    simulate a user's click to configure annotation selections on the rendered
    selectable list. After that, the testing framework will assert a request
    that reflects the annotation configuration \sout{is} received by the server. As
    well, assert a log is shown on the client device console indicating the
    request has been sent.
  \end{description}
\item \label{FRT6}
  \begin{description}
  \item[Type] Functional, Dynamic, Automated
  \item[Initial State] The system is running and actively connected to
    practitioners.
  \item[Input/Condition] Practitioners initiate updates to annotation
    configurations.
  \item[Output/Result] The system receives and processes the updated annotation
    configurations.
  \item[How Test Will Be Performed] Automated tests will be conducted to verify
    that the system successfully listens to updates on annotation configurations
    from practitioners. In a controlled test environment, practitioners will
    initiate updates to annotation configurations. The test will then verify
    that the system correctly receives and processes the updated annotation
    configurations, ensuring that it can effectively listen to and respond to
    practitioner-initiated updates.
  \end{description}
\item \label{FRT7}
  \begin{description}
  \item[Type] Functional, Dynamic, Automated
  \item[Initial State] The server is running and actively receiving annotation
    configuration updates.
  \item[Input/Condition] In a controlled test environment, the practitioner-client
    initiates the update of an annotation configuration. The update is sent to
    the server for processing.
  \item[Output/Result] The expected result is that the server correctly processes
    the received annotation configuration from the practitioner-client.
  \item[How Test Will Be Performed] In the test environment, a simulated annotation
    configuration update is initiated by the practitioner-client. The system
    will process this update, and the test will verify that the processing is
    accurate, ensuring that the server correctly interprets and acts upon the
    received annotation configuration.
  \end{description}
\item \label{FRT8}
  \begin{description}
  \item[Type] Functional, Dynamic, Automated
  \item[Initial State] The server has received and processed the annotation
    configuration.
  \item[Input/Condition] The server uses the received annotation configuration to
    configure machine learning pipelines.
  \item[Output/Result] The machine learning pipelines are arranged and configured
    based on the annotation configuration.
  \item[How Test Will Be Performed] Automated tests will be performed to ensure
    that the system can configure machine learning pipelines based on annotation
    configuration. Sample annotation configurations will be used to assess the
    correct arrangement and setup of machine learning components within the
    system.
  \end{description}
\item \label{FRT9}
  \begin{description}
  \item[Type] Functional, Dynamic, Automated
  \item[Initial State] The machine learning pipelines are configured and active.
  \item[Input/Condition] The instructor's video stream is processed with the
    annotation configuration.
  \item[Output/Result] The instructor's video stream is rendered with accurate
    annotations.
  \item[How Test Will Be Performed] Dynamic tests will be conducted to validate the
    rendering process of the instructor's video stream with accurate
    annotations. Real or simulated video streams, combined with annotation
    configurations, will be used to confirm that the system correctly renders
    the video with the expected annotations.
  \end{description}
\item \label{FRT10}
  \begin{description}
  \item[Type] Functional, Dynamic, Automated
  \item[Initial State] The server is actively connected to practitioner\rt{-}client.
  \item[Input/Condition] The annotated video stream is generated and ready for
    transmission.
  \item[Output/Result] The annotated video stream is transmitted to each
    practitioner-client through their established connections.
  \item[How Test Will Be Performed] Automated tests will be employed to assess
    the transmission of annotated video streams from the server to practitioner
    clients. The test will involve simulating the generation of annotated video
    streams and verifying their successful transmission to practitioner-clients
    through their established connections.
  \end{description}
\item \label{FRT11}
  \begin{description}
  \item[Type] Functional, Dynamic, Automated
  \item[Initial State] The signaling server is running.
  \item[Input/Condition] Signaling requests for WebRTC connections are initiated.
  \item[Output/Result] The signaling server consistently responds to requests and
    establishes WebRTC connections.
  \item[How Test Will Be Performed] Automated tests will be used to evaluate the
    availability and reliability of the signaling server. These tests will
    involve monitoring and verifying the responsiveness of the signaling server
    when requests for WebRTC connections are initiated, ensuring its consistent
    availability.
  \end{description}
\item \label{FRT12}
  \begin{description}
  \item[Type] Functional, Dynamic, Automated
  \item[Initial State] The client application is running, but the user \sout{didn’t do}\rt{hasn’t done}
    any operations yet
  \item[Input/Condition] The user joining \rt{the} video stream session.
  \item[Output/Result] A button to identify if a user is an instructor or a
    practitioner is rendered.
  \item[How Test Will Be Performed] Test will be done automatically by unit
    testing framework in the CI pipeline. The test code will simulate a user's
    clicks to join a video stream session, then assert the button to identify if
    a user is an instructor or a practitioner showed up on the user's screen
    within the next \sout{1 second}\rt{MAX\_RESPONSE\_TIME}.
  \end{description}
\end{enumerate}

\subsection{Tests for Nonfunctional Requirements}

\begin{enumerate}[NFR-T1]
\item \label{NFRT1}
  \begin{description}
  \item[Type] \sout{Functional, }Dynamic, Manual
  \item[Initial State] The system is in a typical operational state with all
    components and services running, including the user interface.
  \item[Input/Condition] User interactions such as button clicks and menu
    selections.
  \item[Output/Result] The user is able to interact with the client application and
    understand the response from and results yielded by the system.
  \item[How Test Will Be Performed] Test will be done manually by developers in
    coordination with representatives of the intended user groups of different
    ages and various experiences in practicing Tai Chi. The users will be asked
    to complete a series of navigations within the user interface and
    interactions with the system, such as creating, joining, and leaving a video
    conference session. The user experience will be evaluated by the developers
    in terms of the completeness of the tasks, the time it takes to complete the
    tasks, as well as the user’s experience in completing the tasks gathered
    through a short and informal interview. The test is considered successful if
    \sout{the} 4 out of 5 developers agree that the overall user experience meets their
    expectation\rt{s}. Otherwise, the test is considered a failure.
  \end{description}
\item \label{NFRT2}
  \begin{description}
  \item[Type] Structural, Dynamic, Manual
  \item[Initial State] The system is in a typical operational state with all
    components and services running, including the user interface.
  \item[Input/Condition] User interactions such as button clicks and menu
    selections.
  \item[Output/Result] System response to user interactions.
  \item[How Test Will Be Performed] Test will be done manually by developers
    running the system. Testers will perform various user interactions, such as
    button clicks and menu selections, across different areas of the system UI.
    A test scenario should be designed for each use case to cover all possible
    interactions within that use case. Use performance monitoring tools to
    record the response time for each interaction. Repeat each test scenario at
    least \sout{5}\rt{TEST\_ACCOUNTS\_NUMBERS} times for each use case under different system loads to account for
    variability. Compare the recorded response times against the 1-second
    requirement. If the system consistently responds within this timeframe for
    all test scenarios, the test is considered successful. Any deviations beyond
    \sout{1 second}\rt{MAX\_RESPONSE\_TIME} will be considered a failure.
  \end{description}
\item \label{NFRT3}
  \begin{description}
  \item[Type] Structural, Dynamic, Manual
  \item[Initial State] The system is in a stable operational state with necessary
    services and components active. No users are currently logged in.
  \item[Input/Condition] \sout{5}\rt{TEST\_ACCOUNTS\_NUMBERS} test accounts (or more) for users. Predefined user
    actions/scripts (relevant to normal system interactions during peak
    periods).
  \item[Output/Result] The system remains stable and responsive. All user
    transactions are processed successfully. System performance metrics (such as
    response time, throughput, \rt{and} resource utilization) stay within acceptable
    limits.
  \item[How Test Will Be Performed] Set up an automated load testing tool and
    create test scripts mimicking typical user actions during peak periods. This
    might include logging in, accessing services, executing standard operations,
    and logging out. Initiate the test by having \sout{5}\rt{MIN\_NUM\_USER\_SUPPORTED} users (controlled by the load
    testing software) log into the system simultaneously, executing the
    predefined scripts that mimic peak load activities. Gradually increase the
    system load by adding more users (if applicable) to ensure it can handle
    more than the minimal load without performance degradation. Monitor system
    performance indicators, focusing on metrics like transaction success rate,
    response times, \rt{and} CPU/memory usage. Continuously record system metrics
    throughout the testing duration. After completing the test, analyze the data
    to assess whether the system could sustain \sout{the} a minimum of \sout{5}\rt{MIN\_NUM\_USER\_SUPPORTED} users during
    simulated peak conditions without performance degradation.
  \end{description}
\item \label{NFRT4}
  \begin{description}
  \item[Type] Structural, Dynamic, Manual
  \item[Initial State] The system is operational, with all services and components
    running. User accounts for test\rt{s} are set up, and no tasks are being
    performed.
  \item[Input/Condition] Erroneous user input data. Improper user actions.
  \item[Output/Result] The system detects and rejects invalid inputs or actions,
    providing clear and helpful error messages to the user. Transactions or
    actions are either rolled back safely or do not proceed until valid input is
    provided.
  \item[How Test Will Be Performed] Identify common user actions and points in the
    system where user errors could occur. Design a series of test cases that
    deliberately introduce various user errors. These might include inputting
    invalid data, using incorrect sequences of commands, or attempting
    unauthorized actions. Prepare the test environment, ensuring that the system
    is running and that tools for monitoring system behaviour (like log
    watchers, debuggers, etc.) are in place. Monitor how the system handles
    these errors, noting any deviations from expected behaviour, such as system
    crashes, unclear or missing error messages, incorrect handling, or data
    corruption. Check system performance indicators to ensure that the errors
    did not affect the system's stability or consume excessive resources.
    Document the outcomes of each test case, including the errors introduced,
    the system's response, and whether the result was as expected.
  \end{description}
\item \label{NFRT5}
  \begin{description}
  \item[\sout{Type}] \sout{Functional, Dynamic, Manual}
  \item[\sout{Initial State}] \sout{The system is in a stable, operational state with the core
    functionalities running. No extensions are installed at the outset of the
    test.}
  \item[\sout{Input/Condition}] \sout{One or more sample extensions ready for installation}
  \item[\sout{Output/Result}] \sout{Extensions are successfully installed on the system. The
    system recognizes and integrates new extensions' functionalities. System
    stability and core functionality remain uncompromised after the addition of
    the extensions.}
  \item[\sout{How Test Will Be Performed}] \sout{Identify or develop sample extensions for the
    test. These should be varied to represent different types of functionality
    users might add to the system. Ensure the system is running in a controlled
    test environment that replicates the end-user environment. Manually add the
    extensions to the system, following the documented procedures. Observe the
    installation process for any discrepancies, errors, or issues that arise at
    the same time. After installation, verify that each extension is functioning
    within the system as intended. Check interoperability with existing features
    to ensure that there are no conflicts or issues. Test the system's core and
    extended functionalities to ensure all operate correctly.}
  \end{description}
\item \label{NFRT6}
  \begin{description}
  \item[Type] Structu\rt{r}al, Dynamic, Automated
  \item[Initial State] The system, with the SFU component, is in a stable,
    operational state. Initially, there is a base number of video streams being
    processed, which is within the SFU's current capacity.
  \item[Input/Condition] A load testing tool or custom script capable of simulating
    multiple, simultaneous video streams.
  \item[Output/Result] The SFU successfully processes an increasing number of video
    streams.
  \item[How Test Will Be Performed] Configure a load testing tool to simulate video
    streams that mimic real-world usage. This could involve setting up dummy
    user connections each initiating video streams. Run the system with the base
    number of video stream\sout{s}. Gradually increase the number of simulated video
    streams being sent through the SFU, starting from the base number and
    increasing in pre-defined increments. Identify the point(s) at which video
    quality begins to degrade beyond acceptable levels, or system resources
    reach critical usage. Collect and analyze data to determine the maximum
    number of video streams the SFU can handle while maintaining acceptable
    quality of service.
  \end{description}
\item \label{NFRT7}
  \begin{description}
  \item[Type] Structural, Dynamic, Manual
  \item[Initial State] This system is in a typical operational state with all
    components and services running, the user is in a live session.
  \item[Input/Condition] Network Interruption/Network Resumption
  \item[Output/Result] The system attempts to resume the previous session.
  \item[How Test Will Be Performed] Test will be done manually by developers
    running the system. Testers will join a live session from either the
    instructor\rt{-}client application or the practitioner\rt{-}client application. Tester
    will intentionally disrupt the internet connection on the device that the
    system is running on, wait for \sout{30 seconds}\rt{TEST\_DISCONNECT\_DURATION}, and resume the internet
    connection on the device to simulate an application crash or internet
    interruption. The tester shall restart the application and then manually
    inspect the system UI to see if the user is able to automatically rejoin the
    previous live session after the network resumption. The test will be
    repeated \sout{for} \sout{5}\rt{TEST\_ACCOUNTS\_NUMBERS} times on both the instructor\rt{-}client application and the
    practitioner\rt{-}client application under different system loads. The test is
    considered successful if the user automatically joins the previous live
    session after application restart when \rt{the} network resumes. The test will be
    considered a failure if, after restarting the application, the user is
    unable to automatically enter a live session or joins the wrong live
    session.
  \end{description}
\item \label{NFRT8}
  \begin{description}
  \item[Type] Structural, Dynamic, Manual
  \item[Initial State] This system is in a typical operational state with all
    components and services running, the user is in a live session.
  \item[Input/Condition] Network instability
  \item[Output/Result] A pop\rt{-}up window for network instability warning.
  \item[How Test Will Be Performed] Test will be done manually by developers
    running the system. Testers will join a live session from either the
    instructor\rt{-}client application or the practitioner\rt{-}client application. The
    tester will produce network fluctuation/instability on the internet that the
    system has access to. The tester shall monitor the system UI to inspect for
    the appearance of network instability warning. The test will be repeated \sout{for}
    \sout{5}\rt{TEST\_ACCOUNTS\_NUMBERS} times on both the instructor\rt{-}client application and the practitioner
    client application under different system loads. The test is considered
    successful if a pop\rt{-}up window for network instability warning is displayed
    after the network fluctuation/instability is created. Otherwise, if no pop
    up window is displayed to provide \rt{a} network instability warning, the test will
    be considered a failure.
  \end{description}
\item \label{NFRT9}
  \begin{description}
  \item[Type] Structural, Dynamic, Manual
  \item[Initial State] This system is in a typical operational state with all
    components and services running.
  \item[Input/Condition] Turn down the primary signaling server.
  \item[Output/Result] The redundant server takes over until the primary server
    resumes.
  \item[How Test Will Be Performed] Test will be done manually by developers
    running the system. Tester will intentionally turn down the running primary
    signaling server by introducing a controlled fault. Tester shall monitor the
    system’s behaviour during the failover process by observing the annotation
    delay, video stream quality and system response to user interactions. The
    test is considered successful if the redundant server takes over within an
    acceptable time frame, the system continues to operate without any service
    interruptions, and the system is able to transition back to using the
    primary server smoothly when restored.
  \end{description}
\item \label{NFRT10}
  \begin{description}
  \item[Type] Structural, Dynamic, Manual
  \item[Initial State] The system is in a typical operational state with all
    components and services running. The user is in a live session with a 
    media\rt{-}capturing device plugged in.
  \item[Input/Condition] Disconnect the media\rt{-}capturing device from the user’s
    computing device.
  \item[Output/Result] A pop\rt{-}up window which warns the user that the client
    application has lost connection to the media\rt{-}capturing device and prompts
    the user to reconnect the device to resume the live session.
  \item[How Test Will Be Performed] The test will be done manually by developers
    running the system. A tester will make sure a functional media\rt{-}capturing
    device is securely connected to the computer. The tester will then run the
    instructor\rt{-}client application on the computer, create a live session, and
    make sure that the application receives and is able to send media streams
    from the capturing device. The tester will proceed to unplug the media
    capturing device from the computer. The tester shall monitor the client
    application and look for any pop up warnings. This test shall be repeated
    for \sout{text}\rt{TEST\_ACCOUNTS\_NUMBERS} times. The test is considered successful if a pop up warning about
    media\rt{-}capturing devices being disconnected is displayed every time the media
    capturing device is disconnected. Otherwise, if no pop up warning is
    displayed, the test is considered a failure.
  \end{description}
\item \label{NFRT11}
  \begin{description}
  \item[Type] Structural, Dynamic, Manual
  \item[Initial State] The system is in a typical operational state with all
    components and services running.
  \item[Input/Condition] Connect a media\rt{-}capturing device to the user’s computing
    device.
  \item[Output/Result] A pop up window which warns the user that the video
    capturing device does not meet \rt{the} required specification\rt{s} if the client
    application detects that the video input stream has a resolution lower than
    MIN\_RES.
  \item[How Test Will Be Performed] The test will be done manually by developers
    running the system. A tester shall prepare media\rt{-}capturing devices of
    various specifications. The tester shall make sure that at least one of them
    captures video at a resolution below MIN\_RES, at least one at
    MIN\_RES, and at least one higher than MIN\_RES. The tester
    shall connect each media\rt{-}capturing device to the client application. The
    test is considered successful if a pop up warning about video input not
    meeting the minimum required resolution is displayed whenever a capturing
    device with \rt{a} resolution below MIN\_RES. Otherwise, the test is
    considered a failure.
  \end{description}
\item \label{NFRT12}
  \begin{description}
  \item[Type] Structural, Dynamic, Manual
  \item[Initial State] The system is in a typical operational state with all
    components and services running.
  \item[Input/Condition] \sout{t}\rt{T}he user starts a live session and begins streaming.
  \item[Output/Result] Accurate instructional annotations are rendered and
    overlaid on the instructor’s video stream.
  \item[How Test Will Be Performed] The test will be done manually by developers
    running the system. The tester will run the instructor\rt{-}client application on
    a computer, create a live session, and make sure that the application
    receives and is able to send media streams from the capturing device. The
    tester shall then monitor the annotations produced and rendered by the
    annotating machine learning pipeline, export samples of the results into
    local media files, and ask other developers in the team to review and
    evaluate the annotated videos. The test is considered successful if 4 out of
    5 developers are satisfied with the result. Otherwise, the test is
    considered a failure.
  \end{description}
\item \label{NFRT13}
  \begin{description}
  \item[Type] Structural, Dynamic, Manual
  \item[Initial State] The system is in a typical operational state with all
    components and services running.
  \item[Input/Condition] Connect more than one media\rt{-}capturing device to the
    user’s computing device.
  \item[Output/Result] A \sout{dialog}\rt{dialogue} listing all the detected capturing devices is
    displayed to prompt the user \sout{for selecting}\rt{to select} a device to use.
  \item[How Test Will Be Performed] The test will be done manually by developers
    running the system. A tester shall prepare media\rt{-}capturing devices of
    various types. The tester shall make sure that all prepared capturing
    devices are functional and establish secure connections to the computer
    running the client application. The test is considered successful if all
    media\rt{-}capturing devices are listed and selectable in a \sout{dialog}\rt{dialogue} in the client
    application. Upon selecting any of the listed devices, the client
    application would switch to the selected device and fetch media streams from
    the device. Failing \sout{in completing}\rt{to complete} any of the described interactions would be
    considered a failure of this test case.
  \end{description}
\item \label{NFRT14}
  \begin{description}
  \item[Type] Structural, Dynamic, Manual
  \item[Initial State] This system is in a typical operational state with all
    components and services running, with a live session created.
  \item[Input/Condition] The live video and audio stream from instructor\rt{-}client
    application.
  \item[Output/Result] Audio stream and video stream with annotations on
    practitioner\rt{-}client application.
  \item[How Test Will Be Performed] Test will be done manually by developers
    running the system. A tester will run the instructor\rt{-}client application,
    create a live session, and start streaming using media\rt{-}capturing devices for
    30 minutes. User performance monitoring tools to measure and record the
    delay between the video stream frames and annotation frames. The test will
    be repeated \sout{for} at least \sout{5}\rt{TEST\_ACCOUNTS\_NUMBERS} times under different system loads. The test is
    considered successful if the recorded delay remains under MAX\_DELAY
    through each live session. Any deviations beyond MAX\_DELAY will be
    considered a failure.
  \end{description}
\item \label{NFRT15}
  \begin{description}
  \item[Type] Structural, Static, Manual
  \item[Initial State] The system is in a typical operational state with all
    components and services running.
  \item[Input/Condition] The user starts the client application for instructors.
  \item[Output/Result] A message with detailed instruction\rt{s} to set up a media
    capturing device and a list of cautions is displayed. The user is informed
    that for generating accurate annotations, their full body must be within the
    field of view from the media\rt{-}capturing device.
  \item[How Test Will Be Performed] The test will be done manually by developers
    running the system. A tester launches the client application for instructors
    and looks for any instructional messages or cautions. The test is considered
    successful if the tester is able to properly set up the media\rt{-}capturing
    device by strictly following the instructions displayed, such that their
    entire body is visible from the perspective of the capturing device.
    Otherwise, if any problem or confusion occurs during the setup process, the
    test is considered a failure.
  \end{description}
\item \label{NFRT16}
  \begin{description}
  \item[Type] Structural, Dynamic, Manual
  \item[Initial State] The software system is fully developed, stable, and ready
    for testing. The different test environments for the latest versions of
    Windows, Linux, and macOS are set up, each with default settings.
  \item[Input/Condition] The latest stable versions of Windows, Linux, and
    macOS. Test cases \rt{are} designed to cover all the main functionalities of the
    system.
  \item[Output/Result] The system functions correctly on all mentioned operating
    systems without crashes, unexpected behaviour, or significant performance
    issues.
  \item[How Test Will Be Performed] Set up test environments with the latest
    versions of Windows, Linux, and macOS, ensuring they meet the system
    requirements for the software. Prepare a suite of test cases that cover the
    full range of the system's functionality. These should include common user
    tasks and interactions with the system. Install or deploy the system in each
    test environment. Execute the suite of test cases in each operating system
    environment, documenting any discrepancies in functionality, performance, or
    user interface issues specific to each platform. Compare results across the
    different operating systems to ensure that features and functionalities are
    consistent. Validate that performance is consistent across different
    operating systems and that there are no platform-specific lags or speed
    issues.
  \end{description}
\item \label{NFRT17}
  \begin{description}
  \item[Type] Structural, Dynamic, Manual
  \item[Initial State] The system is fully developed, with all features
    operational, and is hosted in a stable environment accessible via the web.
    Test environments with the latest versions of Chrome, Firefox, Safari, and
    Edge are established, each configured with default browser settings.
  \item[Input/Condition] Latest stable versions of Chrome, Firefox, Safari, and
    Edge. A series of test cases \rt{are} designed to fully evaluate the functionalities
    and features of the system.
  \item[Output/Result] The system operates as intended on all tested web
    browsers without significant functionality issues, crashes, or severe
    performance degradation. Features and visual elements are consistent across
    all browsers.
  \item[How Test Will Be Performed] Prepare detailed test cases covering all
    aspects of system functionality, including user interface, input fields,
    navigation, performance, and security aspects. Ensure the availability of
    the latest versions of each browser and that each test environment reflects
    the end-user's setting as closely as possible. In each browser, manually
    execute all test cases, performing all the actions an end-user would do.
    This includes testing visual elements, interactive components, and backend
    functionality (like form submission, data processing, etc.) for consistency.
    Compare results for each browser to check for consistency in feature
    behaviour and data processing. Validate response times and performance
    metrics to ensure there are no browser-specific lags or speed issues.
  \end{description}
\item \label{NFRT18}
  \begin{description}
  \item[Type] Structural, Dynamic, Manual
  \item[Initial State] The system is fully developed and operational. The
    testing environments represent a range of standard personal computer and
    laptop configurations (covering various manufacturers, system
    specifications, and age of devices) equipped with cameras and, where
    applicable, microphones. The devices are running compatible operating
    systems with the necessary drivers installed.
  \item[Input/Condition] Range of standard computers or laptops with various
    specifications, but all within the commonly accepted 'standard' range for
    current users. Devices have functional cameras and optional microphones.
    Test script detailing system operation tasks.
  \item[Output/Result] \sout{The system operates without significant delays, errors,
  or crashes.}\rt{The system operates with delays not exceeding\\ MAX\_ACCEPTABLE\_DELAY, 
  error rates below MAX\_ERROR\_RATE, and without system crashes over a continuous 
  operational period of\\ MIN\_OPERATIONAL\_PERIOD.}
  \item[How Test Will Be Performed] Identify a range of standard personal
    computers and laptops that reflect the typical user base's hardware. This
    range should include variations in processor speed, memory, age, and
    manufacturer. Ensure that each test machine is equipped with a functional
    camera and, if necessary for the system's functionality, a microphone.
    Install or access the system on each device, ensuring proper installation of
    peripheral drivers and any necessary system software. Prepare a test script
    that encompasses everyday tasks users would perform on the system, ensuring
    it thoroughly tests interaction with camera and microphone functions.
    Execute the test script on each device, interacting with the system as an
    ordinary user would. This process should include tasks that specifically
    utilize the camera and microphone to test the system’s handling of these
    resources. Observe and document system performance, noting any lag,
    stuttering, quality issues with audio/video, crashes, or errors. Review the
    quality of the audio and video captured or utilized by the system, checking
    for any delays, synchronization issues, or loss of quality.
  \end{description}
\item \label{NFRT19}
  \begin{description}
  \item[Type] Structural, Static, Manual
  \item[Initial State] The system is fully developed, with comprehensive
    documentation on its architecture, codebase, dependencies, and update
    procedures. The development environment is available for testing maintenance
    procedures, including tools for version control, testing, and deployment.
  \item[Input/Condition] Simulated or real maintenance tasks.
  \item[Output/Result] Successful completion of maintenance tasks without
    introducing significant issues or disruptions to the system's operation.
  \item[How Test Will Be Performed] Begin with a comprehensive review of the
    system documentation, focusing on the clarity, completeness, and accuracy of
    information related to system architecture, dependencies, and procedures for
    updates and maintenance tasks. Analyze the system's codebase for adherence
    to coding standards, modularity, use of dependencies, commenting, and
    documentation within the code. Identify any areas of technical debt,
    outdated dependencies, or overly complex structures that could complicate
    maintenance efforts. Simulate various maintenance scenarios, such as
    applying an update, fixing a bug, or adding a new feature. This process
    should cover a range of complexity and potential impact on different parts
    of the system. Monitor the ease with which these tasks can be completed, the
    risk of introducing errors, and the effectiveness of the documentation and
    procedures. Test the system’s ability to roll back updates by simulating a
    scenario where an update introduces a significant issue. Evaluate the
    effectiveness and ease of reversing changes without losing data or causing
    system downtime.
  \end{description}
\item \label{NFRT20}
  \begin{description}
  \item[\sout{Type}] \sout{Structural, Dynamic, Manual}
  \item[\sout{Initial State}] \sout{The system is fully operational, and a maintenance/update
    plan is ready, detailing the procedures to be followed. A stable pre-update
    version of the system is running, and a controlled environment is available
    to simulate the maintenance process without affecting live users.}
  \item[\sout{Input/Condition}] \sout{Tools and resources necessary for the maintenance task
    (e.g., update packages, scripts, database migrations, etc.).}
  \item[\sout{Output/Result}] \sout{The total time taken for the procedure is recorded, from
    initiation to completion, including the system being fully operational
    post-update.}
  \item[\sout{How Test Will Be Performed}] \sout{Prepare the test environment to match the
    live system as closely as possible in terms of data, configurations, and
    system dependencies. Ready the maintenance/update package along with
    detailed procedures and any scripts or commands necessary for the process.
    Begin the timed maintenance process, initiating the update procedure as
    documented. A timer or time log should be used to record the process's
    duration accurately. Perform the steps required for the update, which may
    include system shutdown, backup, application of update packages, database
    migrations, system configuration adjustments, and restart procedures.
    Document each step, noting the time taken and any complications or
    deviations from the expected process. Once the update steps are completed,
    perform a comprehensive system check to verify that all functionalities are
    operational and that the update has been applied successfully. Analyze the
    outcomes of the update, including any improvements or regressions in system
    performance. Review the time log to calculate the total duration of the
    maintenance/update process. This calculation should encompass all stages,
    from the initial shutdown to the restoration of full functionality.If the
    process took longer than the specified 4-hour window, identify the stages
    that consumed more time than expected and analyze the reasons for this
    delay.}
  \end{description}
\item \label{NFRT21}
  \begin{description}
  \item[\sout{Type}] \sout{Structural, Dynamic, Automated}
  \item[\sout{Initial State}] \sout{The system is fully operational, and user accounts
    containing mock personal information are available for testing. Security
    measures, such as encryption protocols and compliance measures, are in
    place.}
  \item[\sout{Input/Condition}] \sout{Automated security testing tools or penetration testing
    tools for vulnerability scanning. Test scripts \rt{are} designed to simulate typical
    user interactions, including data input and retrieval.}
  \item[\sout{Output/Result}] \sout{Confirmation that stored data, especially personal
    information\sout{,} is encrypted or securely hashed within the database or storage
    system.}
  \item[\sout{How Test Will Be Performed}] \sout{Use network monitoring tools to capture the
    data being transmitted during the test sessions. This activity might include
    creating user accounts, logging in, or transmitting personal information.
    Analyze the captured data to verify that it's encrypted and that sensitive
    information isn't exposed in plaintext or easily decipherable formats. Then
    access the storage system (database, file storage) and inspect how user
    data, particularly personal information\sout{,} is stored. Verify that sensitive
    data is encrypted or hashed, making it unreadable without proper
    authorization or decryption keys. Implement automated security tools to scan
    for vulnerabilities that could be exploited to gain unauthorized access to
    sensitive data. These might include outdated software, misconfigurations, or
    known vulnerabilities within the system components. Document all findings
    from the security tests, including any potential vulnerabilities identified,
    successful and failed breach attempts, and the security status of data
    transmission and storage.}
  \end{description}
\item \label{NFRT22}
  \begin{description}
  \item[Type] Structural, Static, Manual
  \item[Initial State] The system is in a stable state, ready for testing. The
    user interface for account creation or any other information input is
    accessible, and documentation related to data handling is available.
  \item[Input/Condition] Guidelines or criteria specifying what constitutes
    "essential" versus "nonessential" information for the system’s purpose.
    Access to the system's user interfaces (UI) where data submission forms are
    present.
  \item[Output/Result] Comparison of requested data against the criteria for
    essential information.
  \item[How Test Will Be Performed] Examine all user interfaces (UIs) that
    prompt data entry forms used during regular operations. Based on the
    system’s purpose and functionality, classify the information that the system
    needs to operate effectively. Define what is considered "essential"
    information that the system cannot function without. Identify any
    information that may be deemed "non-essential," meaning that its collection
    is not justified based on the system’s core functions. Compare the data
    collected from \rt{the} entry with the classifications. Highlight any data fields
    that appear to be non-essential for the system’s operation.
  \end{description}
\item \label{NFRT23}
  \begin{description}
  \item[Type] Structural, Dynamic, Manual
  \item[Initial State] This system is in a typical operational state with all
    components and services running.
  \item[Input/Condition] Attempted unauthorized modifications to the system
    including access to system files and modifications to system configurations.
  \item[Output/Result] A determination of whether the system are able to prevent
    access or modifications from unauthorized users.
  \item[How Test Will Be Performed] Test will be done manually by developers
    running the system. Tester will simulate an unauthorized user of the system
    and make the following attempts:
    \begin{itemize}
    \item Access system settings.
    \item Modify Critical system configuration files.
    \item Gain access to media streams in the system.
    \end{itemize}
    The test will be performed on both the instructor\rt{-}client application and
    practitioner\rt{-}client application. The test is considered successful if all
    attempts by the unauthorized user are prevented by the system. The test will
    be considered a failure if any successful unauthorized access or
    modification to the system \sout{were}\rt{is} made during the testing process.
  \end{description}
\item \label{NFRT24}
  \begin{description}
  \item[Type] Structural, Dynamic, Manual
  \item[Initial State] The system is in a typical operational state with all
    components and services running.
  \item[Input/Condition] The user starts a live session and begins streaming.
  \item[Output/Result] A prompt for granting access to use the media\rt{-}capturing
    devices is displayed.
  \item[How Test Will Be Performed] Test will be done manually by developers
    running the system. A tester will make sure a functional media\rt{-}capturing
    device is securely connected to the computer. The tester will then run the
    instructor\rt{-}client application on the computer. Upon creating a live session,
    the tester shall look for a prompt and they are able to consent to giving
    the client application access to the media\rt{-}capturing device. If the video
    stream starts, or the media stream from the capturing device is displayed on
    the screen before the tester’s consent, the test is considered a failure.
    Otherwise, the test is considered successful.
  \end{description}
\item \label{NFRT25}
  \begin{description}
  \item[Type] Structural, Dynamic, Manual
  \item[Initial State] The system is in a typical operational state with all
    components and services running.
  \item[Input/Condition] The user starts a live session and begins streaming.
  \item[Output/Result] An indicator that the media\rt{-}capturing device is in use
    and that the user is being recorded is visible in the client application.
  \item[How Test Will Be Performed] Test will be done manually by developers
    running the system. A tester will make sure a functional media\rt{-}capturing
    device is securely connected to the computer. The tester will then run the
    instructor\rt{-}client application on the computer, create a live session, and
    make sure that the application receives and is able to send media streams
    from the capturing device for 30 minutes. The tester shall look for an
    indicator that the media\rt{-}capturing device is in use. If such an indicator
    cannot be found, or if the indicator disappears at any moment of time during
    the whole testing process, the test is considered a failure. Otherwise, the
    test is considered successful.
  \end{description}
\item \label{NFRT26}
  \begin{description}
  \item[Type] Structural, Dynamic, Manual
  \item[Initial State] The system is in a typical operational state with all
    components and services running.
  \item[Input/Condition] The user ends a live session and stops streaming.
  \item[Output/Result] The indicator that the media\rt{-}capturing device is in use
    disappears. The media stream from the capturing device is no longer
    displayed on the screen.
  \item[How Test Will Be Performed] Test will be done manually by developers
    running the system. A tester will make sure a functional media\rt{-}capturing
    device is securely connected to the computer. The tester will then run the
    instructor\rt{-}client application on the computer, create a live session, and
    make sure that the application receives and is able to send media streams
    from the capturing device. The tester shall then manually end the live
    stream session. The tester shall look for the indicator that the media
    capturing device is in use. If such an indicator is no longer visible and
    the media stream from the capturing device is not displayed on screen, the
    test is considered successful. Otherwise, the test is considered a failure.
  \end{description}
\item \label{NFRT27}
  \begin{description}
  \item[Type] Structural, Static, Manual
  \item[Initial State] This system is in a typical operational state with all
    components and services running.
  \item[Input/Condition] User inputs
  \item[Output/Result] System response to user inputs
  \item[How Test Will Be Performed] Test will be done manually by developers
    running the system. Testers will perform various user interactions that
    cover all possible use cases of the system and inspect if any offensive
    cultural symbols/language are displayed on \rt{the} UI. The test is considered
    successful if all testers agree that there is no offensive cultural
    symbol/language displayed through the testing process.
  \end{description}
\item \label{NFRT28}
  \begin{description}
  \item[Type] Structural, Static, Manual
  \item[Initial State] The system is in a typical operational state with all
    components and services running.
  \item[Input/Condition] A set of ISO/IEC 12207 standards.
  \item[Output/Result] A determination of whether the system and its associated
    processes, documentation, and activities comply with the ISO/IEC 12207
    standards.
  \item[How Test Will Be Performed] Test will be performed by: 1. Testers
    carefully review all project documentation, including software development
    processes, design documentation\sout{s}, coding standards and testing plans to
    ensure they all align with the ISO/IEC 12207 standards. 2. Testers examine
    the processes and activities involved in the software development life cycle
    to determine if they adhere to the ISO/IEC 12207 standards. The test is
    considered successful if all testers agree that the system and its
    associated processes, documentation, and activities comply with the ISO/IEC
    12207 standards.
  \end{description}
\item \label{NFRT29}
  \begin{description}
  \item[Type] Structural, Dynamic, Manual
  \item[Initial State] This system is in a typical operational state with all
    components and services running.
  \item[Input/Condition] Various user interactions.
  \item[Output/Result] The computers continue to function within acceptable
    performance parameters throughout the test.
  \item[How Test Will Be Performed] Test will be done manually by developers
    running the system. Testers will perform various user interactions, such as
    button clicks and menu selections, and joining/creating live sessions. The
    input user interactions shall cover all possible use cases of the system.
    Repeat the test on different computing devices. The tester shall monitor the
    CPU temperature of the computer that the system is running on. The test is
    considered successful if the CPU temperature of the computer that the system
    is running on remains under MAX\_TEMP through the test. Otherwise,
    the test is considered a failure.
  \end{description}
\item \label{NFRT30}
  \begin{description}
  \item[Type] Structural, Static, Manual
  \item[Initial State] This system is in a typical operational state with all
    components and services running.
  \item[Input/Condition] User inputs
  \item[Output/Result] System response from user inputs
  \item[How Test Will Be Performed] Test will be done manually by developers
    running the system. Testers will perform various user interactions, such as
    button clicks and menu selections, and joining/creating live sessions.
    Testers will use the system as Tai Chi instructors and Tai Chi practitioners
    and evaluate the usability and user experience of the system. The test is
    considered successful if all testers agree that the system does not affect
    their physical or mental health. Otherwise, if any tester states that the
    system harms their physical or mental health, the test is considered a
    failure.
  \end{description}
\end{enumerate}

\subsection{Traceability Between Test Cases and Requirements}

Please refer to Table \ref{tab:frt}, Table \ref{tab:nfrt1}, and Table \ref{tab:nfrt2}.

\setlength{\tabcolsep}{2pt}
\newgeometry{margin=2cm}
\begin{landscape}
  \begin{table}[h!]
    \centering
    \begin{tabular}{|c|c|c|c|c|c|c|c|c|c|c|c|c|c|c|c|c|} \hline
               & FR1 & FR2 & FR3 & FR4 & FR5 & FR6 & FR7 & FR8 & FR9 & FR10 & FR11 & FR12 & FR13 & FR14 & FR15 & FR16 \\ \hline
      \ref{FRT1}  & X   & X   &     &     &     &     &     &     &     &      &      &      &      &      &      &      \\ \hline
      \ref{FRT2}  &     &     & X   &     &     &     &     &     &     &      &      &      &      &      &      & X    \\ \hline
      \ref{FRT3}  &     &     &     & X   & X   &     &     &     &     &      &      &      &      &      &      &      \\ \hline
      \ref{FRT4}  &     &     &     &     &     & X   &     &     &     &      &      &      &      &      & X    &      \\ \hline
      \ref{FRT5}  &     &     &     &     &     &     &     & X   &     &      &      &      &      &      &      &      \\ \hline
      \ref{FRT6}  &     &     &     &     &     &     & X   &     &     &      &      &      &      &      &      &      \\ \hline
      \ref{FRT7}  &     &     &     &     &     &     &     &     & X   &      &      &      &      &      &      &      \\ \hline
      \ref{FRT8}  &     &     &     &     &     &     &     &     &     & X    &      &      &      &      &      &      \\ \hline
      \ref{FRT9}  &     &     &     &     &     &     &     &     &     &      & X    &      &      &      &      &      \\ \hline
      \ref{FRT10} &     &     &     &     &     &     &     &     &     &      &      & X    &      &      &      &      \\ \hline
      \ref{FRT11} &     &     &     &     &     &     &     &     &     &      &      &      & X    &      &      &      \\ \hline
      \ref{FRT12} &     &     &     &     &     &     &     &     &     &      &      &      &      & X    &      &      \\ \hline
    \end{tabular}
    \caption{Traceability matrix showing the connections between test cases
      and functional requirements}
    \label{tab:frt}
  \end{table}
  \begin{table}[h!]
    \centering
    \begin{tabular}{|c|c|c|c|c|c|c|c|c|c|c|c|c|c|c|c|c|} \hline
                & LF1 & UH1 & UH2 & PR1 & PR2 & PR3 & PR4 & PR5 & PR6 & PR7 & PR8 & PR9 & PR10 & PR11 & PR12 & PR13 \\ \hline
      \ref{NFRT1}  & X   & X   & X   &     &     &     &     &     &     &     &     &     &      &      &      &      \\ \hline
      \ref{NFRT2}  &     &     &     & X   & X   &     &     &     &     &     &     &     &      &      &      &      \\ \hline
      \ref{NFRT3}  &     &     &     &     & X   & X   &     &     &     &     &     &     &      &      &      &      \\ \hline
      \ref{NFRT4}  &     &     &     &     &     &     & X   &     &     &     &     &     &      &      &      &      \\ \hline
      \ref{NFRT5}  &     &     &     &     &     &     &     & X   &     &     &     &     &      &      &      &      \\ \hline
      \ref{NFRT6}  &     &     &     &     &     &     &     &     & X   &     &     &     &      &      &      &      \\ \hline
      \ref{NFRT7}  &     &     &     &     &     &     &     &     &     & X   &     &     &      &      &      &      \\ \hline
      \ref{NFRT8}  &     &     &     &     &     &     &     &     &     &     & X   &     &      &      &      &      \\ \hline
      \ref{NFRT9}  &     &     &     &     &     &     &     &     &     &     &     & X   &      &      &      &      \\ \hline
      \ref{NFRT10} &     &     &     &     &     &     &     &     &     &     &     &     & X    &      &      &      \\ \hline
      \ref{NFRT11} &     &     &     &     &     &     &     &     &     &     &     &     &      & X    &      &      \\ \hline
      \ref{NFRT12} &     &     &     &     &     &     &     &     &     &     &     &     &      &      & X    &      \\ \hline
      \ref{NFRT13} &     &     &     &     &     &     &     &     &     &     &     &     &      &      &      & X    \\ \hline
    \end{tabular}
    \caption{Traceability matrix showing the connections between test cases
      and non-functional requirements}
    \label{tab:nfrt1}
  \end{table}
  \begin{table}[h!]
    \centering
    \begin{tabular}{|c|c|c|c|c|c|c|c|c|c|c|c|c|c|c|c|c|c|} \hline
                & PR14 & PR15 & OE1 & OE2 & OE3 & MS1 & MS2 & SR1 & SR2 & SR3 & SR4 & SR5 & SR6 & CR1 & LR1 & HS1 & HS2 \\ \hline
      \ref{NFRT14} & X    &      &     &     &     &     &     &     &     &     &     &     &     &     &     &     &     \\ \hline
      \ref{NFRT15} &      & X    &     &     &     &     &     &     &     &     &     &     &     &     &     &     &     \\ \hline
      \ref{NFRT16} &      &      & X   &     &     &     &     &     &     &     &     &     &     &     &     &     &     \\ \hline
      \ref{NFRT17} &      &      &     & X   &     &     &     &     &     &     &     &     &     &     &     &     &     \\ \hline
      \ref{NFRT18} &      &      &     &     & X   &     &     &     &     &     &     &     &     &     &     &     &     \\ \hline
      \ref{NFRT19} &      &      &     &     &     & X   &     &     &     &     &     &     &     &     &     &     &     \\ \hline
      \ref{NFRT20} &      &      &     &     &     &     & X   &     &     &     &     &     &     &     &     &     &     \\ \hline
      \ref{NFRT21} &      &      &     &     &     &     &     & X   &     &     &     &     &     &     &     &     &     \\ \hline
      \ref{NFRT22} &      &      &     &     &     &     &     &     & X   &     &     &     &     &     &     &     &     \\ \hline
      \ref{NFRT23} &      &      &     &     &     &     &     &     &     & X   &     &     &     &     &     &     &     \\ \hline
      \ref{NFRT24} &      &      &     &     &     &     &     &     &     &     & X   &     &     &     &     &     &     \\ \hline
      \ref{NFRT25} &      &      &     &     &     &     &     &     &     &     &     & X   &     &     &     &     &     \\ \hline
      \ref{NFRT26} &      &      &     &     &     &     &     &     &     &     &     &     & X   &     &     &     &     \\ \hline
      \ref{NFRT27} &      &      &     &     &     &     &     &     &     &     &     &     &     & X   &     &     &     \\ \hline
      \ref{NFRT28} &      &      &     &     &     &     &     &     &     &     &     &     &     &     & X   &     &     \\ \hline
      \ref{NFRT29} &      &      &     &     &     &     &     &     &     &     &     &     &     &     &     & X   &     \\ \hline
      \ref{NFRT30} &      &      &     &     &     &     &     &     &     &     &     &     &     &     &     &     & X   \\ \hline
    \end{tabular}
    \caption{Traceability matrix showing the connections between test cases
      and non-functional requirements continued}
    \label{tab:nfrt2}
  \end{table}
\end{landscape}
\restoregeometry

\section{Unit Test Description}

This section provides a detailed description of the guidelines for conducting
unit testing of the system.

\subsection{Unit Testing Scope}

The primary focus of the unit tests in this project will be on assessing the
functionality of individual components. To ensure extensive coverage, test
coverage metrics will be employed to quantify and oversee testing activities.
Integration of unit tests into the Continuous Integration (CI) pipeline will
guarantee that new modifications do not disrupt existing functional components.

The following modules are considered out of scope for unit testing:
\begin{description}
\item[ML Model Accuracy] This exclusion is due to the decision to employ an
  external model.
\item[Client Code Accessing User's Device Hardware] This aspect will be managed by
  the WebRTC framework, thus exempting it from unit testing.
\item[Network Connections between Components] Testing of network connections is
  beyond the scope of unit testing.

\end{description}
More specifically, the majority of non-functional requirements exclude
verification via unit testing. Non-functional testing methodologies tend to
qualify as performance testing and preclude unit testing validation.

All the parts that are not out of scope for testing should be unit\rt{-}tested.

\newpage

\section{Appendix}

\subsection{Usability Survey Questions}
\label{sec:survey}

\rt{Rating scales are defined as:}
\begin{description}
  \item[\rt{1}] \rt{Well below expectations}
  \item[\rt{2}] \rt{Below expectations}
  \item[\rt{3}] \rt{Met expectations}
  \item[\rt{4}] \rt{Above expectations}
  \item[\rt{5}] \rt{Beyond expectations}
\end{description}

Questions for the user experience survey:

\begin{enumerate}
\item On a scale of 1 to 5 \sout{(beyond expectation)}, how would you rate your overall
  experience with our Tai Chi video conferencing application?
\item On a scale of 1 to 5 \sout{(beyond expectation)}, how would you rate the experience
  in reading the in-app instructions and navigating through the application?
\item On a scale of 1 to 5 \sout{(beyond expectation)}, how would you rate the video
  conference quality, considering the clarity, latency, and the accuracy of
  annotation overlays?
\item On a scale of 1 to 5 \sout{(beyond expectation)}, how would you rate your overall
  Tai Chi learning experience with the aid of the application?
\item Are there any specific issues or challenges you've encountered while using
  the application?
\item How satisfied are you with the speed and responsiveness of the application
  during your Tai Chi sessions?
\item Do you find that the annotations and real-time guidance provided by the
  application enhance your Tai Chi learning experience?
\item Have you experienced any disruptions or crashes during your sessions? If yes,
  please describe.
\item Have you ever received any notifications or warnings related to the security
  of your sessions, e.g., when media-capturing devices are accessed?
\item Have you received sufficient guidance on using the application effectively
  for your Tai Chi practice?
\item Are there any specific content or learning materials you would like to see
  added to the application?
\end{enumerate}

\newpage{}
\section*{Appendix --- Reflection}

\paragraph*{Anhao}
I am responsible for coming up with the test plan for functional requirements.
During the test plan development, I learned about various of testing methods,
such as Dynamic testing - Black-box testing - Functional testing, Unit testing
and Integration testing. I had past experience with others but integration
testing. As integration testing involves evaluating the interaction and
interoperability of different system components, I need to learn more about how
to conduct integration testing on our system.

To learn more about integration testing, I will start by understanding its
fundamental concepts, including assessing interactions between system
components. Identify integration points, define test scenarios that mimic
real-world usage, and choose appropriate testing techniques. Develop detailed
test cases and set up a test environment mirroring the production setup. Execute
tests, monitor interactions, and assess system behaviour. Analyze results,
document findings, and address integration issues, retesting scenarios after
resolution. After knowing the details of integration tests, then I can apply it
effectively to ensure component interactions in the system work harmoniously.

\paragraph*{Xunzhou}
My primary focus while composing this verification and validation plan is to
ensure that the product we develop meets the performance and usability
requirements. The overall user experience of a software product is significantly
influenced by its performance and usability. Planning out test cases under
various use case scenarios provides us with an opportunity to view the product
from a different perspective and critically analyze all the ways in which the
system might fail.

The other part of the document for which I am responsible is the software
validation plan. The key lesson I've learned from crafting this section is the
crucial distinction between software validation and software verification.
Verification is the process of confirming whether the software is being
developed correctly and whether it adheres to its specified design and
development requirements. In contrast, validation is the process of evaluating
the final software product to ensure that it fulfills its intended purpose and
meets the needs of the end-users. Keeping this distinction in mind, I have been
able to create a focused, feasible, and comprehensive plan for validating our
product.

Reflecting on the overall work completed in this document, the two portions I've
worked on are indeed closely related to each other. User satisfaction serves as
a solid indicator of valid software. A software application that performs well
and is easy to use is more likely to satisfy and retain users.

\paragraph*{Kehao}

I was primarily focused on the documentation aspect, which involved developing
the design verification plan, the V\&V plan verification plan and system tests
for non-functional requirements. These tasks exposed me to an opportunity to
gain insights into the in-depth details of quality assurance for a software
system.

While working on the system tests for non-functional requirements, I delved into
the performance and security aspects of our system. This was an enriching
learning experience as I had to envision various usage scenarios and
systematically design test cases to evaluate the system’s performance and
\sout{behavior}\rt{behaviour} under different conditions. It was during this process that I realized
the critical role that the system’s performance and security play in shaping the
overall user experience. Meanwhile, the development of the design verification
plan and the V\&V plan verification plan taught me valuable lessons about the
importance of thoroughly verifying that our software adheres to its intended
design and development requirements.

Reflecting on my work, I’ve realized the interconnectedness of these
documentation tasks. The design verification plan ensures that our system is
built according to the design, which, in turn, has a direct impact on the system
tests for non-functional requirements. Through the process of developing these
documents, I gained a deeper understanding of the importance of planning,
verification, and validation in delivering a software product \sout{aht}\rt{that} not only meets
design specifications but also satisfied the \sout{use-users}\rt{users} by performing effectively
and reliably.

To further enhance the related skills, I plan to take proactive steps in the
future. First, I will seek opportunities to engage in more real-world projects
and interact closely with the quality assurance team to acquire hands-on
experience. Also, I will explore relevant articles, papers and courses about
software testing and quality assurance in order to gain theoretical knowledge
and stay updated with the industry best practices.

\paragraph*{Qi}

I am tasked with developing the test plan for the functional requirements,
leveraging my proficiency in various functional testing tools. I have observed
that certain ambiguities within our requirements become apparent during the
formulation of the verification plan. This realization brings to mind the
concept of Test Driven Development, wherein creating unit tests prior to coding
offers a precise assessment of the actual implementation. In our context, this
practice enhances the clarity of our functional requirements, serving as a
guiding principle for our development process.

Throughout the formulation of this Verification and Validation (V\&V) plan, I
have come to understand the necessity of not only ensuring the correctness of
the code through testing but also devising strategies to validate whether our
requirements have been met. Additionally, establishing corresponding metrics is
vital for enhancing user experiences post-software development.

While I lack familiarity with certain tools specific to WebRTC testing, I view
this as an opportunity for learning and skill development. Regarding the tools
we have chosen, there is a wealth of online tutorials and forums available to
acquire the requisite knowledge and skills. Furthermore, the utilization of
generative AI proves to be a valuable resource when troubleshooting challenges.

\paragraph*{Qianlin}

In the process of our recent project documentation, my primary role was devising
testing methods for certain non-functional requirements. Given the intrinsic
characteristics of non-functional requirements, many of our tests leaned heavily
towards manual procedures and user input. This pivot towards manual processes
underscored the importance of physical testing sessions, ensuring our
application ticked all the boxes in terms of requirements. However, the
challenge wasn't just about conducting tests, but doing so efficiently and
effectively. This means considering aspects like identifying the right testing
protocols for post-development app testing. Moreover, I had a deeper
understanding of non-functional requirements testing through the documentation
procedure. While my foundational knowledge serves me well, there's a vast ocean
of methodologies and best practices that await exploration. Further, the troves
of open-source projects on platforms like GitHub offer another layer of
knowledge, especially when it comes to deciphering testing reports. Furthermore,
I'll dedicate time to analyze open-source projects for practical insights. Given
my role, I've decided to prioritize test planning by enrolling in a specialized
online tutorial, ensuring our testing processes are both thorough and efficient.

\end{document}
\documentclass{article}

\usepackage{booktabs}
\usepackage{tabularx}
\usepackage{soul}

% tabularx layouts https://tex.stackexchange.com/a/84401
\newcolumntype{l}{X}
\newcolumntype{m}{>{\hsize=.4\hsize}X}
\newcolumntype{s}{>{\hsize=.3\hsize}X}

\title{Development Plan\\\progname}

\author{\authname}

\date{25 September 2023}

%% Comments

\usepackage{color}

\newif\ifcomments\commentstrue %displays comments
%\newif\ifcomments\commentsfalse %so that comments do not display

\ifcomments
\newcommand{\authornote}[3]{\textcolor{#1}{[#3 ---#2]}}
\newcommand{\todo}[1]{\textcolor{red}{[TODO: #1]}}
\else
\newcommand{\authornote}[3]{}
\newcommand{\todo}[1]{}
\fi

\newcommand{\wss}[1]{\authornote{blue}{SS}{#1}} 
\newcommand{\plt}[1]{\authornote{magenta}{TPLT}{#1}} %For explanation of the template
\newcommand{\an}[1]{\authornote{cyan}{Author}{#1}}

%% Common Parts

\newcommand{\progname}{SFWRENG 4G06 Capstone Design Project} % PUT YOUR PROGRAM NAME HERE
\newcommand{\projname}{MotionMingle} % project name
\newcommand{\authname}{Team \#18, InfiniView-AI
\\ Anhao Jiao
\\ Kehao Huang
\\ Qianlin Chen
\\ Shu Qi
\\ Xunzhou Ye
} % AUTHOR NAMES

\usepackage{hyperref}
    \hypersetup{colorlinks=true, linkcolor=blue, citecolor=blue, filecolor=blue,
                urlcolor=blue, unicode=false}
    \urlstyle{same}


\begin{document}

\maketitle

\begin{table}[hp]
  \caption{Revision History} \label{TblRevisionHistory}
  \begin{tabularx}{\textwidth}{llX}
    \toprule
    \textbf{Date} & \textbf{Developer(s)} & \textbf{Change} \\
    \midrule
    25 September 2023 & AJ, KH, QC, QS, XY & Initial draft \\
    12 November 2023 & XY & Updated plan for PoC \\
    \bottomrule
  \end{tabularx}
\end{table}

The following document lists several aspects of our team's plan to approach and
implement the features for Project \projname. This plan includes meetings, means
of communication, workflows, proof of concepts, and deliverable timelines.

We are committed to following and executing the development plans outlined here.
We will work closely with both our supervisors and the course instructor. We
will maintain transparency and honesty in our work progress, and we will voice
any concerns and suggestions. If necessary, we will iterate on the scope of the
project and refine completed work and documentation to ensure the successful
completion of the capstone project.

\section{Team Meeting Plan}

\subsection{Supervisor Meetings}

\emph{Meetings detail below has been discussed and confirmed with the supervisors.}

\begin{description}
\item[Time] Wednesdays at 5:00 p.m.
\item[Length] 30 minutes to 1 hour
\item[Frequency] Once every two weeks
\item[Location] Prof. Rong Zheng's office in Information Technology Building (ITB)
  or a close-by meeting room if available.
\item[Attendance Expectation] All team members and both supervisors are present.
\end{description}

\subsubsection{Additional Meetings}

\emph{Out-of-schedule supervisor meetings could be arranged with Andrew Mitchell if
  needed.}

\begin{description}
\item[Time] Flexible time on any day other than Wednesday
\item[Length] 30 minutes to 1 hour
\item[Location] Microsoft Teams
\item[Attendance Expectation] All team members and Andrew Mitchell are present.
\end{description}

\subsection{Scheduled Team Meetings}

\subsubsection{Regular Lectures}

\begin{description}
\item[Time, Frequency, Location] As arranged by Prof. Spencer
\item[Length] 50 minutes
\item[Attendance Expectation] At least two team members are present.
\end{description}

\subsubsection{Meetings Outside Lectures}

\begin{description}
\item[Time] Sundays at 8:30 p.m., Wednesdays at 6:00 p.m. or right after the
  supervisor meeting
\item[Length] 1 hour
\item[Frequency] Every week
\item[Location] Discord
\item[Attendance Expectation] All team members are present.
\item[Sprint Review] Every other week on Sundays. Every team member take turns to
  report on what has been done.
\end{description}

\subsection{Meeting Details and Rules}

\subsubsection{Github Issues}

Each meeting is planned and documented in the form of a Github issue. The detail
of the issue (the title or the description) shall include the date, time,
location, agenda, and records of attendance. The meeting agenda shall be
prepared ahead of time, either done collaboratively at the end of the previous
meeting, or summarized by the note taker after the previous meeting. Meeting
agenda shall clearly outline all the topics to be discussed. All team members
are free to refine or add to the list of topics after the issue has be initially
created. Meeting attendance will be recorded by Markdown checkboxes. An issue
shall be closed after the meeting has taken place. Decisions made in a meeting
will be announced and documented in the form of comments under the meeting
issue.

\subsubsection{Absence and No-show Policy}

In case a team member is unable to attend a meeting, the member must inform the
team \emph{at least 1 hour before the meeting}. Notice of absence shall be
reported in the form of comments under the Github issue corresponds to the
missed meeting. Absence without an ahead-of-time notice counts as a no-show.

No-show in meetings shall be recorded and, as a result, will be reflected in the
team contribution evaluation factor. Frequent or regular absence of meetings,
even if the absence is being reported and documented, will also negatively
impact one member's team contribution factor. For a contribution factor on the
scale of 0 to 1.1, a tentative conversion from the number of absence or no-shows
is as follow:
\begin{itemize}
\item For any consecutive no-show, or 3 consecutive notified absence, take 0.1 off
  from the contribution factor;
\item For every 5 consecutive presence, add 0.1 if the contribution factor is lower
  than 1.
\end{itemize}

After a reported absence or no-show, the member is responsible for catching up
with the contents discussed in the missed meeting, by either going over the
meeting minutes at their free time, or by consulting other teammates.


\section{Team Communication Plan}

Microsoft Teams is used for online meetings with supervisors. A group chat on
Teams with all team members and both supervisors is used for asking questions
and sharing project resources, including links to related academic papers,
preliminary work done, technical documents of existing codes.

Discord is the primary platform for hosting online group meetings outside
lecture time. A dedicated Discord server is created for team communications for
the project. Discord server events is created for each online meeting as a
reminder for all team members. A Discord text channel is used for miscellaneous
communications like socializing, non-project related chats.

\section{Team Member Roles}

\begin{table}[hp]
  \caption{Team member roles} \label{TblTeamMemberRoles}
  \begin{tabularx}{1.0\linewidth}[h]{smml}
    \toprule
    \textbf{Member}            & \textbf{Role}                               & \textbf{Expertise}               & \textbf{Responsibility}                                                                      \\ \hline
    \midrule
    Xunzhou (Joe) Ye     & Developer, Co-host Leader, Note Taker & \LaTeX, Python             & Machine Learning Model, Pipeline Optimizations                                         \\ \hline
    Qianlin (Maris) Chen & Developer                             & \LaTeX, TypeScript, Python & UI/UX Design, Application Implementation, Full Stack Development                       \\ \hline
    Shu (Tommy) Qi       & Developer, Co-host Leader, Note Taker & TypeScript, Python         & Machine Learning Model, Integration of Machine Learning Models and Application Backend \\ \hline
    Kehao (Chris) Huang  & Developer                             & \LaTeX, Python, TypeScript & Machine Learning Model, Integration of Machine Learning Models and Application Backend \\ \hline
    Anhan Jiao           & Developer                             & \LaTeX, Python             & UI/UX Design, Application Implementation, Full Stack Development                       \\
    \bottomrule
  \end{tabularx}
\end{table}

\section{Workflow Plan}

We are committed to a typical git workflow in modern production environments.
There is one centralized git repository for the codebase of the project hosted
on Github. The main branch in the repository is where all functioning and tested
codes located. Tags on commits in the main branch will be used to indicate
snapshots of the repository for deliverables grading.

Relevant product features will be brought up and discussed in team meetings. The
implementation of features will be decomposed into subtasks and then assigned to
a team member. A Github issue labeled ``feature'' will be created for tracking
the progress of implementation and any feedback or discussion related to the
subtask. If a member encounters unexpected errors or behaviors outside the scope
of their work during development or testing, they should file an issue with a
``bug'' label. The issue will be discussed in the next meeting to determine the
owner of the bug fix. Regression tests shall be created following a close of bug
issue for documentation purposes. The Github Project board will be used to
manage and organize all the issues.

Following the creation of an issue, a git branch off from main will be created
and assigned to this issue. The owner of the issue take full ownership control
over the branch. Once the development for the issue conclude, the issue will
close following a successful merge of the dedicated branch to the main branch
and the deletion of the dedicated branch.

The main branch is protected. Pull request and code review are require to merge
changes into main. Approvals from \emph{at least two} other team members are
required to pass the review. Continuous Integration/Continuous Development
(CI/CD) in the form of automated Github actions is used to perform coding
standard checks, unit tests, comprehensive regression tests upon each pull
request. Successful execution of the actions listed above is also required for
merging into main.

A Github issue shall be opened for each deliverable. Equal contribution to the
written part of deliverables is assumed by default. Each team member shall
upload the draft of their parts to a shared Google Doc with minimal formatting.
The use of Google Doc is intended for easy collaborations and synchronous
editing in deliverable review meetings. Team members take turn to collect all
the pieces from the Google Doc and finalize the document with appropriate \LaTeX\
formatting. This member is also responsible for merging the document change into
the main branch and initiating a review for other members to check and agree on
the final version of the document for submission. Checklists inherited from the
capstone repository template shall be used to guide such review process.

\section{Proof of Concept Demonstration Plan}

\subsection{Plan for Demonstration in November}

A functioning application supporting real-time video conference. \st{and minimal
  annotations overlayed on one video stream. Such annotation could be either
  static or generated by the machine learning models from the preliminary work.}

\subsection{Most Significant Risk}

It might be the scope of the project being too large and the difficulty of
integrating annotations in the real-time communication application, which may
affect the overall timeline and functionality of the project. However, with
careful planning, coordination, and prioritization of tasks, these risks can be
mitigated to ensure the final output meets the desired goals and objectives. It
is important to manage expectations and communicate any potential limitations or
challenges to stakeholders to ensure a realistic understanding of what can be
achieved within the given constraints.

\subsection{Will a part of the implementation be difficult?}

The implementation of the real-time communication application with server render
architecture is difficult due to the complexity of integrating annotations. It
may require careful planning and coordination to ensure seamless integration
between the app and the ML model for generating annotations.

\subsection{Will testing be difficult?}

Yes, the difficulties can be summarized into the following two aspects:

\subsubsection{Testing the annotation}

\begin{itemize}
\item Ensuring the AI annotation performs well in different scenarios and with
  varying input data may require extensive testing and validation.
\item Evaluating the accuracy of the AI annotation may be challenging due to the
  subjective nature of some annotations. Manual verification and comparison with
  ground truth annotations may be required.
\item Assessing the robustness of the AI annotation against potential edge cases
  and outliers may pose difficulties in testing.
\end{itemize}

\subsubsection{Testing the real-time video streaming application}

\begin{itemize}
\item Ensuring the real-time video streaming functionality works seamlessly across
  different devices and network conditions can be challenging.
\item Testing the stability and reliability of the video streaming feature,
  especially under high traffic and bandwidth constraints, may pose
  difficulties.
\item Evaluating the synchronization of audio and video streams in real time can be
  complex and require specialized testing methods.
\item Assessing the performance of the video streaming feature, such as latency,
  buffering, and quality, may require extensive testing and monitoring.
\item Validating the compatibility of the video streaming feature with various
  platforms can be time-consuming and challenging.
\end{itemize}

\subsubsection{Is a required library difficult to install?}

There are no difficult libraries to install for this project. The required
libraries for the machine learning models and the application are mostly
open-source and easily accessible.

\subsubsection{Will portability be a concern?}

No. Since it is an application compatible with all platforms, it can be accessed
and used on any device.

\section{Technology}

\begin{description}
\item[Programming Language] Python, JavaScript/TypeScript
\item[Linter Tool] Flake8, ESlint
\item[Testing Framework] pytest, jest
\item[Code Coverage] pytest-cov, jest
\item[Version Control] git
\item[CI/CD] Github Actions
\item[Libraries] WebRTC, PyTorch
\item[Tools] Visual Studio Codes, Emacs, Google Doc, JetBrains IDEs
\end{description}

\section{Coding Standard}

\begin{itemize}
\item For the Python portion of the codebase, we will loosely follow the
  \href{https://peps.python.org/pep-0008/}{PEP 8 Python Style Guide}.
\item For the JavaScript/TypeScript portion of the codebase, we will loosely follow
  the \href{https://google.github.io/styleguide/jsguide.html}{Google JavaScript
    Style Guide}.
\item Snakecase with all capital letters shall be used exclusively for constants.
\item Adequate documentation and comments in codes is sufficient. Full
  documentation of all APIs is not required. However, high code readability is
  expected. Developers shall add comments if they found them helpful in
  improving readability.
\end{itemize}

\section{Project Scheduling}

\begin{itemize}
\item For each deliverable, there shall be at least one team meeting to breakdown
  the tasks and distribute the work to all team members.
\item Before each deliverable, there shall be at least one team meeting for all
  team members to review and finalize the document for submission. We aim to
  have at least 95\% of the deliverable done \emph{1 day before the deadline}.
\end{itemize}

\end{document}
\documentclass{article}

\usepackage{float}
\restylefloat{table}

\usepackage{booktabs}

\title{Team Contributions: Rev 0\\\progname}

\author{\authname}

\date{}

\input{../Comments}
%% Common Parts

\newcommand{\progname}{SFWRENG 4G06 Capstone Design Project} % PUT YOUR PROGRAM NAME HERE
\newcommand{\projname}{MotionMingle} % project name
\newcommand{\authname}{Team \#18, InfiniView-AI
\\ Anhao Jiao
\\ Kehao Huang
\\ Qianlin Chen
\\ Shu Qi
\\ Xunzhou Ye
} % AUTHOR NAMES

\usepackage{hyperref}
    \hypersetup{colorlinks=true, linkcolor=blue, citecolor=blue, filecolor=blue,
                urlcolor=blue, unicode=false}
    \urlstyle{same}


\begin{document}

\maketitle

\section{Demo Plans}

The team will showcase a working prototype of the Motion Mingle video conferencing platform to demonstrate the basic functionalities and user experience. The presented prototype is intended to include the following functionalities and behaviours:
\begin{itemize}
    \item A user interface for practitioner clients and instructor clients.
    \item The ability to broadcast the annotated video stream from the instructor to multiple practitioner clients.
    \item The ability to configure the desired annotation from each practitioner client.
    \item A SFU server that processes the video stream from the instructor, renders visual effects through machine learning pipelines, and broadcasts annotated video streams to practitioner clients according to their configuration.
\end{itemize}
\section{Meeting Attendance}

\wss{For each team member how many team meetings have they attended since the
POC demo.  This number should be determined from the meeting issues in the
team's repo.  The first entry in the table should be the total number of team
meetings held by the team.}

\begin{table}[H]
\centering
\begin{tabular}{ll}
\toprule
\textbf{Student} & \textbf{Meetings}\\
\midrule
Total & Num\\
Name 1 & Num\\
Name 2 & Num\\
Name 3 & Num\\
Name 4 & Num\\
Name 5 & Num\\
\bottomrule
\end{tabular}
\end{table}

\wss{If needed, an explanation for the counts can be provided here.}

\section{Lecture Attendance}

\wss{For each team member how many lectures have they attended since the POC
demo.  This number should be determined from the lecture issues in the team's
repo.  The first entry in the table should be the total number of lectures since
the POC demo.}

\begin{table}[H]
\centering
\begin{tabular}{ll}
\toprule
\textbf{Student} & \textbf{Lectures}\\
\midrule
Total & Num\\
Name 1 & Num\\
Name 2 & Num\\
Name 3 & Num\\
Name 4 & Num\\
Name 5 & Num\\
\bottomrule
\end{tabular}
\end{table}

\wss{If needed, an explanation for the lecture attendance can be provided here.}

\section{Commits}

\wss{For each team member how many commits to the main branch have been made
since the POC demo.  The total is the total number of commits for the entire
team since the POC demo.  The percentage is the percentage of the total commits
made by each team member.}

The number of commits in the main branch:
\begin{table}[H]
\centering
\begin{tabular}{lll}
\toprule
\textbf{Student} & \textbf{Commits} & \textbf{Percent}\\
\midrule
Total & 4 & 100\% \\
Anhao Jiao & 4 & 100\% \\
Kehao Huang & 4 & 100\% \\
Qianlin Chen & 4 & 100\% \\
Qi Shu & 4 & 100\% \\
Xunzhou Ye & 4 & 100\% \\
\bottomrule
\end{tabular}
\end{table}

The number of commits in the main branch of our project is not a accurate indicator of the team member contributions. As we squash all the commits in pull requrests.

The number of commits in unmerged branches:
\begin{table}[H]
    \centering
    \begin{tabular}{lll}
    \toprule
    \textbf{Student} & \textbf{Commits} & \textbf{Percent}\\
    \midrule
    Total & 19 & 100\% \\
    Anhao Jiao & 2 & 10.5\% \\
    Kehao Huang & 2 & 10.5\% \\
    Qianlin Chen & 3 & 15.8\% \\
    Qi Shu & 16 & 84.2\% \\
    Xunzhou Ye & 4 & 21.0\% \\
    \bottomrule
    \end{tabular}
    \end{table}

Works in our project is distributed as we discussed. Certain parts of the project may require more work than others, 
Members who work on those parts will have more commits. As well as research contributions can not be reflected by the number of commits.

\section{Issue Tracker}

\wss{For each team member how many issues have they authored (including open and
closed issues) and how many have they been assigned (only counting closed
issues).}

\begin{table}[H]
\centering
\begin{tabular}{lll}
\toprule
\textbf{Student} & \textbf{Authored (O+C)} & \textbf{Assigned (C only)}\\
\midrule
Name 1 & Num & Num \\
Name 2 & Num & Num \\
Name 3 & Num & Num \\
Name 4 & Num & Num \\
Name 5 & Num & Num \\
\bottomrule
\end{tabular}
\end{table}

\wss{If needed, an explanation for the counts can be provided here.}

\section{CICD}

\wss{Say how CICD is used in your project}

\end{document}
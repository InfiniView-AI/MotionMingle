\documentclass{article}

\usepackage{tabularx}
\usepackage{booktabs}

\title{Reflection Report on \progname}

\author{\authname}

\date{}

\input{../Comments}
%% Common Parts

\newcommand{\progname}{SFWRENG 4G06 Capstone Design Project} % PUT YOUR PROGRAM NAME HERE
\newcommand{\projname}{MotionMingle} % project name
\newcommand{\authname}{Team \#18, InfiniView-AI
\\ Anhao Jiao
\\ Kehao Huang
\\ Qianlin Chen
\\ Shu Qi
\\ Xunzhou Ye
} % AUTHOR NAMES

\usepackage{hyperref}
    \hypersetup{colorlinks=true, linkcolor=blue, citecolor=blue, filecolor=blue,
                urlcolor=blue, unicode=false}
    \urlstyle{same}


\begin{document}

\maketitle

\plt{Reflection is an important component of getting the full benefits from a
learning experience.  Besides the intrinsic benefits of reflection, this
document will be used to help the TAs grade how well your team responded to
feedback.  In addition, several CEAB (Canadian Engineering Accreditation Board)
Learning Outcomes (LOs) will be assessed based on your reflections.}

\section{Changes in Response to Feedback}

\plt{Summarize the changes made over the course of the project in response to
feedback from TAs, the instructor, teammates, other teams, the project
supervisor (if present), and from user testers.}

\plt{For those teams with an external supervisor, please highlight how the feedback 
from the supervisor shaped your project.  In particular, you should highlight the 
supervisor's response to your Rev 0 demonstration to them.}

\subsection{SRS}
\subsubsection{Changes in Response to TA Feedback}
\begin{enumerate}
    \item TA review: Python, JavaScript, etc are not really imposed technical choices.\newline
            Changes: Eliminate the mandated technical choices, as our project does not require the specified technology.
    \item TA review:  Unclear what unit you're using for your upload and download speed.\newline
            Changes: Include the speed unit in the Symbolic Constants section. For example: USER\_SATISFACTION\_PERCENTAGE = 
\end{enumerate}
\subsubsection{Changes in Response to Peer Review}

\subsection{Hazard Analysis}
\subsubsection{Changes in Response to TA Feedback}
\begin{enumerate}
        \item Move the definitions section to the beginning.\\
                Changes: Move the Glossary to the beginning of the document.
        \item The introduction section is unclear and confusing to read.\\
                Changes: Revise the introduction.
        \item Some things like STUN and TURN are never defined.\\
                Changes: Add undefined terms to the Glossary.
        \item There are existing "magic" numbers instead of constants.\\
                Changes:Add undefined numbers to the Symbolic Constants table
        \item Some assumptions are too big and ambiguous.\\
                Changes: Revised all the assumptions and implemented a series of measurements to mitigate them.
        \item PR9 is not clearly defined.\\
                Changes: Revised PR9 and adjusted the rationale behind this requirement.
\end{enumerate}
\subsubsection{Changes in Response to Peer Review}
\begin{enumerate}
        \item Add STUN/TURN server to glossary and change the glossary location to the beginning.\\
                Changes: Reorganized the document from TA's feedback
        \item Ambiguous effects of failure for ML annotation pipeline.\\
                Changes: Clearly stated effects of failure for both failure mode: Inaccurate Annotation Produced and Latency in
                Annotation.
        \item Ambiguity of Critical Assumption.\\
                Chnages: Issues was addressed from TA's feedback.        
\end{enumerate}

\subsection{Design and Design Documentation}

\subsubsection{Changes in Response to TA Feedback}

\begin{enumerate}
        \item Be specific about dates in the timeline.\\
        Changes: Updated the project timeline with specific dates for 
                all milestones and deliverables, ensuring clear expectations for project progression.
        \item AC1 \& 2 are very confusing.\\
        Changes: Clarified Acceptance Criteria 1 and 2 by redefining the terms 
                and objectives, ensuring they are succinct and unambiguous.
        \item UC10: laws can change, how will you be able to adapt to conform to 
        new/updated laws?\\
        Changes: Incorporated a regulatory review process into the project plan to ensure compliance with current and future laws.
        \item UC11: you should be writing these as if they're for a "real" product 
        and not just a Capstone project. The client is definitely not an unanticipated change.\\
        Changes: Revise to ensure the product remains viable and adaptable for broader 
        deployment beyond the scope of the initial capstone project.
        \item UC13: this is more of a "meta" level idea about the project, rather 
        than about the project (product) itself - isn't really necessary to include here.\\
        Changes: Removed this unlikely change.
\end{enumerate}    

\subsubsection{Changes in Response to Peer Review}
\begin{enumerate}
        \item Module Guide Section 3.1 missing Links to relevent documents\\
        Changes: Updated Section 3.1 of the Module Guide to include hyperlinks to all pertinent documents, providing easy access and navigation for readers.
        \item MIS lack of Exceptions.\\
        Changes: Enhanced the Module Interface Specification (MIS) by defining exception handling procedures and including them in the corresponding sections, ensuring robust system behavior under error conditions.
\end{enumerate}    

\subsection{VnV Plan and Report}
\subsubsection{VnVplan: Changes in Response to TA Feedback}
\begin{enumerate}
        \item Some minor grammatical errors. \\
        Changes: Conducted a comprehensive proofreading of the VnV Plan and corrected all grammatical errors to enhance the document's clarity and professionalism.
        \item Write the document much more professionally, should never explicitly 
        reference anything about the team being the "weak link".\\
        Changes: Revised the language throughout the document to maintain a professional tone, removing any self-referential comments about the team's shortcomings.
        \item Some things in the plan are not explained.\\
        Changes: Expanded the explanations for all elements within the plan, ensuring that the document is self-explanatory and comprehensive.
        \item Make better use of your symbolic constants section.\\
        Changes: Refactored the symbolic constants section to improve readability and accessibility, and integrated these constants throughout the document for consistent reference.
        \item Be more clear about your 1 to 5 scale in the testing plan.\\
        Changes: Defined the 1 to 5 scale in the testing plan with explicit criteria for each level, ensuring a clear understanding of the test result expectations.
\end{enumerate}   
\subsubsection{VnVplan: Changes in Response to Peer Review}
\begin{enumerate}
        \item Relevant documents should include design documents (even if they 
        haven't been written yet), such as the Module Guide and Module Interface 
        Specification.\\
        Changes: Listed all anticipated design documents in the VnV Plan with provisional titles and descriptions, outlining their purpose and interrelations.
        \item Specify what kinds of testing strategies your teammates will employ 
        for specific modules of the project.\\
        Changes: Detailed individual testing strategies tailored for each module, assigned to specific team members, with clear objectives and methodologies.
        \item NFR-T1 specifies functional interaction to test but LF-1 is purely 
        non functional and does not specify any functionality. Consider adding 
        in another functional requirement to specify this functionality or alter 
        the NFR-T1 appropriately.\\
        Changes: Modified NFR-T1 to delineate the boundary between functional and non-functional requirements clearly
        \item NFR-T18 fit criteria not specific\\
        Changes: Elaborated on the fit criteria for NFR-T18, providing measurable and observable conditions for successful implementation.
\end{enumerate} 
\subsubsection{VnV Report: Changes in Response to TA Feedback}

\subsubsection{VnV Report: Changes in Response to Peer Review}


\section{Design Iteration (LO11)}

\plt{Explain how you arrived at your final design and implementation.  How did
the design evolve from the first version to the final version?} 

\section{Design Decisions (LO12)}

\plt{Reflect and justify your design decisions.  How did limitations,
 assumptions, and constraints influence your decisions?}

\section{Economic Considerations (LO23)}

\plt{Is there a market for your product? What would be involved in marketing your 
product? What is your estimate of the cost to produce a version that you could 
sell?  What would you charge for your product?  How many units would you have to 
sell to make money? If your product isn't something that would be sold, like an 
open source project, how would you go about attracting users?  How many potential 
users currently exist?}

\section{Reflection on Project Management (LO24)}

\plt{This question focuses on processes and tools used for project management.}

\subsection{How Does Your Project Management Compare to Your Development Plan}

\plt{Did you follow your Development plan, with respect to the team meeting plan, 
team communication plan, team member roles and workflow plan.  Did you use the 
technology you planned on using?}

\subsection{What Went Well?}

\plt{What went well for your project management in terms of processes and 
technology?}

\subsection{What Went Wrong?}

\plt{What went wrong in terms of processes and technology?}

\subsection{What Would you Do Differently Next Time?}

\plt{What will you do differently for your next project?}

\end{document}
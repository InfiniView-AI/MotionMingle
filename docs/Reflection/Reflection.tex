\documentclass{article}

\usepackage{tabularx}
\usepackage{booktabs}

\title{Reflection Report on \progname}

\author{\authname}

\date{}

%% Comments

\usepackage{color}

\newif\ifcomments\commentstrue %displays comments
%\newif\ifcomments\commentsfalse %so that comments do not display

\ifcomments
\newcommand{\authornote}[3]{\textcolor{#1}{[#3 ---#2]}}
\newcommand{\todo}[1]{\textcolor{red}{[TODO: #1]}}
\else
\newcommand{\authornote}[3]{}
\newcommand{\todo}[1]{}
\fi

\newcommand{\wss}[1]{\authornote{blue}{SS}{#1}} 
\newcommand{\plt}[1]{\authornote{magenta}{TPLT}{#1}} %For explanation of the template
\newcommand{\an}[1]{\authornote{cyan}{Author}{#1}}

%% Common Parts

\newcommand{\progname}{SFWRENG 4G06 Capstone Design Project} % PUT YOUR PROGRAM NAME HERE
\newcommand{\projname}{MotionMingle} % project name
\newcommand{\authname}{Team \#18, InfiniView-AI
\\ Anhao Jiao
\\ Kehao Huang
\\ Qianlin Chen
\\ Shu Qi
\\ Xunzhou Ye
} % AUTHOR NAMES

\usepackage{hyperref}
    \hypersetup{colorlinks=true, linkcolor=blue, citecolor=blue, filecolor=blue,
                urlcolor=blue, unicode=false}
    \urlstyle{same}


\begin{document}

\maketitle

\plt{Reflection is an important component of getting the full benefits from a
learning experience.  Besides the intrinsic benefits of reflection, this
document will be used to help the TAs grade how well your team responded to
feedback.  In addition, several CEAB (Canadian Engineering Accreditation Board)
Learning Outcomes (LOs) will be assessed based on your reflections.}

\section{Changes in Response to Feedback}

\plt{Summarize the changes made over the course of the project in response to
feedback from TAs, the instructor, teammates, other teams, the project
supervisor (if present), and from user testers.}

\plt{For those teams with an external supervisor, please highlight how the feedback 
from the supervisor shaped your project.  In particular, you should highlight the 
supervisor's response to your Rev 0 demonstration to them.}

\subsection{SRS}
\subsubsection{Changes in Response to TA Feedback}
\begin{enumerate}
    \item TA review: Python, JavaScript, etc are not really imposed technical choices.\newline
            Changes: Eliminate the mandated technical choices, as our project does not require the specified technology.
    \item TA review:  Unclear what unit you're using for your upload and download speed.\newline
            Changes: Include the speed unit in the Symbolic Constants section. For example: USER\_SATISFACTION\_PERCENTAGE = 
\end{enumerate}
\subsubsection{Changes in Response to Peer Review}

\subsection{Hazard Analysis}
\subsubsection{Changes in Response to TA Feedback}
\begin{enumerate}
        \item Move the definitions section to the beginning.\\
                Changes: Move the Glossary to the beginning of the document.
        \item The introduction section is unclear and confusing to read.\\
                Changes: Revise the introduction.
        \item Some things like STUN and TURN are never defined.\\
                Changes: Add undefined terms to the Glossary.
        \item There are existing "magic" numbers instead of constants.\\
                Changes:Add undefined numbers to the Symbolic Constants table
        \item Some assumptions are too big and ambiguous.\\
                Changes: Revised all the assumptions and implemented a series of measurements to mitigate them.
        \item PR9 is not clearly defined.\\
                Changes: Revised PR9 and adjusted the rationale behind this requirement.
\end{enumerate}
\subsubsection{Changes in Response to Peer Review}
\begin{enumerate}
        \item Add STUN/TURN server to glossary and change the glossary location to the beginning.\\
                Changes: Reorganized the document from TA's feedback
        \item Ambiguous effects of failure for ML annotation pipeline.\\
                Changes: Clearly stated effects of failure for both failure mode: Inaccurate Annotation Produced and Latency in
                Annotation.
        \item Ambiguity of Critical Assumption.\\
                Chnages: Issues was addressed from TA's feedback.        
\end{enumerate}
\subsection{Design and Design Documentation}
\subsubsection{Changes in Response to TA Feedback}

\subsubsection{Changes in Response to Peer Review}

\subsection{VnV Plan and Report}
\subsubsection{Changes in Response to TA Feedback}

\subsubsection{Changes in Response to Peer Review}

\section{Design Iteration (LO11)}

\plt{Explain how you arrived at your final design and implementation.  How did
the design evolve from the first version to the final version?} 

\section{Design Decisions (LO12)}

\plt{Reflect and justify your design decisions.  How did limitations,
 assumptions, and constraints influence your decisions?}

\section{Economic Considerations (LO23)}

\plt{Is there a market for your product? What would be involved in marketing your 
product? What is your estimate of the cost to produce a version that you could 
sell?  What would you charge for your product?  How many units would you have to 
sell to make money? If your product isn't something that would be sold, like an 
open source project, how would you go about attracting users?  How many potential 
users currently exist?}

\section{Reflection on Project Management (LO24)}

\plt{This question focuses on processes and tools used for project management.}

\subsection{How Does Your Project Management Compare to Your Development Plan}

\plt{Did you follow your Development plan, with respect to the team meeting plan, 
team communication plan, team member roles and workflow plan.  Did you use the 
technology you planned on using?}

\subsection{What Went Well?}

\plt{What went well for your project management in terms of processes and 
technology?}

\subsection{What Went Wrong?}

\plt{What went wrong in terms of processes and technology?}

\subsection{What Would you Do Differently Next Time?}

\plt{What will you do differently for your next project?}

\end{document}
\documentclass{article}

\usepackage{tabularx}
\usepackage{booktabs}

\title{Reflection Report on \progname}

\author{\authname}

\date{}

%% Comments

\usepackage{color}

\newif\ifcomments\commentstrue %displays comments
%\newif\ifcomments\commentsfalse %so that comments do not display

\ifcomments
\newcommand{\authornote}[3]{\textcolor{#1}{[#3 ---#2]}}
\newcommand{\todo}[1]{\textcolor{red}{[TODO: #1]}}
\else
\newcommand{\authornote}[3]{}
\newcommand{\todo}[1]{}
\fi

\newcommand{\wss}[1]{\authornote{blue}{SS}{#1}} 
\newcommand{\plt}[1]{\authornote{magenta}{TPLT}{#1}} %For explanation of the template
\newcommand{\an}[1]{\authornote{cyan}{Author}{#1}}

%% Common Parts

\newcommand{\progname}{SFWRENG 4G06 Capstone Design Project} % PUT YOUR PROGRAM NAME HERE
\newcommand{\projname}{MotionMingle} % project name
\newcommand{\authname}{Team \#18, InfiniView-AI
\\ Anhao Jiao
\\ Kehao Huang
\\ Qianlin Chen
\\ Shu Qi
\\ Xunzhou Ye
} % AUTHOR NAMES

\usepackage{hyperref}
    \hypersetup{colorlinks=true, linkcolor=blue, citecolor=blue, filecolor=blue,
                urlcolor=blue, unicode=false}
    \urlstyle{same}


\begin{document}

\maketitle

\plt{Reflection is an important component of getting the full benefits from a
learning experience.  Besides the intrinsic benefits of reflection, this
document will be used to help the TAs grade how well your team responded to
feedback.  In addition, several CEAB (Canadian Engineering Accreditation Board)
Learning Outcomes (LOs) will be assessed based on your reflections.}

\section{Changes in Response to Feedback}

\plt{Summarize the changes made over the course of the project in response to
feedback from TAs, the instructor, teammates, other teams, the project
supervisor (if present), and from user testers.}

\plt{For those teams with an external supervisor, please highlight how the feedback 
from the supervisor shaped your project.  In particular, you should highlight the 
supervisor's response to your Rev 0 demonstration to them.}

\subsection{SRS}
\subsubsection{Changes in Response to TA Feedback}
\begin{enumerate}
    \item Feedbacks and related changes on Document Content:
    \begin{itemize}
        \item Feedback: The choice of programming languages like Python and JavaScript is not inherently a technical imposition. \\
                Change: We changed the Imposed Technical Choice to N/A.
        \item Feedback: Assuming that "Understanding of the Annotation" without proper user testing is considered a weak foundation for the product's design. \\
                Change: The document content has been revised to include user testing for validating the effectiveness of annotations and to incorporate continuous user feedback for improving clarity and comprehensiveness.
        \item Feedback: The cognitive abilities of the user base may vary, and this needs to be taken into account when designing the product. \\
                Change: The section on Cognitive Ability has been updated to the interface will be designed to be intuitive across a spectrum of cognitive abilities and will be refined through usability testing.
        \item Feedback: Many laptops, including MacBooks, only have 720p cameras, so expecting users to have 1080p cameras might be unreasonable. \\
                Change: The Supported Video Resolutions section has been modified to clarify that the system supports video resolutions from 480p as the minimum to 1080p as the maximum, 
                addressing a broader range of webcam capabilities for user devices.
        \end{itemize}
    \item Feedbacks and related changes on Document Organization:
    \begin{itemize}
        \item Feedback: Clarify the units used for upload and download speeds, indicating whether they are in megabits or megabytes and defining these symbolic constants.\\
                Change: We add the unit for upload and download speeds and define symbolic constants with their respective values at the start of the document.
        \item Feedback: Include definitions and explanations of acronyms like STUN, TURN, SFU, etc., not just what they stand for.\\
                Change: We added the explanations of acronyms in the Glossary section.
    \end{itemize}
    \item Feedbacks and related changes on Formatting and Style:
    \begin{itemize}
        \item Feedback: Missing hyperlinks in some cases.\\
                Change: We added the missing hyperlinks(e.g. traceability matrix).
    \end{itemize}
    \item Feedbacks and related changes on Notation and Conventions:
    \begin{itemize}
        \item Feedback: Define everything at the start - and for acronyms define them, don't just give the expanded meaning.\\
                Change: Defined all the constant at the beginning under Glossary and Symbolic Constants sections.
    \end{itemize}
    \item Feedbacks and related changes on What not How (Abstract):
    \begin{itemize}
        \item Feedback: Some things like WebRTC don't need to be mentioned at this point. Any realtime protocol could theoretically be used.\\
                Change: We changed WebRTC to Real-Time video streaming or Real-Time Communoication.
    \end{itemize}
    \item Feedbacks and related changes on Complete, Correct and Unambiguous:
    \begin{itemize}
        \item Feedback: Some functional requirements feel a little "big", and some NFRs in particular are ambiguous.\\
                Change: The document will be revised to remove ambiguities and ensure that requirements are specific, measurable, and testable.
                (e.g. FR6 has been updated to specify the types of annotations that users can configure, providing clear options and addressing the previous feedback for more detailed information.)
    \end{itemize}
    \item Feedbacks and related changes on Verifiable Requirements:
    \begin{itemize}
        \item Feedback: Fit criteria are sometimes ambiguous.\\
                Change: Fit criteria ambiguity has been addressed by adding specific range(e.g. USER SATISFACTION PERCENTAGE for UH1).
    \end{itemize}
    \item Feedbacks and related changes on Focus on Users (Part of Enterprise Design Thinking):
    \begin{itemize}
        \item Feedback: Not completely clear who the users are.\\
                Change: We defined the term elderly people by specifing their aged 65 years and older. 
    \end{itemize}
    \item Feedbacks and related changes on Phase In Plan:
    \begin{itemize}
        \item Feedback: Does not include timeline for when things of different priorities will be phased in.\\
                Change: We included the outlining a clear priority and schedule for project milestones.
    \end{itemize}
    \item Feedbacks and related changes on Basis for Design:
    \begin{itemize}
        \item Feedback: Haven't given any indication as to the types of annotations you will provide.\\
                Change: We included the specific annotations used in our project.
    \end{itemize}
    \item Feedbacks and related changes on LO\_Reflect:
    \begin{itemize}
        \item Feedback: Reflection Appendix is missing.\\
                Change: We added the missing Relection at the end of document.
    \end{itemize}
    \item Feedbacks and related changes on LO\_SpecMath:
    \begin{itemize}
        \item Feedback: The state diagram feels a bit small; does not capture the system very fully.\\
                Change: We broadered the diagram, and include more states in the diagram.
    \end{itemize}
\end{enumerate}
\subsubsection{Changes in Response to Peer Review}
\begin{itemize}
        \item Feedback: Need a stronger fit criterias for FR and NFR.\\
                Change: Enhanced fit criterias by adding some specific range and definitions for the ambiguous FR and NFR 
                (UH1, UH2, OE2, OE3, MS1, HS1, HS2, SR1, and SR2).
        \item Feedback: Missing FR for Audio Capability.\\
                Change: Added the FR for Audio Capability.
\end{itemize}
\subsubsection{Changes in Response to Supervisor Feedback}
\begin{itemize}
        \item Feedback: Should include the new annotation, Center of Mass Visulization.\\
                Change: We changed the Clock Annotation to Center of Mass Visualization and defined it in Glossary.
\end{itemize}
\subsection{Hazard Analysis}
\subsubsection{Changes in Response to TA Feedback}
\begin{enumerate}
        \item Move the definitions section to the beginning.\\
                Changes: Move the Glossary to the beginning of the document.
        \item The introduction section is unclear and confusing to read.\\
                Changes: Revise the introduction.
        \item Some things like STUN and TURN are never defined.\\
                Changes: Add undefined terms to the Glossary.
        \item There are existing "magic" numbers instead of constants.\\
                Changes:Add undefined numbers to the Symbolic Constants table
        \item Some assumptions are too big and ambiguous.\\
                Changes: Revised all the assumptions and implemented a series of measurements to mitigate them.
        \item PR9 is not clearly defined.\\
                Changes: Revised PR9 and adjusted the rationale behind this requirement.
\end{enumerate}
\subsubsection{Changes in Response to Peer Review}
\begin{enumerate}
        \item Add STUN/TURN server to glossary and change the glossary location to the beginning.\\
                Changes: Reorganized the document from TA's feedback
        \item Ambiguous effects of failure for ML annotation pipeline.\\
                Changes: Clearly stated effects of failure for both failure mode: Inaccurate Annotation Produced and Latency in
                Annotation.
        \item Ambiguity of Critical Assumption.\\
                Chnages: Issues was addressed from TA's feedback.        
\end{enumerate}

\subsection{Design and Design Documentation}

\subsubsection{Changes in Response to TA Feedback}

\begin{enumerate}
        \item Be specific about dates in the timeline.\\
        Changes: Updated the project timeline with specific dates for 
                all milestones and deliverables, ensuring clear expectations for project progression.
        \item AC1 \& 2 are very confusing.\\
        Changes: Clarified Acceptance Criteria 1 and 2 by redefining the terms 
                and objectives, ensuring they are succinct and unambiguous.
        \item UC10: laws can change, how will you be able to adapt to conform to 
        new/updated laws?\\
        Changes: Incorporated a regulatory review process into the project plan to ensure compliance with current and future laws.
        \item UC11: you should be writing these as if they're for a "real" product 
        and not just a Capstone project. The client is definitely not an unanticipated change.\\
        Changes: Revise to ensure the product remains viable and adaptable for broader 
        deployment beyond the scope of the initial capstone project.
        \item UC13: this is more of a "meta" level idea about the project, rather 
        than about the project (product) itself - isn't really necessary to include here.\\
        Changes: Removed this unlikely change.
\end{enumerate}    

\subsubsection{Changes in Response to Peer Review}
\begin{enumerate}
        \item Module Guide Section 3.1 missing Links to relevent documents\\
        Changes: Updated Section 3.1 of the Module Guide to include hyperlinks to all pertinent documents, providing easy access and navigation for readers.
        \item MIS lack of Exceptions.\\
        Changes: Enhanced the Module Interface Specification (MIS) by defining exception handling procedures and including them in the corresponding sections, ensuring robust system behavior under error conditions.
\end{enumerate}    

\subsection{VnV Plan and Report}
\subsubsection{VnVplan: Changes in Response to TA Feedback}
\begin{enumerate}
        \item Some minor grammatical errors. \\
        Changes: Conducted a comprehensive proofreading of the VnV Plan and corrected all grammatical errors to enhance the document's clarity and professionalism.
        \item Write the document much more professionally, should never explicitly 
        reference anything about the team being the "weak link".\\
        Changes: Revised the language throughout the document to maintain a professional tone, removing any self-referential comments about the team's shortcomings.
        \item Some things in the plan are not explained.\\
        Changes: Expanded the explanations for all elements within the plan, ensuring that the document is self-explanatory and comprehensive.
        \item Make better use of your symbolic constants section.\\
        Changes: Refactored the symbolic constants section to improve readability and accessibility, and integrated these constants throughout the document for consistent reference.
        \item Be more clear about your 1 to 5 scale in the testing plan.\\
        Changes: Defined the 1 to 5 scale in the testing plan with explicit criteria for each level, ensuring a clear understanding of the test result expectations.
\end{enumerate}   
\subsubsection{VnVplan: Changes in Response to Peer Review}
\begin{enumerate}
        \item Relevant documents should include design documents (even if they 
        haven't been written yet), such as the Module Guide and Module Interface 
        Specification.\\
        Changes: Listed all anticipated design documents in the VnV Plan with provisional titles and descriptions, outlining their purpose and interrelations.
        \item Specify what kinds of testing strategies your teammates will employ 
        for specific modules of the project.\\
        Changes: Detailed individual testing strategies tailored for each module, assigned to specific team members, with clear objectives and methodologies.
        \item NFR-T1 specifies functional interaction to test but LF-1 is purely 
        non functional and does not specify any functionality. Consider adding 
        in another functional requirement to specify this functionality or alter 
        the NFR-T1 appropriately.\\
        Changes: Modified NFR-T1 to delineate the boundary between functional and non-functional requirements clearly
        \item NFR-T18 fit criteria not specific\\
        Changes: Elaborated on the fit criteria for NFR-T18, providing measurable and observable conditions for successful implementation.
\end{enumerate} 
\subsubsection{VnV Report: Changes in Response to TA Feedback}

\subsubsection{VnV Report: Changes in Response to Peer Review}


\section{Design Iteration (LO11)}

\plt{Explain how you arrived at your final design and implementation.  How did
the design evolve from the first version to the final version?} 

\section{Design Decisions (LO12)}

\plt{Reflect and justify your design decisions.  How did limitations,
 assumptions, and constraints influence your decisions?}

\section{Economic Considerations (LO23)}

\plt{Is there a market for your product? What would be involved in marketing your 
product? What is your estimate of the cost to produce a version that you could 
sell?  What would you charge for your product?  How many units would you have to 
sell to make money? If your product isn't something that would be sold, like an 
open source project, how would you go about attracting users?  How many potential 
users currently exist?}

\section{Reflection on Project Management (LO24)}

\plt{This question focuses on processes and tools used for project management.}

\subsection{How Does Your Project Management Compare to Your Development Plan}

\plt{Did you follow your Development plan, with respect to the team meeting plan, 
team communication plan, team member roles and workflow plan.  Did you use the 
technology you planned on using?}

\subsection{What Went Well?}

\plt{What went well for your project management in terms of processes and 
technology?}

\subsection{What Went Wrong?}

\plt{What went wrong in terms of processes and technology?}

\subsection{What Would you Do Differently Next Time?}

\plt{What will you do differently for your next project?}

\end{document}
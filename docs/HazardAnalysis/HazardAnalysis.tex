\documentclass{article}

\usepackage{booktabs}
\usepackage{tabularx}
% https://tex.stackexchange.com/questions/12703/how-to-create-fixed-width-table-columns-with-text-raggedright-centered-raggedlef
\newcolumntype{L}[1]{>{\raggedright\let\newline\\\arraybackslash\hspace{0pt}}p{#1}}


\usepackage{hyperref}
\hypersetup{
    colorlinks=true,       % false: boxed links; true: colored links
    linkcolor=red,          % color of internal links (change box color with linkbordercolor)
    citecolor=green,        % color of links to bibliography
    filecolor=magenta,      % color of file links
    urlcolor=cyan           % color of external links
}

\usepackage[shortlabels]{enumitem}
\usepackage{pdflscape}
\usepackage{ltablex}
\usepackage{multirow}
\usepackage{makecell}
\usepackage{array}
\usepackage{geometry}
\usepackage{siunitx}
\usepackage{longtable}
\usepackage{ulem}
\usepackage[style=ieee]{biblatex}
\addbibresource{../../refs/References.bib}


\title{Hazard Analysis\\\progname}

\author{\authname}

\date{20 October 2023}

%% Comments

\usepackage{color}

\newif\ifcomments\commentstrue %displays comments
%\newif\ifcomments\commentsfalse %so that comments do not display

\ifcomments
\newcommand{\authornote}[3]{\textcolor{#1}{[#3 ---#2]}}
\newcommand{\todo}[1]{\textcolor{red}{[TODO: #1]}}
\else
\newcommand{\authornote}[3]{}
\newcommand{\todo}[1]{}
\fi

\newcommand{\wss}[1]{\authornote{blue}{SS}{#1}} 
\newcommand{\plt}[1]{\authornote{magenta}{TPLT}{#1}} %For explanation of the template
\newcommand{\an}[1]{\authornote{cyan}{Author}{#1}}

%% Common Parts

\newcommand{\progname}{SFWRENG 4G06 Capstone Design Project} % PUT YOUR PROGRAM NAME HERE
\newcommand{\projname}{MotionMingle} % project name
\newcommand{\authname}{Team \#18, InfiniView-AI
\\ Anhao Jiao
\\ Kehao Huang
\\ Qianlin Chen
\\ Shu Qi
\\ Xunzhou Ye
} % AUTHOR NAMES

\usepackage{hyperref}
    \hypersetup{colorlinks=true, linkcolor=blue, citecolor=blue, filecolor=blue,
                urlcolor=blue, unicode=false}
    \urlstyle{same}


\begin{document}

\maketitle
\thispagestyle{empty}

~\newpage

\pagenumbering{roman}

\begin{table}[hp]
  \caption{Revision History} \label{TblRevisionHistory}
  \begin{tabularx}{\textwidth}{llX}
    \toprule
    \textbf{Date} & \textbf{Developer(s)} & \textbf{Change}\\
    \midrule
    13 October 2023 & AJ, KH, QC, QS, XY & Initial draft \\
    \midrule
    15 October 2023 & AJ, KH, QS, XY & System Boundaries and Components,\\
    & & Critical Assumptions,\\
    & & Failure Mode and Effect Analysis,\\
    & & Safety and Security Requirements\\
    \midrule
    & QC & Introduction,\\
    & & Scope and Purpose of Hazard Analysis\\
    \midrule
    19 October 2023 & AJ, KH, QS, XY & Failure Mode and Effect Analysis,\\
    & & Safety and Security Requirements,\\
    & & Roadmap\\
    \midrule
    & QC & Introduction,\\
    & & Scope and Purpose of Hazard Analysis\\
    \midrule\\
    25 March 2024 & AJ, KH, QC, QS, XY & Rev1 \\
    \bottomrule
  \end{tabularx}
\end{table}

\newpage{}

\tableofcontents{}
\listoftables{}

\newpage{}

\pagenumbering{arabic}

\section{Introduction}

In alignment with the foundational principles laid out by
\textcite{leveson_engineering_2011}, \sout{a ``hazard'' within the context of our Tai
Chi instruction platform can be identified as any inherent property or external
condition. It could potentially cause the system to deviate from its intended
functionality, particularly when interacting with various environmental factors.}
\textcolor{red}{a 'hazard' within the context of our Tai Chi instruction platform 
refers to any condition, event, or circumstance that could adversely affect the 
system’s operation, integrity, or safety. This includes software malfunctions, 
hardware failures, user errors, or external disruptions. A hazard becomes a concern 
when it has the potential to interfere with the platform’s ability to deliver 
instructional content effectively or to maintain a safe, secure, and engaging 
learning environment for the users}

This document is dedicated to a comprehensive hazard analysis of our innovative
Tai Chi video conferencing application, identifying the hazards and emphasizing
the appropriate actions for the hazards.

\subsection{\textcolor{red}{Glossary}}

\begin{description}
\item[Tai Chi] A classical Chinese martial art system practiced for health
  promotion and rehabilitation.
\item[Instructor] A person who teaches a Tai Chi class through an online conference
  system.
\item[Practitioner] A person who learns Tai Chi through an online conference
  system.
\item[Machine Learning Model] A mathematical model designed to find patterns and make predictions or decisions based on data
\item[Annotation Pipeline] A sequence of processing elements connected in series, which are responsible for generating annotations
\item[SFU] Selective Forwarding Unit, a component in real-time communication systems like WebRTC that routes and selectively forwards audio and video streams from one participant to others
\item[\textcolor{red}{STUN/TURN servers}] \textcolor{red}{A component in real-time communication systems that is responsible for establishing and maintaining connections.}
\end{description}

\subsection{\textcolor{red}{Symbolic Constants}}

\begin{table}[h]
  \caption{Symbolic constants in Hazard Analysis}
  \begin{tabularx}{1.0\linewidth}[h]{ll} \toprule
    \textbf{Symbol} & \textbf{Value} \\ \midrule
    MAX\_DELAY \label{const:delay} & \SI{500}{\milli\second} \\
    MIN\_RES \label{const:res} & 720p \\ \bottomrule
  \end{tabularx}
  \label{tab:syms}
\end{table}



\section{Scope and Purpose of Hazard Analysis}

This document describes the scope and purpose of hazard analysis for our
WebRTC-based Tai Chi instruction application, focusing on identifying potential
hazards within specific system boundaries and components, and prescribing
comprehensive mitigation strategies. While acknowledging that users' diverse
hardware configurations are beyond our control, the system is designed for broad
compatibility, assuming standard web browser functionality on the user's device.
Our analysis operates under the critical assumption that all application
functionalities, particularly those related to real-time instructional
mechanics, are performing as intended, thereby circumventing the need to predict
various user inputs. Emphasis is placed on fortifying key components---backend
server and UI---against potential failures. Through this analysis, we commit to
ensuring an uninterrupted, secure, and user-centric experience, essential for
the virtual dissemination and mastery of Tai Chi practices.


\section{System Boundaries and Components}

The system's boundaries are carefully defined to provide a clear understanding
of the components that interact with and are integral to our Tai Chi
video-conferencing application. These boundaries primarily encompass two
categories of components: System Components and Environment Components.

By outlining system boundaries and components, we aim to establish a framework
for hazard analysis that emphasizes the interplay between these key components
and the broader environmental factors. This holistic approach allows us to
identify and address potential hazards effectively while working to maintain the
application's integrity and user satisfaction.

\subsection{System Components}
\begin{itemize}
\item Client application
\item Signaling and media stream routing unit
\item Machine learning annotation pipeline
\end{itemize}

System Components comprise the essential elements that constitute our
application. These components are at the core of the system's functionality,
facilitating user interactions, data routing, and real-time machine
learning-based annotations.

\subsection{Environment Components}

\begin{itemize}
\item Personal computing devices
\item Media capturing device
\end{itemize}

Environment Components encompass the external factors that influence the
system's operation. These components are external to the system but play a
critical role in ensuring a seamless and productive user experience.

\section{Critical Assumptions}

To ensure that the hazard assessment and analysis process remains transparent,
accountable, and adaptable to changing circumstances, the following assumptions
are made:

\begin{enumerate}
\item \textcolor{red}{While the system is designed to mitigate and handle unintentional 
  user errors and common misuse scenarios, it may not be fully resilient against 
  sophisticated, intentionally malicious activities designed to exploit system 
  vulnerabilities or deceive advanced machine learning algorithms.}
\item \textcolor{red}{It is anticipated that users will share only legal content on 
the conferencing platform. Nevertheless, it is understood that there may be attempts 
to share abusive, criminal, or pornographic content, and measures should be considered 
to detect and prevent the dissemination of such materials.}
\item \textcolor{red}{It is assumed that users have the physical capability to interact 
with and operate the system. This does not preclude the exploration of accessibility 
options to ensure that the system is inclusive and accommodating of users with physical 
disabilities.}
\item \textcolor{red}{Hazard analysis for the media capturing device component is 
scoped to consider typical use cases, excluding extreme conditions outside the intended 
use of the platform.}
\item \sout{The user does not intentionally attempt to break the system, such as
providing deceptive inputs that aim to trick machine learning models
(adversarial attacks).}
\item \sout{Only legal content is shared on the conferencing platform. The user does not
  deliberately exploit the system to spread abusive, criminal, pornographic
  content.}
\item \sout{The user has no physical disability, meaning that users are presumed to have
  the physical capability to interact with and operate the system without
  encountering any limitations related to physical disabilities.}
\item \sout{Hazard analysis for the media capturing device component only applies to
  instructors.}
\end{enumerate}

\subsection{\textcolor{red}{Potential Mitigation Strategies}}
\textcolor{red}{The landscape of cybersecurity and system integrity presents a multitude of challenges 
that require diligent anticipation and proactive management. Recognizing that no system 
can be entirely immune to adversarial actions, the following potential mitigation strategies 
have been identified. These strategies are designed to enhance the resilience of the platform, 
reduce the likelihood of malicious exploitation, and ensure the integrity and continuity 
of the service provided. While some may extend beyond the current scope of the capstone 
project, they are integral to a comprehensive approach to system security and user safety. 
The subsequent list delineates methods and practices that could be implemented to safeguard 
the platform against a range of adversarial threats.}
\begin{enumerate}
\item \textcolor{red}{Implement Robust Input Validation: Use strong input validation 
checks to ensure that only expected types of data are processed by the system. This 
can help prevent a variety of injection attacks.}
\item \textcolor{red}{Employ Anomaly Detection: Utilize machine learning models for 
anomaly detection to identify unusual patterns that may indicate adversarial behavior.}
\item \textcolor{red}{Rate Limiting and CAPTCHA: Implement rate limiting to prevent 
automated attacks and CAPTCHA challenges to distinguish between human and automated access.}
\item \textcolor{red}{User Behavior Analytics: Analyze user behavior to identify 
potentially malicious actions and have measures in place for a quick response.}
\end{enumerate}
\section{Failure Mode and Effect Analysis}

In the Failure Mode and Effect Analysis (FMEA) section, we employ a structured
methodology to systematically identify potential failure modes, assess their
effects, and prioritize recommended actions to mitigate hazards and enhance the
safety and performance of our Tai Chi video conferencing application. Table
\ref{tab:fmea} summaries the FMEA for our project.

\setlength{\tabcolsep}{2pt}
\newgeometry{margin=2cm}
\begin{landscape}
  \begin{longtable}[h]{L{0.09\linewidth}L{0.1\linewidth}L{0.15\linewidth}L{0.13\linewidth}L{0.35\linewidth}L{0.05\linewidth}L{0.04\linewidth}}
    \caption{FMEA table} \label{tab:fmea} \\ \toprule
    \textbf{Component}
    & \textbf{Failure Mode}
    & \textbf{Effects of Failure}
    & \textbf{Causes of Failure}
    & \textbf{Recommended Action}
    & \textbf{SR}
    & \textbf{Ref} \\ \midrule
    \endfirsthead
    \textbf{Component}
    & \textbf{Failure Mode}
    & \textbf{Effects of Failure}
    & \textbf{Causes of Failure}
    & \textbf{Recommended Action}
    & \textbf{SR}
    & \textbf{Ref} \\ \midrule
    \endhead
    \multicolumn{7}{r@{}}{continue on next page \ldots} \\
    \endfoot
    \bottomrule
    \endlastfoot
    \multirow[c]{2}{1\linewidth}{Client Application}
    & Unauthorized access to media capturing devices
    & Invasion of user privacy
    & Lack of considerations for user privacy in the design process
    & \vspace{-1.1\topsep}
      \begin{itemize}[nosep,topsep=0pt,leftmargin=10pt]
      \item Ask for permission to access media capturing devices
      \item Have an indicator when a media capturing device is in use
      \item Revoke access as soon as the capturing device is no longer needed
      \end{itemize}
      \vspace{-1.1\topsep}
    & \ref{SR4}, \ref{SR5}
    & H1-1 \\
    & Unresponsive UI
    & The user interface is unresponsive to the user interaction
    & \vspace{-1.1\topsep}
      \begin{itemize}[nosep,topsep=0pt,leftmargin=10pt]
      \item Delayed response from the server
      \item Insufficient client-side resource
      \end{itemize}
      \vspace{-1.1\topsep}
    & \vspace{-1.1\topsep}
      \begin{itemize}[nosep,topsep=0pt,leftmargin=10pt]
      \item Design the system with redundant processing capacity in mind
      \item Test with workload larger than that in the expected usage scenario
      \end{itemize}
      \vspace{-1.1\topsep}
    & \ref{PR1}
    & H1-2 \\ \midrule
    \multirow[c]{2}{1\linewidth}{Signaling and Stream Routing Unit}
    & Signaling server down
    & New WebRTC connections cannot be established
    & Server hardware failures
    & Configure the system to automatically switch to a backup signaling
      server when the primary server experiences downtime.
    & \ref{PR9}
    & H2-1 \\
    & SFU overload
    & \vspace{-1.1\topsep}
      \begin{itemize}[nosep,topsep=0pt,leftmargin=10pt]
      \item Decreased video and audio quality
      \item Session crashes
      \end{itemize}
      \vspace{-1.1\topsep}
    & \vspace{-1.1\topsep}
      \begin{itemize}[nosep,topsep=0pt,leftmargin=10pt]
      \item Unexpected spikes in number of participants
      \item Insufficient resource
      \end{itemize}
      \vspace{-1.1\topsep}
    & \vspace{-1.1\topsep}
      \begin{itemize}[nosep,topsep=0pt,leftmargin=10pt]
      \item Conduct stress tests to determine the system’s maximum capacity.
      \item Once reaches the maximum capacity, the system will put new requests for
        creating or joining sessions on hold.
      \end{itemize}
      \vspace{-1.1\topsep}
    & \ref{PR2}, \ref{PR6}
    & H2-2 \\ \midrule
    \multirow[c]{2}{1\linewidth}{ML Annotation Pipeline}
    & Inaccurate annotation produced
    & \sout{Negatively impact learning outcomes.}\textcolor{red}{Inaccurate annotation 
    produced leads to misaligned or incorrect information being associated with the 
    learning content. This can cause confusion, misunderstandings, and the potential 
    propagation of misinformation among users, hindering their ability to accurately 
    learn and apply new knowledge.}
    & \vspace{-1.1\topsep}
      \begin{itemize}[nosep,topsep=0pt,leftmargin=10pt]
      \item Corrupted input data.
      \item Low-fidelity input data.
      \end{itemize}
      \vspace{-1.1\topsep}
    & \vspace{-1.1\topsep}
      \begin{itemize}[nosep,topsep=0pt,leftmargin=10pt]
      \item Set a confidence threshold. Refuse to process if below the threshold.
        Forward feedback to the front end.
      \item Increase allowance/tolerance of “bad” data, increase the robustness of
        the annotation pipeline.
      \end{itemize}
      \vspace{-1.1\topsep}
    & \ref{PR12}
    & H3-1 \\
    & Latency in Annotation
    & \sout{Negatively impact learning outcomes.}\textcolor{red}{Latency in annotation 
    may result in delayed synchronization between the educational content and its corresponding 
    annotations. This can disrupt the learning flow, reduce engagement, and prevent timely 
    comprehension of the material presented. Users may experience frustration and a lack of 
    continuity that can diminish the overall effectiveness of the learning session.}
    & \vspace{-1.1\topsep}
      \begin{itemize}[nosep,topsep=0pt,leftmargin=10pt]
      \item Heavy system load, inefficient machine learning model
      \item The high volume of render requests.
      \end{itemize}
      \vspace{-1.1\topsep}
    & \vspace{-1.1\topsep}
      \begin{itemize}[nosep,topsep=0pt,leftmargin=10pt]
      \item Improve the efficiency of the machine learning model.
      \item Allocate additional resources to accommodate higher loads.
      \end{itemize}
      \vspace{-1.1\topsep}
    & \ref{PR14}
    & H3-2 \\ \midrule
    \multirow[c]{1}{1\linewidth}{Personal Computing Device}
    & The application is not running correctly
    & App crashed
    & \vspace{-1.1\topsep}
      \begin{itemize}[nosep,topsep=0pt,leftmargin=10pt]
      \item Insufficient running memory.
      \item Outdated OS version.
      \end{itemize}
      \vspace{-1.1\topsep}
    & Automatically save conference metadata, try to reconnect after the application restart.
    & \ref{HS1}, \ref{PR7}
    & H4-1\\
    & Network Interruption
    & Client-Server connection lost
    & No internet connection on the user’s side.
    & Retry connection after a predetermined delay.
    & \ref{PR7}
    & H4-2\\
    & Network stability fluctuation
    & Inefficient bit rate and low-resolution
    & Unstable internet connection on the user’s side
    & \vspace{-1.1\topsep}
      \begin{itemize}[nosep,topsep=0pt,leftmargin=10pt]
      \item Monitor network quality in real-time. Warn users if instability is
        detected.
      \item Put user on hold if network problem persists.
      \end{itemize}
      \vspace{-1.1\topsep}
    & \ref{PR8}
    & H4-3\\ \midrule
    \multirow[c]{4}{1\linewidth}{Media Capturing Device}
    & Device lost connection
    & Loss of source video stream
    & A malfunctioning physical device or loose device connection
    & Send warnings to users when video or audio devices are disconnected
    & \ref{PR10}
    & H5-1\\
    & Vision obscured
    & Part of the instructor's body is invisible from the perspective of the video-capturing device
    & Misplaced capturing device; The user moves or rotates the video-capturing device by accident
    & Send warnings to the user when no full human body is detected in the capturing device
    & \ref{PR15}
    & H5-2\\
    & Insufficient resolution
    & Low-fidelity output data
    & Limited hardware capability
    & \vspace{-1.1\topsep}
      \begin{itemize}[nosep,topsep=0pt,leftmargin=10pt]
      \item Perform hardware capability examination, and warn the user if incapable
        hardware detected
      \item Specify and notify the user of the minimum system
        requirements/environment for running the application
      \end{itemize}
      \vspace{-1.1\topsep}
    & \ref{PR11}
    & H5-3\\
    & Multiple devices detected
    & The client application is unable to select the correct media-capturing device
    & Multiple media-capturing devices are connected to the machine running the application
    & Send warnings to the user when multiple media capturing devices are detected, and ask users to select the one they wants to use
    & \ref{PR13}
    & H5-4\\
  \end{longtable}
\end{landscape}
\restoregeometry

\section{Safety and Security Requirements}

New requirements identifiers are highlighted in bold.

\subsection{Performance Requirements}

\subsubsection{Speed Requirements}

\begin{enumerate}[PR1]
\item The system shall respond to user interactions (e.g. button clicks, menu
  selections) within 1 second. \label{PR1}
  \begin{description}
  \item[Rationale] To provide a responsive and smooth user experience.
  \item[Fit Criteria] User interactions result in near-instantaneous system
    responses under typical conditions.
  \item[Priority] Medium
  \end{description}
\end{enumerate}
\begin{enumerate}[label=\textbf{PR\arabic*}]
  \setcounter{enumi}{13}
\item The system shall provide annotations with minimal delay, ensuring real-time
  alignment with the instructor's live stream. \label{PR14}
  \begin{description}
  \item[Rationale] To ensure the highest quality of instructional annotations that
    effectively enhance the user experience and learning process. The latency of
    annotation should not be noticeable.
  \item[Fit Criteria] The generated annotation should have less than MAX\_DELAY
    latency between the stream image and annotation.
  \item[Priority] High
  \end{description}
\end{enumerate}

\subsubsection{Reliability and Availability Requirements}

\begin{enumerate}[PR1]
  \setcounter{enumi}{1}
\item The signaling server, SFU, and STUN/TURN servers shall operate with high
  reliability, minimizing service interruptions during live
  sessions. \label{PR2}
  \begin{description}
  \item[Rationale] To ensure a consistent and uninterrupted learning experience for users.
  \item[Fit Criteria] Real-time communication services are always available.
  \item[Priority] High
  \end{description}
\end{enumerate}
\begin{enumerate}[label=\textbf{PR\arabic*}]
  \setcounter{enumi}{6}
\item The system shall be able to resume the previous session when the session is
  accidentally terminated due to an application crash or internet interruption.
  \label{PR7}
	\begin{description}
  \item[Rationale] To enhance the overall user experience by minimizing disruptions
    caused by unforeseen events.
	\item[Fit Criteria] The system shall automatically save snapshots of conference
    metadata, try reconnecting after the application successfully restarts or
    internet access resumes.
	\item[Priority] Medium
  \end{description}
\item The system shall monitor the user's network quality while the user is using
  the application. \label{PR8}
	\begin{description}
  \item[Rationale] To ensure the conference quality of other users.
	\item[Fit Criteria] The system shall warn the user if network instability is
    detected, and put the user on hold if the issue persists.
	\item[Priority] Medium
  \end{description}
\item \sout{The system shall be running when the primary signaling server is
  down.}\textcolor{red}{The core functionalities of the system(starting the session, 
  joining the session, getting annotated video) shall remain operational in the 
  event that the primary signaling server is down.} \label{PR9}
	\begin{description}
  \item[Rationale] \sout{To enhance system resilience and reliability and reduce system
    downtime.} \textcolor{red}{This requirement is to enhance system resilience 
    and reliability by ensuring that essential services can continue without 
    interruption, thereby reducing system downtime.}
	\item[Fit Criteria] A redundant signaling server shall be maintained alongside
    the primary signaling server, and shall be deployed when the primary
    signaling server is down.
	\item[Priority] Medium
  \end{description}
\item The system shall send warnings to users when video/audio capturing devices
  are disconnected. \label{PR10}
  \begin{description}
  \item[Rationale] The source video stream from the instructor is essential for a
    demonstrational conference session. If these devices become disconnected
    without warning, users may not be aware of the issue, leading to confusion
    and frustration.
	\item[Fit Criteria] The system should send clear and user-friendly warnings or
    notifications when it detects the disconnection of video or audio capturing
    devices.
	\item[Priority] Critical
  \end{description}
\item The system shall ensure that the quality of the video stream captured meets
  the minimum resolution requirement. \label{PR11}
	\begin{description}
  \item[Rationale] To ensure the quality of the input data to the ML pipeline.
	\item[Fit Criteria] The system shall perform hardware capability examination in
    any detected and authorized video capturing device, and notify users of the
    required resolution rate of MIN\_RES.
	\item[Priority] Critical
  \end{description}
\item The system shall generate accurate annotation on top of the user's live
  stream. \label{PR12}
	\begin{description}
  \item[Rationale] To ensure the highest quality of instructional annotations that
    effectively enhance the user experience and learning process. The generated
    annotation should be accurate enough.
	\item[Fit Criteria] The system should generate annotation that meets 4 out of 5
    team members' accuracy expectations. The accuracy expectations can be met by
    team members manually checking the annotation.
	\item[Priority] High
  \end{description}
\item The system shall use the media capturing devices the user specified when
  multiple types of capturing devices are detected. \label{PR13}
	\begin{description}
  \item[Rationale] To ensure the user experience, users should be able to specify
    the media capturing device they want to use.
	\item[Fit Criteria] When the system detects the presence of multiple media
    capturing devices of the same type (e.g. cameras, microphones), it shall
    display a notification to inform the user of this condition. The system will
    then prompt the user to select which media capturing device to utilize
    through a device selection form. This form shall allow the user to choose
    between the available devices of each type that were detected. Upon
    submission of the form, the chosen media capturing devices will be activated
    for use within the application.
	\item[Priority] Medium
  \end{description}
  \stepcounter{enumi}
\item The system shall make sure the view of the subject is within the field of
  view of the media capturing device. \label{PR15}
	\begin{description}
  \item[Rationale] Having a complete view of the subject's body ensures the quality
    of data feeding into the system for analyzing human body motions.
	\item[Fit Criteria] The system shall present detailed instructions for the user
    to properly set up the media capturing device, making sure the body of the
    subject is fully visible from the perspective of the camera.
  \item[Priority] Critical
  \end{description}
\end{enumerate}

\subsubsection{Scalability of Extensibility Requirements}

\begin{enumerate}[PR1]
  \setcounter{enumi}{5}
\item The Selective Forwarding Unit (SFU) shall be scalable to accommodate
  an increasing number of simultaneous video streams as the user base
  grows. \label{PR6}
  \begin{description}
  \item[Rationale] To support a growing user community without performance
    degradation.
  \item[Fit Criteria] The SFU can handle at least 10 simultaneous
    video streams during peak usage.
  \item[Priority] Medium
  \end{description}
\end{enumerate}

\subsubsection{Health and Safety Requirements}

\begin{enumerate}[HS1]
\item The system shall not cause the computers to overload. \label{HS1}
	\begin{description}
  \item[Rationale] The system should not overload the users' computers.
	\item[Fit Criteria] The hardware running the system is under normal temperature.
	\item[Priority] Medium
  \end{description}
\item The system shall not affect users' physical and mental health. \label{HS2}
  \begin{description}
  \item[Rationale] The system must not harm users' health and safety.
  \item[Fit Criteria] Users feel comfortable using the system in various
    situations.
  \item[Priority] Critical
  \end{description}
\end{enumerate}

\subsection{Security Requirements}

\begin{enumerate}[label=\textbf{SR\arabic*}]
  \setcounter{enumi}{3}
\item The system shall access media capturing devices only when user permission is
  granted. \label{SR4}
	\begin{description}
  \item[Rationale] To protect user privacy
	\item[Fit Criteria] A dialogue shall be displayed to ask for user permission to
    access media capturing devices.
	\item[Priority] Critical
  \end{description}
\item The system shall ensure the user is always aware of any active
  media capturing device. \label{SR5}
	\begin{description}
  \item[Rationale] To protect user privacy
	\item[Fit Criteria] An indicator shall be presented for each active media
    capturing device.
	\item[Priority] Critical
  \end{description}
\item The system shall not retain access to any media capturing device when they
  are not needed for video conferencing sessions. \label{SR6}
	\begin{description}
  \item[Rationale] To protect user privacy
	\item[Fit Criteria] The access to any media capturing device is terminated as
    soon as the session ends or the user exits the session.
	\item[Priority] Critical
  \end{description}
\end{enumerate}

\section{Roadmap}

In the hazard analysis documentation for our Tai Chi video conferencing
application, we have identified and prioritized a comprehensive set of safety
and security requirements together with other non-functional requirements
through the process of discovering potential hazards. These requirements are
essential for mitigating potential hazards, ensuring the reliability and
integrity of our system, and creating a secure and user-centric experience.

Requirements that address fundamental safety and security concerns, such as user
privacy, system reliability, and user well-being, are given critical priority.
These requirements include \ref{PR10}, \ref{PR11}, \ref{PR15}, \ref{HS2},
\ref{SR4}, \ref{SR5}, and \ref{SR6}. Requirements related to the seamless
operation of our system during both normal and unforeseen events, such as system
crashes or network interruptions, are considered high priority. These
requirements include \ref{PR14}, \ref{PR2}, and \ref{PR12}. Requirements that
focus on system performance, scalability, and the quality of user experience are
rated as medium priority. \ref{PR1}, \ref{PR6}, \ref{PR7}, \ref{PR8}, \ref{PR9},
\ref{PR13}, and \ref{HS1} fall under this category.

Given the limited time and human resource within the capstone project timeline,
all requirements of ``critical'' priority shall be implemented by the end of the
capstone project. Other requirements with lower priorities could be met by
future implementations. This roadmap will serve as a guide to ensure that our
Tai Chi video conferencing application evolves in a way that aligns with our
commitment to safety, security and user satisfaction.

\printbibliography[heading=bibnumbered]{}

\end{document}
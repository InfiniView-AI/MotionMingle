\documentclass[12pt, titlepage]{article}

\usepackage{amsmath, mathtools}

\usepackage[round]{natbib}
\usepackage{amsfonts}
\usepackage{amssymb}
\usepackage{graphicx}
\usepackage{colortbl}
\usepackage{xr}
\usepackage{hyperref}
\usepackage{longtable}
\usepackage{xfrac}
\usepackage{tabularx}
\usepackage{float}
\usepackage{siunitx}
\usepackage{booktabs}
\usepackage{multirow}
\usepackage[section]{placeins}
\usepackage{caption}
\usepackage{fullpage}
\usepackage{ulem}
\newcommand{\rt}[1]{\textcolor{red}{#1}}

\hypersetup{
bookmarks=true,     % show bookmarks bar?
colorlinks=true,       % false: boxed links; true: colored links
linkcolor=red,          % color of internal links (change box color with linkbordercolor)
citecolor=blue,      % color of links to bibliography
filecolor=magenta,  % color of file links
urlcolor=cyan          % color of external links
}

\usepackage{array}

\externaldocument[SRS-]{../../SRS/SRS}

\input{../../Comments}
%% Common Parts

\newcommand{\progname}{SFWRENG 4G06 Capstone Design Project} % PUT YOUR PROGRAM NAME HERE
\newcommand{\projname}{MotionMingle} % project name
\newcommand{\authname}{Team \#18, InfiniView-AI
\\ Anhao Jiao
\\ Kehao Huang
\\ Qianlin Chen
\\ Shu Qi
\\ Xunzhou Ye
} % AUTHOR NAMES

\usepackage{hyperref}
    \hypersetup{colorlinks=true, linkcolor=blue, citecolor=blue, filecolor=blue,
                urlcolor=blue, unicode=false}
    \urlstyle{same}


\begin{document}

\title{Module Interface Specification for \progname{}}

\author{\authname}

\date{\today}

\maketitle

\pagenumbering{roman}

\section{Revision History}

\begin{tabularx}{\textwidth}{p{3cm}X}
  \toprule
  {\bf Date} & {\bf Notes}                                                  \\
  \midrule
  Jan 12     & Add MIS for UI components                                    \\
  Jan 13     & Add MIS for Media Control components, RTC Control components \\
  Jan 14     & Add MIS for Backend components                               \\
  Jan 17     & Revise before submission                                     \\
  Mar 27     & Rev1                                                         \\
  \bottomrule
\end{tabularx}

~\newpage
\section{Symbols, Abbreviations and Acronyms}

\begin{tabularx}{\textwidth}{p{2cm}X}
  \toprule
  {\bf Symbol} & {\bf Description}                                                                                     \\
  \midrule
  MG           & Module Guide                                                                                          \\
  M            & Module                                                                                                \\
  MIS          & Module Interface Specification                                                                        \\
  HTTP         & Hypertext Transfer Protocol                                                                           \\
  OS           & Operating System                                                                                      \\
  STUN         & Session Traversal Utilities for NAT - a type of server needed for setting up peer-to-peer connections \\
  RTC          & Real-Time Communication                                                                               \\
  SFU          & Selective Forwarding Unit - A software unit that can selectively forward video streams                \\
  API          & Application Programming Interface                                                                     \\
  SDP          & Session Description Protocol                                                                          \\
  WebRTC       & Web Real-Time Communication                                                                           \\
  CM           & Center of Mass Annotation Module                                                                      \\
  HPE          & Human Pose Estimation Annotation Module                                                               \\
  \bottomrule
\end{tabularx}

\newpage

\tableofcontents

\newpage

\pagenumbering{arabic}

\section{Introduction}

The following document details the Module Interface Specifications for the
\projname application. Complementary documents include the Module Guide.

The full documentation and implementation can be found at \href{https://github.com/InfiniView-AI/MotionMingle}{MotionMingle.git}.


\section{Notation}

The following tables summarize the primitive data types, derived data types, and
other derived data types from aiortc, av, aiohttp, React and Web APIs libraries
that are used by \projname.

The structure of the MIS for modules comes from \citet{HoffmanAndStrooper1995},
with the addition that template modules have been adapted from
\cite{GhezziEtAl2003}.  The mathematical notation comes from Chapter 3 of
\citet{HoffmanAndStrooper1995}.  For instance, the symbol := is used for a
multiple assignment statement and conditional rules follow the form $(c_1
\Rightarrow r_1 | c_2 \Rightarrow r_2 | ... | c_n \Rightarrow r_n )$.

\subsection{Primitive Data Types}

The following table summarizes the primitive data types used by \projname.

\begin{center}
  \renewcommand{\arraystretch}{1.2}
  \noindent
  \begin{tabular}{l l p{7.5cm}}
    \toprule
    \textbf{Data Type} & \textbf{Notation} & \textbf{Description} \\
    \midrule
    character & char & a single symbol or digit \\
    integer & $\mathbb{Z}$ & a number without a fractional component in (-$\infty$, $\infty$) \\
    natural number & $\mathbb{N}$ & a number without a fractional component in [1, $\infty$) \\
    real & $\mathbb{R}$ & any number in (-$\infty$, $\infty$) \\
    boolean & $\mathbb{B}$ & a value of either True or False \\
    \bottomrule
  \end{tabular}
\end{center}

\noindent
The specification of \projname uses some derived data types: sequences, strings, and
tuples. Sequences are lists filled with elements of the same data type. Strings
are sequences of characters. Tuples contain a list of values, potentially of
different types. In addition, \projname uses functions, which
are defined by the data types of their inputs and outputs. Local functions are
described by giving their type signature followed by their specification.

\subsection{Data Types from Libraries}

The following table summarizes the data types provided by external libraries and
used by \projname.

\begin{table}[H]
  \centering
  \renewcommand{\arraystretch}{1.2}
  \noindent
  \begin{tabular}{l l p{7.5cm}}
    \toprule
    \textbf{Data Type}          & \textbf{Notation}           & \textbf{Description}                                                                                                                                                                                                                                            \\
    \midrule
    VideoStreamTrack      & VideoStreamTrack      & A dummy video track that reads green frames.                                                                                                                                                                                                              \\
    MediaRelay            & MediaRelay            & A media source that relays one or more tracks to multiple consumers.                                                                                                                                                                                      \\
    RTCPeerConnection     & RTCPeerConnection     & An interface represents a WebRTC connection between the local computer and a remote peer.                                                                                                                                                                 \\
    MediaStreamTrack      & MediaStreamTrack      & A single media track within a media stream.                                                                                                                                                                                                               \\
    RTCSessionDescription & RTCSessionDescription & An interface describes the potential connection and how it's configured. Each RTCSessionDescription consists of a description type indicating which part of the offer or answer negotiation process it describes and of the SDP descriptor of the session \\
    JSON                  & JSON                  & JavaScript Object Notation, it is a text-based open standard data interchange setup and only provides a data encoding specification.                                                                                                                      \\
    RTCTrackEvent         & RTCTrackEvent         & An event triggered by adding a MediaStreamTrack                                                                                                                                                                                                           \\
    MediaStream           & MediaStream           & A stream of data that usually carries media data                                                                                                                                                                                                          \\
    HTMLVideoElement      & HTMLVideoElement      & A react type representation of the video element in Hypertext Markup Language                                                                                                                                                                             \\
    React.component       & React.component       & A independent and reusable bits of react code that outputs HTML elements                                                                                                                                                                                  \\
    \bottomrule
  \end{tabular}
  \caption{Data types from libraries}
\end{table}

\section{Module Decomposition}

The following table is taken directly from the Module Guide document for this project.

\begin{table}[H]
  \centering
  \begin{tabular}{p{0.3\textwidth} p{0.6\textwidth}}
    \toprule
    \textbf{Level 1} & \textbf{Level 2} \\
    \midrule
    {Hardware-Hiding Module} & \\
    \midrule
    \multirow{7}{0.3\textwidth}{Behaviour-Hiding Module}
               & User Authentication Module \\
               & Instructor View Module \\
               & Practitioner View Module \\
               & Annotation Configuration Module \\
               & RTC Control Module \\
    \midrule
    \multirow{3}{0.3\textwidth}{Software Decision Module}
               & STUN Server Module \\
               & App Module \\
               & Video Transform Module \\
               & Human Pose Estimation Annotation Module \\
               & Center of Mass Annotation Module \\
               & SFU Server Module \\
    \bottomrule
  \end{tabular}
  \caption{Module Hierarchy}
  \label{TblMH}
\end{table}

\newpage

\section{MIS of RTC Control Module} \label{sec:rtcctrl}

\subsection{Module}

RTCControl

\subsection{Uses}

\noindent Web APIs

\noindent STUN Server Module

\subsection{Syntax}

\subsubsection{Exported Constants}

N/A

\subsubsection{Exported Access Programs}

\begin{center}
  \begin{tabular}{p{4cm} p{4cm} p{4cm} p{2cm}}
    \hline
    \textbf{Name}               & \textbf{In}             & \textbf{Out}            & \textbf{Exceptions} \\
    \hline
    createPeerConnection  & JSON              & RTCPeerConnection & -             \\
    closeRemoteConnection & RTCPeerConnection & -                 & -             \\
    negotiate             & RTCPeerConnection & -                 & -             \\
    \hline
  \end{tabular}
\end{center}

\subsection{Semantics}

\subsubsection{State Variables}

N/A

\subsubsection{Environment Variables}

\noindent STUN\_SERVER\_ADDRESS: string --- represents the address of the STUN server

\noindent SFU\_BROADCAST\_API: string --- represents the API endpoint for SFU broadcast API

\noindent SFU\_CONSUME\_API: string --- represents the API endpoint for SFU consume API

\subsubsection{Assumptions}

SFU server and STUN servers are running in normal conditions.

\subsubsection{Access Routine Semantics}

\noindent createPeerConnectionWith(config: JSON):
\begin{itemize}
\item transition: N/A
\item output: pc := RTCPeerConnection --- initializes a new RTCPeerConnection based
  on the given configuration.
\item exception: \sout{N/A} \rt{ConfigurationError --- thrown if the configuration is invalid or missing required information.}
\end{itemize}


\noindent closeRemoteConneciton(pc: RTCPeerConnection):
\begin{itemize}
\item transition: pc.signalingState := closed --- closes peer connection and send a
  signal to the connected peer connection.
\item output: N/A
\item exception: \sout{N/A} \rt{InvalidStateError --- thrown if the connection is already closed or not in a valid state to be closed.}
\end{itemize}


\noindent negotiate(pc: RTCPeerConnection):
\begin{itemize}
\item transition:

  pc.localDescription := RTCSessionDescriptionInit

  pc.remoteDescription := RTCSessionDescriptionInit

  sets the local description of the peer connection to its generated SDP, and
  set the remote description of the peer connection to its received SDP from
  SFU\_BROADCAST\_API.
\item output: N/A
\item exception: \sout{N/A} \rt{NegotiationError --- thrown if the negotiation fails due to invalid SDP or connection state.}
\end{itemize}


\noindent getRemoteStream(pc: RTCPeerConnection):
\begin{itemize}
\item transition: pc.event := getRemoteEvent(pc).streams
\item output: N/A
\item exception: \sout{N/A} \rt{StreamNotFoundError --- thrown if the remote stream cannot be found or is not accessible.}
\end{itemize}

\subsubsection{Local Functions}

\noindent getRemoteEvent(pc: RTCPeerConnection):
\begin{itemize}
\item transition: N/A
\item output: pc.event := RTCTrackEvent
\item exception: \sout{N/A} \rt{EventNotFoundError --- thrown if the event related to the remote track is not found or is not triggered.}
\end{itemize}

\section{MIS of Media Control Module} \label{sec:mediactrl}

\subsection{Module}

MediaContorl

\subsection{Uses}

\noindent Web APIs

\subsection{Syntax}

\subsubsection{Exported Constants}

N/A

\subsubsection{Exported Access Programs}

\begin{center}
  \begin{tabular}{p{4cm} p{4cm} p{4cm} p{2cm}}
    \hline
    \textbf{Name}          & \textbf{In}   & \textbf{Out}      & \textbf{Exceptions} \\
    \hline
    setMicEnabled    & Boolean & -           & -             \\
    setCameraEnabled & Boolean & -           & -             \\
    getStream        & -       & MediaStream & -             \\
    \hline
  \end{tabular}
\end{center}

\subsection{Semantics}

\subsubsection{State Variables}

\noindent isMicEnabled: Boolean

\noindent isCameraEnabled: Boolean

\subsubsection{Environment Variables}

\noindent Microphone

\noindent Camera

\subsubsection{Assumptions}

User's devices have a functioning screen, camera and microphone.

\subsubsection{Access Routine Semantics}

\noindent setMicEnabled(isEnabled: Boolean):
\begin{itemize}
\item transition: isMicEnabled := isEnabled
\item output: N/A
\item exception: N/A
\end{itemize}

\section{MIS of Media Control Module} \label{sec:mediactrl}

\subsection{Module}

MediaContorl

\subsection{Uses}

\noindent Web APIs

\subsection{Syntax}

\subsubsection{Exported Constants}

N/A

\subsubsection{Exported Access Programs}

\begin{center}
  \begin{tabular}{p{4cm} p{4cm} p{4cm} p{2cm}}
    \hline
    \textbf{Name}          & \textbf{In}   & \textbf{Out}      & \textbf{Exceptions} \\
    \hline
    setMicEnabled    & Boolean & -           & -             \\
    setCameraEnabled & Boolean & -           & -             \\
    getStream        & -       & MediaStream & -             \\
    \hline
  \end{tabular}
\end{center}

\subsection{Semantics}

\subsubsection{State Variables}

\noindent isMicEnabled: Boolean

\noindent isCameraEnabled: Boolean

\subsubsection{Environment Variables}

\noindent Microphone

\noindent Camera

\subsubsection{Assumptions}

User's devices have a functioning screen, camera and microphone.

\subsubsection{Access Routine Semantics}

\noindent setMicEnabled(isEnabled: Boolean):
\begin{itemize}
\item transition: isMicEnabled := isEnabled
\item output: N/A
\item exception: \sout{N/A} \rt{DeviceAccessError --- thrown if the microphone cannot be accessed or permissions are not granted.}
\end{itemize}

\noindent setCameraEnabled(isEnabled: Boolean):
\begin{itemize}
\item transition: isCameraEnabled := isEnabled
\item output: N/A
\item exception: \sout{N/A} \rt{DeviceAccessError --- thrown if the microphone cannot be accessed or permissions are not granted.}
\end{itemize}

\noindent getStream():
\begin{itemize}
\item transition: N/A
\item output: returns the user media stream based on the state value
  isCameraEnabled and isMicEnabled
\item exception: \sout{N/A} \rt{DeviceAccessError --- thrown if the microphone cannot be accessed or permissions are not granted.}
\end{itemize}

\subsubsection{Local Functions}

N/A


\section{MIS of Instructor View Module} \label{sec:instrcview}

\subsection{Module}

Instructor

\subsection{Uses}

\noindent Media Control Module

\noindent RTC Control Module

\noindent Annotation Configuration Module

\noindent React

\noindent Web APIs

\subsection{Syntax}

\subsubsection{Exported Constants}

N/A

\subsubsection{Exported Access Programs}

\begin{center}
  \begin{tabular}{p{4cm} p{4cm} p{4cm} p{2cm}}
    \hline
    \textbf{Name}    & \textbf{In} & \textbf{Out}          & \textbf{Exceptions} \\
    \hline
    Instructor & -     & React.component & -             \\
    \hline
  \end{tabular}
\end{center}

\subsection{Semantics}

\subsubsection{State Variables}

\noindent remoteVideoRef: HTMLVideoElement

\noindent selfVideoRef: HTMLVideoElement

\noindent peerConnection: RTCPeerConnection

\subsubsection{Environment Variables}

\noindent Screen

\subsubsection{Assumptions}

User's devices have a functioning screen, camera and microphone.

\subsubsection{Access Routine Semantics}

\noindent Instructor():
\begin{itemize}
\item transition: N/A
\item output: renders a react component of the instructor view page
\item exception: \sout{N/A} \rt{RenderError --- thrown if the component fails to render.}
\end{itemize}

\subsubsection{Local Functions}

\noindent setPeerConnection(pc: RTCPeerConnection):
\begin{itemize}
\item transition: peerConnection := pc
\item output: N/A
\item exception: \sout{N/A} \rt{ConnectionError --- thrown if the peer connection cannot be established.}
\end{itemize}

\noindent getSelfVideo():
\begin{itemize}
\item transition:

  selfVideoRef.current.video.srcObject:= MediaControl.getStream()

  render video stream from the local camera to screen
\item output: N/A
\item exception: \sout{N/A} \rt{VideoStreamError --- thrown if the local video stream cannot be accessed or is not available.}
\end{itemize}

\noindent startRemoteSharing():
\begin{itemize}
\item transition: peerConnection.addTrack := MediaControl.getStream()
\item output: N/A
\item exception: \sout{N/A} \rt{ShareStartError --- thrown if the stream cannot be added to the peer connection or sharing cannot be initiated.}
\end{itemize}

\noindent stopRemoteSharing():
\begin{itemize}
\item transition:

  remoteVideoRef.current.video.srcObject = null

  peerConnection.close:= true

  stops the remote video on the user’s screen and close the RTCPeerConnection
\item output: N/A
\item exception: \sout{N/A} \rt{ShareStopError --- thrown if stopping the remote sharing fails or the connection cannot be closed.}
\end{itemize}

\noindent getRemoteVideo():
\begin{itemize}
\item transition: get remote video coming from the SFU server and render it on the
  user's screen.
\item output: N/A
\item exception: \sout{N/A} \rt{RemoteVideoError --- thrown if the remote video cannot be retrieved or displayed.}
\end{itemize}


\section{MIS of Practitioner View Module} \label{sec:pracview}

\subsection{Module}

Practitioner

\subsection{Uses}

\noindent Media Control Module

\noindent RTC Control Module

\noindent Annotation Configuration Module

\noindent React

\noindent Web APIs

\subsection{Syntax}

\subsubsection{Exported Constants}

N/A

\subsubsection{Exported Access Programs}

\begin{center}
  \begin{tabular}{p{4cm} p{4cm} p{4cm} p{2cm}}
    \hline
    \textbf{Name}      & \textbf{In} & \textbf{Out}          & \textbf{Exceptions} \\
    \hline
    Practitioner & -     & React.component & -             \\
    \hline
  \end{tabular}
\end{center}

\subsection{Semantics}

\subsubsection{State Variables}

\noindent remoteVideoRef: HTMLVideoElement

\noindent peerConnection: RTCPeerConnection

\subsubsection{Environment Variables}

\noindent Screen

\subsubsection{Assumptions}

User's devices have a functioning screen.

\subsubsection{Access Routine Semantics}

N/A

\subsubsection{Local Functions}

\noindent setPeerConnection(pc: RTCPeerConnection):
\begin{itemize}
\item transition: peerConnection := pc
\item output: N/A
\item exception: \sout{N/A} \rt{ConnectionSetupFailure --- thrown if the peer connection cannot be set due to an invalid or null `pc` argument, or if the connection setup fails.}
\end{itemize}

\noindent getRemoteVideo():
\begin{itemize}
\item transition: get remote video coming from the SFU server and render it on the
  user's screen.
\item output: N/A
\item exception: \sout{N/A} \rt{VideoRetrievalError --- thrown if the remote video cannot be retrieved from the SFU server, or if there is an error in rendering the video on the screen.}
\end{itemize}


\section{MIS of Annotation Configuration Module} \label{sec:annoconfig}

\subsection{Module}

AnnotationConfig

\subsection{Uses}

\noindent RTC Control Module

\noindent React

\subsection{Syntax}

\subsubsection{Exported Constants}

N/A

\subsubsection{Exported Access Programs}

\begin{center}
  \begin{tabular}{p{4cm} p{4cm} p{4cm} p{2cm}}
    \hline
    \textbf{Name}              & \textbf{In}   & \textbf{Out}  & \textbf{Exceptions} \\
    \hline
    setIsSkeletonEnabled & Boolean & -       & -             \\
    setIsCOMEnabled      & Boolean & -       & -             \\
    getIsSkeletonEnable  & -       & Boolean & -             \\
    getIsCOMEnable       & -       & Boolean & -             \\
    \hline
  \end{tabular}
\end{center}

\subsection{Semantics}

\subsubsection{State Variables}

\noindent isSkeletonEnabled: Boolean

\noindent isCOMEnabled: Boolean

\subsubsection{Environment Variables}

N/A

\subsubsection{Assumptions}

N/A

\subsubsection{Access Routine Semantics}

\noindent setIsSkeletonEnabled(isEnabled: Boolean):
\begin{itemize}
\item transition: isSkeletonEnabled := isEnabled
\item output: N/A
\item exception: \sout{N/A} \rt{FeatureToggleError --- thrown if there is an error in toggling the skeleton visualization state.}
\end{itemize}

\noindent setIsCOMEnabled(isEnabled: Boolean):
\begin{itemize}
\item transition: isCOMEnabled := isEnabled
\item output: N/A
\item exception: \sout{N/A} \rt{FeatureToggleError --- thrown if there is an error in toggling the COM visualization state.}
\end{itemize}

\noindent getIsSkeletonEnabled():
\begin{itemize}
\item transition: N/A
\item output: isSkeletonEnabled
\item exception: \sout{N/A} \rt{StateRetrievalError --- thrown if the current state of the skeleton feature cannot be retrieved.}
\end{itemize}

\noindent getIsCOMEnabled():
\begin{itemize}
\item transition: N/A
\item output: isCOMEnabled
\item exception: \sout{N/A} \rt{StateRetrievalError --- thrown if the current state of the COM feature cannot be retrieved.}
\end{itemize}


\subsubsection{Local Functions}

N/A


\section{MIS of App Module} \label{sec:appmodule}

\subsection{Module}
App

\subsection{Uses}
RTC Control Module

Media Control Module

Instructor View Module

Practitioner View Module

Annotation Configuration Module

User Authentication Module

\subsection{Syntax}

\subsubsection{Exported Constants}
None

\subsubsection{Exported Access Programs}
\begin{table}[h!]
  \centering
  \begin{tabular}{llll}
    \hline
    \textbf{Name} & \textbf{In} & \textbf{Out}          & \textbf{Exceptions} \\
    \hline
    App     & -     & React.component & -             \\
    \hline
  \end{tabular}
\end{table}

\subsection{Semantics}

\subsubsection{State Variables}
N/A

\subsubsection{Environment Variables}
N/A

\subsubsection{Assumptions}
N/A

\subsubsection{Access Routine Semantics}

\noindent App():
\begin{itemize}
\item transition: App:= React.component()

  starts React App and render it on the user's device
\item output: N/A
\item exception: N/A
\end{itemize}

\subsubsection{Local Functions}
N/A

\section{MIS of User Authentication Module} \label{sec:userauth}

\subsection{Module}
Auth

\subsection{Uses}
Instructor View Module

Practitioner View Module

\subsection{Syntax}

\subsubsection{Exported Constants}
N/A

\subsubsection{Exported Access Programs}
\begin{table}[h!]
  \centering
  \begin{tabular}{llll}
    \hline
    \textbf{Name} & \textbf{In} & \textbf{Out}          & \textbf{Exceptions} \\
    \hline
    Auth    & -     & React.component & -             \\
    \hline
  \end{tabular}
\end{table}

\subsection{Semantics}

\subsubsection{State Variables}

isUserInstructor: Boolean

\subsubsection{Environment Variables}

N/A

\subsubsection{Assumptions}

N/A

\subsubsection{Access Routine Semantics}

\noindent Auth():
\begin{itemize}
\item transition: Render the authentication page on the user's device, if the user
  clicks on the Instructor button, then jumps to the instructor view page, if
  the user clicks on the practitioner button, jumps to the practitioner view
  page.
\item output: N/A
\item exception: N/A
\end{itemize}

\subsubsection{Local Functions}

\noindent isUserInstructor $\rightarrow$ Instructor view else Practitioner view

\noindent setIsUserInstructor(isEnabled: Boolean):
\begin{itemize}
  \item transition: isUserInstructor := isEnabled
  \item output: N/A
  \item exception: N/A
\end{itemize}

\section{MIS of Video Transform Module} \label{sec:videotransform}

\subsection{Module}
VideoTransformTrack

\subsection{Uses}
HPE, CM

\subsection{Syntax}

\subsubsection{Exported Constants}
kind = ``video''

\subsubsection{Exported Access Programs}
\begin{table}[h!]
  \centering
  \begin{tabular}{llll}
    \hline
    \textbf{Routine name} & \textbf{In}            & \textbf{Out}     & \textbf{Exceptions} \\
    \hline
    \_\_init\_\_    & track, transform & -          & -             \\
    recv            & -                & VideoFrame & -             \\
    \hline
  \end{tabular}
\end{table}

\subsection{Semantics}

\subsubsection{State Variables}

track: VideoStreamTrack

transform: string

\subsubsection{Environment Variables}
N/A

\subsubsection{Assumptions}
\_\_init\_\_ is called before any other access program

\subsubsection{Access Routine Semantics}

\noindent \_\_init\_\_(track, transform):
\begin{itemize}
  \item transition: initiated by track and transform, self.track = track, self.transform = transform
  \item output: out := self
  \item exception: \sout{N/A} \rt{TransformationError --- thrown if the transform type is unrecognized or if the transformation process fails.}
\end{itemize}

\noindent recv(self):
\begin{itemize}
\item transition: Processes a video frame (frame) received from a track. Depending
  on the value of self.transform, it applies one of the following
  transformations:
  \begin{itemize}
  \item ``HPE'': Converts the frame by applying the HPE module
    annotation.
  \item ``CM'': Converts the frame by applying the CM module annotation.
  \item If self.transform is set to any other value, the frame is returned without any
    transformation.
  \end{itemize}
\item output: Returns a new VideoFrame object (new\_frame) that has undergone the
  specified transformation, preserving the original frames timing information
  (timestamps and time base).
  \item exception: \sout{N/A} \rt{InvalidFrameError --- thrown if the frame is null or corrupted; TransformNotAppliedError --- thrown if the transformation cannot be applied.}
\end{itemize}

\subsubsection{Local Functions}
N/A

\section{MIS of SFU Server Module} \label{sec:sfuserver}

\subsection{Module}
SfuServer

\subsection{Uses}
VideoTransformTrack

\subsection{Syntax}

\subsubsection{Exported Constants}
N/A

\subsubsection{Exported Access Programs}
\begin{table}[h!]
  \centering
  \begin{tabular}{llll}
    \hline
    \textbf{Routine name} & \textbf{In}   & \textbf{Out} & \textbf{Exceptions} \\
    \hline
    consumer        & request & -      & -             \\
    broadcast       & request & -      & -             \\
    \hline
  \end{tabular}
\end{table}

\subsection{Semantics}

\subsubsection{State Variables}
N/A

\subsubsection{Environment Variables}
relay: MediaRelay

consumer\_track: VideoStreamTrack

\subsubsection{Assumptions}
N/A

\subsubsection{Access Routine Semantics}

\noindent consumer(request):
\begin{itemize}
\item transition: Processes a WebRTC connection request. The function performs the
  following actions:
  \begin{itemize}
  \item Parses the request to extract session description parameters.
  \item Creates a new RTCPeerConnection object.
  \item Logs the information about the sent track.
  \item Adds a VideoTransformTrack to the peer connection, which includes subscribing
    to a consumer track and applying a specified video transformation.
  \item Sets the remote description of the peer connection based on the received session description.
  \item Creates and sets a local description for the peer connection by generating an answer to the received offer.
  \end{itemize}
\item output: Returns a web response in JSON format. This response contains the SDP
  data and the type of the local description set on the peer connection.
\item exception: \sout{N/A} \rt{ConnectionSetupError --- thrown if the RTCPeerConnection cannot be established; OfferProcessingError --- thrown if the offer cannot be processed or if setting the local/remote description fails.}
\end{itemize}

\noindent broadcast(request):
\begin{itemize}
\item transition: Manages the setup and handling of a WebRTC peer connection for
  broadcasting.

  \begin{itemize}
  \item Parses the incoming request to extract the SDP data.
  \item Initializes a new RTCPeerConnection.
  \item Adds the peer connection to a global set and logs relevant information.
  \item Sets up event handlers for different peer connection events:
    \begin{enumerate}
    \item Connection State Change: Monitors the connection state, logging changes
      and closing the connection if it fails.
    \item Track Reception: Handles received tracks, particularly video tracks, by
      setting a global consumer\_track for later use, and logs when tracks end.
    \item Processes the received offer by setting it as the remote description of
      the peer connection.
    \item Creates and sets a local description for the peer connection in response
      to the offer.
    \end{enumerate}
  \end{itemize}
\item output: Returns a web response in JSON format, containing the SDP data and
  the type of the local description set on the peer connection.
\item exception: \sout{N/A} \rt{ConnectionSetupError --- thrown if the RTCPeerConnection cannot be established; OfferProcessingError --- thrown if the offer cannot be processed or if setting the local/remote description fails.}
\end{itemize}

\subsubsection{Local Functions}
N/A

\section{MIS of Human Pose Estimation Annotation Module} \label{sec:hpe}

\subsection{Module}
HPE

\subsection{Uses}
Numpy, CV2, OS, Sys, Time, Subprocess, Shutil, Socket

\subsection{Syntax}

\subsubsection{Exported Constants}
server\_address, HPE\_address, K, pose, Rt1, R1, t1, P1, Identity, P2

\subsubsection{Exported Access Programs}
\begin{table}[h!]
  \centering
  \begin{tabular}{llll}
    \hline
    \textbf{Name}      & \textbf{In}      & \textbf{Out} & \textbf{Exceptions}       \\
    \hline
    get\_kpts    & Image      & List   & IOError, ValueError \\
    measureJoint & List, List & Tuple  & N/A                 \\
    matchKpts    & List       & List   & N/A                 \\
    get3D        & List, List & List   & N/A                 \\
    \hline
  \end{tabular}
\end{table}

\subsection{Semantics}

\subsubsection{State Variables}
N/A

\subsubsection{Environment Variables}
N/A

\subsubsection{Assumptions}
External libraries are functioning as expected

\subsubsection{Access Routine Semantics}
\noindent get\_kpts(img):
\begin{itemize}
\item transition: Saves the input image to a designated path and calls OpenPose to
  generate keypoints, which are then saved to a JSON file.
\item output: Returns a list of keypoints extracted from the input image.
\item exception: IOError if image saving or reading fails, ValueError if keypoints
  processing fails.
\end{itemize}

\noindent measureJoint(kpts1, kpts2):
\begin{itemize}
\item transition: Computes the length of the spine in each set of keypoints and
  returns them ordered by length.
\item output: Returns a tuple with the first element being the keypoints set with
  the longer spine.
\item exception: \sout{N/A} \rt{KeyPointError --- thrown if keypoints are invalid or insufficient to compute the spine length.}
\end{itemize}

\noindent matchKpts(mirror\_img):
\begin{itemize}
\item transition: Reflects the keypoints from the mirror image to match the real
  image.
\item output: Returns the adjusted keypoints for the mirrored image.
\item exception: \sout{N/A} \rt{ReflectionError --- thrown if keypoints cannot be reflected properly due to incorrect format or data corruption.}
\end{itemize}

\noindent get3D(real\_kpts, mirror\_kpts):
\begin{itemize}
\item transition: Uses the keypoints from the real and mirror images to triangulate
  3D points.
\item output: Returns the 3D coordinates of the keypoints.
\item exception: \sout{N/A} \rt{TriangulationError --- thrown if 3D triangulation cannot be performed due to invalid or mismatched keypoints.}
\end{itemize}

\subsubsection{Local Functions}
N/A

\section{MIS of Center of Mass Annotation Module} \label{sec:centerofmass}

\subsection{Module}
CM

\subsection{Uses}

\noindent numpy: for numerical computations

\noindent params.bodySegParams: for body segmentation parameters

\noindent params.cameraParams: for camera parameters

\subsection{Syntax}

\subsubsection{Exported Constants}

\noindent K, pose, P1, P2, R1, t1, R2, t2 - Camera intrinsic and extrinsic parameters, and
projection matrices derived from them.

\noindent foot\_in\_air\_thresh - Threshold for determining if a foot is in the air.
\noindent CoM\_foot\_thresh - Threshold for determining the supporting foot based on the center of mass.

\subsubsection{Exported Access Programs}
\begin{tabularx}{\textwidth}{X X X X}
  \hline
  \textbf{Name}    & \textbf{In}                                                 & \textbf{Out}                                      & \textbf{Exceptions} \\
  \hline
  getCoM     & points\_3D: 3D points array                           & CoM: Center of Mass point                   & -             \\
  feetStates & CoM: Center of Mass point points\_3D: 3D points array & left\_foot, right\_foot: States of the feet & -             \\
  \hline
\end{tabularx}

\subsection{Semantics}

\subsubsection{State Variables}
N/A

\subsubsection{Environment Variables}
N/A

\subsubsection{Assumptions}
The module assumes that body segment parameters and camera calibration data
provided by the bodySegParams and cameraParams modules are accurate and
reliable.

\subsubsection{Access Routine Semantics}

\noindent getCoM(points\_3D):
\begin{itemize}
\item transition: Calculates the center of mass based on the 3D points of body
  joints.
\item output: Returns the 3D coordinates of the bodys center of mass.
\item exception: \sout{N/A} \rt{CalculationError --- thrown if the center of mass cannot be calculated, possibly due to invalid or insufficient 3D points.}
\end{itemize}

\noindent feetStates(CoM, points\_3D):
\begin{itemize}
\item transition: Determines the state of each foot (left and right) based on their
  position relative to the center of mass and the vertical distance from the
  ground.
\item output: Returns a tuple containing two dictionaries, left\_foot and
  right\_foot, each indicating whether the respective foot is on the ground and
  whether it is supporting body weight.
\item exception: \sout{N/A} \rt{StateDeterminationError --- thrown if the states of the feet cannot be determined, perhaps due to invalid center of mass or 3D points data.}
\end{itemize}

\subsubsection{Local Functions}
N/A

\section{MIS of STUN Server Module} \label{sec:stunserver}

\subsection{Module}
STUN

\subsection{Uses}
N/A

\subsection{Syntax}

\subsubsection{Exported Constants}
STUN\_SERVER\_ADDRESS

\subsubsection{Exported Access Programs}
N/A

\subsection{Semantics}

\subsubsection{State Variables}
N/A

\subsubsection{Environment Variables}
N/A

\subsubsection{Assumptions}
The module assumes that a public STUN server is readily available.

\subsubsection{Access Routine Semantics}
N/A

\subsubsection{Local Functions}
N/A

\newpage

\bibliographystyle {plainnat}
\bibliography {../../../refs/References}

\newpage

\section{Appendix} \label{Appendix}

\wss{Extra information if required}

\end{document}
\documentclass[12pt, titlepage]{article}

\usepackage{booktabs}
\usepackage{tabularx}
\newcolumntype{L}[1]{>{\raggedright\let\newline\\\arraybackslash\hspace{0pt}}p{#1}}
\usepackage{hyperref}
\hypersetup{
    colorlinks,
    citecolor=blue,
    filecolor=black,
    linkcolor=red,
    urlcolor=blue
}
\usepackage[round]{natbib}

\usepackage[shortlabels]{enumitem}
\usepackage{float}
\usepackage{geometry}
\usepackage{pdflscape}
\usepackage{siunitx}

%% Comments

\usepackage{color}

\newif\ifcomments\commentstrue %displays comments
%\newif\ifcomments\commentsfalse %so that comments do not display

\ifcomments
\newcommand{\authornote}[3]{\textcolor{#1}{[#3 ---#2]}}
\newcommand{\todo}[1]{\textcolor{red}{[TODO: #1]}}
\else
\newcommand{\authornote}[3]{}
\newcommand{\todo}[1]{}
\fi

\newcommand{\wss}[1]{\authornote{blue}{SS}{#1}} 
\newcommand{\plt}[1]{\authornote{magenta}{TPLT}{#1}} %For explanation of the template
\newcommand{\an}[1]{\authornote{cyan}{Author}{#1}}

%% Common Parts

\newcommand{\progname}{SFWRENG 4G06 Capstone Design Project} % PUT YOUR PROGRAM NAME HERE
\newcommand{\projname}{MotionMingle} % project name
\newcommand{\authname}{Team \#18, InfiniView-AI
\\ Anhao Jiao
\\ Kehao Huang
\\ Qianlin Chen
\\ Shu Qi
\\ Xunzhou Ye
} % AUTHOR NAMES

\usepackage{hyperref}
    \hypersetup{colorlinks=true, linkcolor=blue, citecolor=blue, filecolor=blue,
                urlcolor=blue, unicode=false}
    \urlstyle{same}


\begin{document}

\title{Verification and Validation Report: \progname} 
\author{\authname}
\date{\today}
	
\maketitle

\pagenumbering{roman}

\section{Revision History}

\begin{tabularx}{\textwidth}{llX}
  \toprule {\bf Date} & {\bf Developer(s)} & {\bf Change} \\
  \midrule  
Mar 6 & Qi Shu & Add "Comparison to existing implementations section" and "Changes due to Testing" section.\\
Mar 6 & Anhao Jiao, Xunzhou Ye & Add "FR Eval" and "NFR Eval" section.\\
Mar 6 & Qianlin Chen & Add Unit Tesing for SFU Server Module and VideoTransformTrack Module;
                      Add Automatic Tesing;
                      Add Code Coverage for Backend Modules.\\
\bottomrule
\end{tabularx}

~\newpage

\section{Symbols, Abbreviations and Acronyms}

\renewcommand{\arraystretch}{1.2}
\begin{tabular}{l l} 
  \toprule		
  \textbf{symbol} & \textbf{description}\\
  \midrule 
  T & Test\\
  \bottomrule
\end{tabular}\\

\wss{symbols, abbreviations or acronyms -- you can reference the SRS tables if needed}

\newpage

\tableofcontents

\listoftables %if appropriate

\listoffigures %if appropriate

\newpage

\pagenumbering{arabic}

This document ...

\section{Functional Requirements Evaluation}

\begin{enumerate}[FR-T1]
  \item \label{FRT1}
    \begin{description}
    \item[Initial State] Client application is running on the user's device, but the
      user didn’t do any operations yet.
    \item[Input/Condition] User clicks on the applicable
      identity(instructor/practitioner) button
    \item[Expected Output] The live stream video window pops out on the user's screen.
    \item[Actual Output] The live stream video window shows on the instructor's screen, 
    it doesn't show on the practitioner's screen.
    \item[Result] Pass
    \end{description}
  \item \label{FRT2}
    \begin{description}
    \item[Initial State] Application running on user’s computer, and the user has
      clicked on “the instructor identity button” to indicate they are a TaiChi
      instructor. A window asking for permission to use the camera on the
      instructor's device popped out.
    \item[Input/Condition] User allow/deny the webcam permission
    \item[Expected Output] The webcam on the instructor’s device is turned on
    \item[Actual Output] The webcam on the instructor’s device is turned on after 
    allowing webcam permission, and the webcam is not turned on after denying permission.
    \item[Result] Pass
    \end{description}
  \item \label{FRT3}
    \begin{description}
    \item[Initial State] both client applications and the server are running.
    \item[Input/Condition] The user clicks on the applicable identity button to
      indicate they are an instructor or a practitioner.
    \item[Expected Output] A log message indicates connection between the user’s device
      and the server has been established.
    \item[Actual Output] Log message confirms a successful connection between the user’s 
    device and the server for both instructor and practitioner.
    \item[Result] Pass
    \end{description}
  \item \label{FRT4}
    \begin{description}
    \item[Type] Functional, Dynamic, Automated
    \item[Initial State] The live stream Window for practitioners.
    \item[Input/Condition] The user’s device has established a connection with the
      server as a practitioner device.
    \item[Expected Output] A request from the client device to the server for accessing
      the list of available annotation configuration.
    \item[Actual Output] The client device sends a request to the server for the list 
    of available annotation configurations.
    \item[Result] Pass
    \end{description}
  \item \label{FRT5}
    \begin{description}
    \item[Initial State] The selectable list of the type of annotations is
      rendered on the user's screen.
    \item[Input/Condition] Practitioner’s selection on the list of types of
      annotations.
    \item[Expected Output] A request(that reflects user’s annotation selection) from
      the client device to the server for updating the annotation configuration,
      with a log indicating the request is sent.
    \item[Actual Output] The client device sends the selected annotation configuration 
    update to the server, and a log entry confirms the request was sent.
    \item[Result] Pass
    \end{description}
  \item \label{FRT6}
    \begin{description}
    \item[Initial State] The system is running and actively connected to
      practitioners.
    \item[Input/Condition] Practitioners initiate updates to annotation
      configurations.
    \item[Expected Output] The system receives and processes the updated annotation
      configurations.
    \item[Actual Output] The server gets the request and user sees the annotation they selected.
    \item[Result] Pass
    \end{description}
  \item \label{FRT7}
    \begin{description}
    \item[Initial State] The server is running and actively receiving annotation
      configuration updates.
    \item[Input/Condition] In a controlled test environment, the practitioner-client
      initiates the update of an annotation configuration. The update is sent to
      the server for processing.
    \item[Expected Output] The expected result is that the server correctly processes
      the received annotation configuration from the practitioner-client.
    \item[Actual Output] The server is not able to receive annotation
    configuration updates while the connection has been established.
    \item[Result] Fail
    \end{description}
  \item \label{FRT8}
    \begin{description}
    \item[Initial State] The server has received and processed the annotation
      configuration.
    \item[Input/Condition] The server uses the received annotation configuration to
      configure machine learning pipelines.
    \item[Expected Output] The machine learning pipelines are arranged and configured
      based on the annotation configuration.
    \item[Actual Output] The server successfully arranges and configures machine 
    learning pipelines according to the received annotation configuration.
    \item[Result] Pass
    \end{description}
  \item \label{FRT9}
    \begin{description}
    \item[Initial State] The machine learning pipelines are configured and active.
    \item[Input/Condition] The instructor's video stream is processed with the
      annotation configuration.
    \item[Expected Output] The instructor's video stream is rendered with
      annotations.
    \item[Actual Output] The instructor's video stream is processed with the 
    annotations.
    \item[Result] Pass
    \end{description}
  \item \label{FRT10}
    \begin{description}
    \item[Initial State] The server is actively connected to practitioner clients.
    \item[Input/Condition] The annotated video stream is generated and ready for
      transmission.
    \item[Expected Output] The annotated video stream is transmitted to each
      practitioner-client through their established connections.
    \item[Actual Output] The annotated video stream is transmitted to practitioner clients.
    \item[Result] Pass
    \end{description}
  \item \label{FRT11}
    \begin{description}
    \item[Initial State] The signaling server is running.
    \item[Input/Condition] Signaling requests for WebRTC connections are initiated.
    \item[Expected Output] The signaling server consistently responds to requests and
      establishes WebRTC connections.
    \item[Actual Output] The signaling server responds to most requests and establishes WebRTC connections
    \item[Result] Pass
    \end{description}
  \item \label{FRT12}
    \begin{description}
    \item[Initial State] The client application is running, but the user didn’t do
      any operations yet
    \item[Input/Condition] The user joining video stream session.
    \item[Expected Output] A button to identify if a user is an instructor or a
      practitioner is rendered.
    \item[Actual Output] The button to identify a user as an instructor or a 
    practitioner is rendered without issues
    \item[Result] Pass
    \end{description}
  \end{enumerate}

\section{Nonfunctional Requirements Evaluation}

\begin{enumerate}[NFR-T1]
  \item \label{NFRT1}
    \begin{description}
    \item[Initial State] The system is in a typical operational state with all
      components and services running, including the user interface.
    \item[Input/Condition] User interactions such as button clicks and menu
      selections.
    \item[Expected Output] The user is able to interact with the client application and
      understand the response from and results yielded by the system.
    \item[Actual Output] All 5 users are able to successfully interact with the application 
    with understandable responses and results from the system.
    \item[Result] Pass
    \end{description}
  \item \label{NFRT2}
    \begin{description}
    \item[Initial State] The system is in a typical operational state with all
      components and services running, including the user interface.
    \item[Input/Condition] User interactions such as button clicks and menu
      selections.
    \item[Expected Output] System response to user interactions.
    \item[Actual Output] The system responds to user interactions promptly and correctly.
    \item[Result] Pass 
    \end{description}
  \item \label{NFRT3}
    \begin{description}
    \item[Initial State] The system is in a stable operational state with necessary
      services and components active. No users are currently logged in.
    \item[Input/Condition] 5 test accounts (or more) for users. Predefined user
      actions/scripts (relevant to normal system interactions during peak
      periods).
    \item[Expected Output] The system remains stable and responsive. All user
      transactions are processed successfully.
    \item[Actual Output] The system remains stable and responsive with all user 
    transactions processed successfully.
    \item[Result] Pass 
    \end{description}
  \item \label{NFRT4}
    \begin{description}
    \item[Initial State] The system is operational, with all services and components
      running. User accounts for test are set up, and no tasks are being
      performed.
    \item[Input/Condition] Erroneous user input data. Improper user actions.
    \item[Expected Output] The system detects and rejects invalid inputs or actions,
      providing clear and helpful error messages to the user. Transactions or
      actions are either rolled back safely or do not proceed until valid input is
      provided.
    \item[Actual Output] No invalid inputs can be made within the system.
    \item[Result] Pass 
    \end{description}
  \stepcounter{enumi}
  \item \label{NFRT6}
    \begin{description}
    \item[Initial State] The system, with the SFU component, is in a stable,
      operational state. Initially, there is a base number of video streams being
      processed, which is within the SFU's current capacity.
    \item[Input/Condition] A load testing tool or custom script capable of simulating
      multiple, simultaneous video streams.
    \item[Expected Output] The SFU successfully processes an increasing number of video
      streams.
    \item[Actual Output] The SFU processes an increased number of video streams, which was up
    to 5 users without significant issues.
    \item[Result] Pass 
    \end{description}
  \item \label{NFRT7}
    \begin{description}
    \item[Initial State] This system is in a typical operational state with all
      components and services running, the user is in a live session.
    \item[Input/Condition] Network Interruption/Network Resumption
    \item[Expected Output] The system attempts to resume the previous session.
    \item[Actual Output] The system did not resume to the previous session.
    \item[Result] Fail
    \end{description}
  \item \label{NFRT8}
    \begin{description}
    \item[Initial State] This system is in a typical operational state with all
      components and services running, the user is in a live session.
    \item[Input/Condition] Network instability
    \item[Expected Output] A pop up window for network instability warning.
    \item[Actual Output] There was no pop up window for network instability warning.
    \item[Result] Fail
    \end{description}
  \item \label{NFRT9}
    \begin{description}
    \item[Initial State] This system is in a typical operational state with all
      components and services running.
    \item[Input/Condition] Turn down the primary signaling server.
    \item[Expected Output] The redundant server takes over until the primary server
      resumes.
    \item[Actual Output] Connections were able to be established without the primary 
    signaling server.
    \item[Result] Pass 
    \end{description}
  \item \label{NFRT10}
    \begin{description}
    \item[Initial State] The system is in a typical operational state with all
      components and services running. The user is in a live session with a media
      capturing device plugged in.
    \item[Input/Condition] Disconnect the media capturing device from the user’s
      computing device.
    \item[Expected Output] A pop up window which warns the user that the client
      application has lost connection to the media capturing device and prompts
      the user to reconnect the device to resume the live session.
    \item[Actual Output] When the media capturing device is disconnected, a pop-up 
    window alerts the user and prompts for reconnection.
    \item[Result] Pass 
    \end{description}
  \item \label{NFRT11}
    \begin{description}
    \item[Initial State] The system is in a typical operational state with all
      components and services running.
    \item[Input/Condition] Connect a media capturing device to the user’s computing
      device.
    \item[Expected Output] A pop up window which warns the user that the video
      capturing device does not meet required specification if the client
      application detects that the video input stream has a resolution lower than
      MIN\_RES.
    \item[Actual Output] The system was not able to check the specification of the capturing 
    device
    \item[Result] Fail 
    \end{description}
  \item \label{NFRT12}
    \begin{description}
    \item[Initial State] The system is in a typical operational state with all
      components and services running.
    \item[Input/Condition] the user starts a live session and begins streaming.
    \item[Expected Output] Accurate instructional annotations are rendered and
      overlaid on the instructor’s video stream.
    \item[Actual Output] The system renders accurate instructional annotations on 
    the instructor’s video stream.
    \item[Result] Pass 
    \end{description}
  \item \label{NFRT13}
    \begin{description}
    \item[Initial State] The system is in a typical operational state with all
      components and services running.
    \item[Input/Condition] Connect more than one media capturing device to the
      user’s computing device.
    \item[Expected Output] A dialog listing all the detected capturing devices is
      displayed to prompt the user for selecting a device to use.
    \item[Actual Output] Feature handled by external API.
    \item[Result] Pass
    \end{description}
  \item \label{NFRT14}
    \begin{description}
    \item[Initial State] This system is in a typical operational state with all
      components and services running, with a live session created.
    \item[Input/Condition] The live video and audio stream from instructor client
      application.
    \item[Expected Output] Audio stream and video stream with annotations on
      practitioner client application.
    \item[Actual Output] The practitioner client application receives only video 
    streams with annotations.
    \item[Result] Fail
    \end{description}
  \item \label{NFRT15}
    \begin{description}
    \item[Initial State] The system is in a typical operational state with all
      components and services running.
    \item[Input/Condition] The user starts the client application for instructors.
    \item[Expected Output] A message with detailed instruction to set up a media
      capturing device and a list of cautions is displayed.
    \item[Actual Output] No message was displayed.
    \item[Result] Fail
    \end{description}
  \item \label{NFRT16}
    \begin{description}
    \item[Initial State] The software system is fully developed, stable, and ready
      for testing. The different test environments for the latest versions of
      Windows, Linux, and macOS are set up, each with default settings.
    \item[Input/Condition] The latest stable versions of Windows, Linux, and
      macOS. Test cases designed to cover all the main functionalities of the
      system.
    \item[Expected Output] The system functions correctly on all mentioned operating
      systems without crashes, unexpected behaviour, or significant performance
      issues.
    \item[Actual Output] The system functions correctly on all mentioned operating 
    systems without any significant issues.
    \item[Result] Pass 
    \end{description}
  \item \label{NFRT17}
    \begin{description}
    \item[Initial State] The system is fully developed, with all features
      operational, and is hosted in a stable environment accessible via the web.
      Test environments with the latest versions of Chrome, Firefox, Safari, and
      Edge are established, each configured with default browser settings.
    \item[Input/Condition] Latest stable versions of Chrome, Firefox, Safari, and
      Edge. A series of test cases designed to fully evaluate the functionalities
      and features of the system.
    \item[Expected Output] The system operates as intended on all tested web
      browsers without significant functionality issues, crashes, or severe
      performance degradation. Features and visual elements are consistent across
      all browsers.
    \item[Actual Output] The system operates as intended across all tested web 
    browsers with consistent features and visual elements.
    \item[Result] Pass 
    \end{description}
  \item \label{NFRT18}
    \begin{description}
    \item[Initial State] The system is fully developed and operational. The
      testing environments represent a range of standard personal computer and
      laptop configurations (covering various manufacturers, system
      specifications, and age of devices) equipped with cameras and, where
      applicable, microphones. The devices are running compatible operating
      systems with the necessary drivers installed.
    \item[Input/Condition] Range of standard computers or laptops with various
      specifications, but all within the commonly accepted 'standard' range for
      current users. Devices have functional cameras and optional microphones.
      Test script detailing system operation tasks.
    \item[Expected Output] The system operates without significant delays, errors,
      or crashes.
    \item[Actual Output] The system operates without significant delays, errors, 
    or crashes across a range of standard computers and laptops.
    \item[Result] Pass 
    \end{description}
  \item \label{NFRT19}
    \begin{description}
    \item[Initial State] The system is fully developed, with comprehensive
      documentation on its architecture, codebase, dependencies, and update
      procedures. The development environment is available for testing maintenance
      procedures, including tools for version control, testing, and deployment.
    \item[Expected Output] Successful completion of maintenance tasks without
    introducing significant issues or disruptions to the system's operation.
    \item[Actual Output] It is impossible to perform maintenance without 
    interrupting live users.
    \item[Result] Fail 
    \end{description}
  
  \addtocounter{enumi}{2}
  \item \label{NFRT22}
    \begin{description}
    \item[Initial State] The system is in a stable state, ready for testing. The
      user interface for account creation or any other information input is
      accessible, and documentation related to data handling is available.
    \item[Input/Condition] Guidelines or criteria specifying what constitutes
      "essential" versus "nonessential" information for the system’s purpose.
      Access to the system's user interfaces (UI) where data submission forms are
      present.
    \item[Expected Output] Comparison of requested data against the criteria for
      essential information.
    \item[Actual Output] The system requests only essential information from users.
    \item[Result] Pass 
    \end{description}
  \item \label{NFRT23}
    \begin{description}
    \item[Initial State] This system is in a typical operational state with all
      components and services running.
    \item[Input/Condition] Attempted unauthorized modifications to the system
      including access to system files and modifications to system configurations.
    \item[Expected Output] A determination of whether the system are able to prevent
      access or modifications from unauthorized users.
    \item[Actual Output] The system prevents all unauthorized access and modifications 
    during the test.
    \item[Result] Pass 
      \begin{itemize}
      \item Access system settings.
      \item Modify Critical system configuration files.
      \item Gain access to media streams in the system.
      \end{itemize}
      The test will be performed on both the instructor client application and
      practitioner client application. The test is considered successful if all
      attempts by the unauthorized user are prevented by the system. The test will
      be considered a failure if any successful unauthorized access or
      modification to the system were made during the testing process.
    \end{description}
  \item \label{NFRT24}
    \begin{description}
    \item[Initial State] The system is in a typical operational state with all
      components and services running.
    \item[Input/Condition] The user starts a live session and begins streaming.
    \item[Expected Output] A prompt for granting access to use the media capturing
      devices is displayed.
    \item[Actual Output] A prompt appears for granting access to media capturing 
    devices during a live session.
    \item[Result] Pass 
    \end{description}
  \item \label{NFRT25}
    \begin{description}
    \item[Initial State] The system is in a typical operational state with all
      components and services running.
    \item[Input/Condition] The user starts a live session and begins streaming.
    \item[Expected Output] An indicator that the media capturing device is in use
      and that the user is being recorded is visible in the client application.
    \item[Actual Output] An indicator is visible in the client application when 
    the media capturing device is in use.
    \item[Result] Pass 
    \end{description}
  \item \label{NFRT26}
    \begin{description}
    \item[Initial State] The system is in a typical operational state with all
      components and services running.
    \item[Input/Condition] The user ends a live session and stops streaming.
    \item[Expected Output] The indicator that the media capturing device is in use
      disappears. The media stream from the capturing device is no longer
      displayed on the screen.
    \item[Actual Output] The indicator disappears, and the media stream is no 
    longer displayed once the session ends.
    \item[Result] Pass 
    \end{description}
  \item \label{NFRT27}
    \begin{description}
    \item[Initial State] This system is in a typical operational state with all
      components and services running.
    \item[Input/Condition] User inputs
    \item[Expected Output] System response to user inputs
    \item[Actual Output] The system responds effectively to user inputs.
    \item[Result] Pass 
    \end{description}
  \item \label{NFRT28}
    \begin{description}
    \item[Initial State] The system is in a typical operational state with all
      components and services running.
    \item[Input/Condition] A set of ISO/IEC 12207 standards.
    \item[Expected Output] A determination of whether the system and its associated
      processes, documentation, and activities comply with the ISO/IEC 12207
      standards.
    \item[Actual Output] The system and its processes comply with ISO/IEC 12207 
    standards as determined through evaluation.
    \item[Result] Pass 
    \end{description}
  \item \label{NFRT29}
    \begin{description}
    \item[Initial State] This system is in a typical operational state with all
      components and services running.
    \item[Input/Condition] Various user interactions.
    \item[Expected Output] The computers continue to function within acceptable
      performance parameters throughout the test.
    \item[Actual Output] The computer's function is not affected throughout 
    various user interactions.
    \item[Result] Pass 
    \end{description}
  \item \label{NFRT30}
    \begin{description}
    \item[Initial State] This system is in a typical operational state with all
      components and services running.
    \item[Input/Condition] User inputs
    \item[Expected Output] System response from user inputs
    \item[Actual Output] The system responds appropriately and efficiently to user inputs.
    \item[Result] Pass 
    \end{description}
  \end{enumerate}
	
\section{Comparison to Existing Implementation}	

Our project shares similarities with two existing types of implementations. The first is a project developed by a group of researchers, focused on annotating Tai Chi videos with detailed features such as skeletal outlines and semantic segmentation. Like their summer Tai Chi annotation initiative, our project also annotates videos. However, ours operates in real-time, unlike the Tai Chi project which processes annotations offline, making ours more adept for live online learning contexts.

Additionally, our project parallels certain aspects of video conferencing tools like Zoom, which offer live annotation capabilities. While Zoom boasts a broader set of features due to its maturity as a platform, our project distinguishes itself in several key areas:

It is designed for one-directional streaming, in contrast to Zoom's multi-directional video conferencing.
Unlike Zoom, where annotations are generated client-side, our project renders annotations server-side, minimizing the hardware demands on viewers' devices.
Our system allows viewers the flexibility to choose their annotations, diverging from Zoom's model where annotations are unified for all viewers as determined by the video sender.
These distinctions underscore our project's unique contributions to video annotation technology, especially in educational and collaborative online environments.

\section{Unit Testing}
\subsection{User Authentication Module}
\begin{enumerate}[UT-UA1]
  \item \label{UT-UA1}
    \begin{description}
    \item[Initial State] The system initializes the Auth component within a mock routing environment.
    \item[Input/Condition] N/A
    \item[Expected Output] The component should display the Motion Mingle banner and a logo image.
    \item[Actual Output] The component displayed the Motion Mingle banner and a logo image.
    \item[Result] Pass
    \end{description}
  \item \label{UT-UA2}
    \begin{description}
    \item[Initial State] The system initializes the "Auth" component within a mock routing environment.
    \item[Input/Condition] The user clicks on the Instructor button.
    \item[Expected Output] The user should be navigated to the '/instructor' path.
    \item[Actual Output] The user was navigated to the '/instructor' path.
    \item[Result] Pass
    \end{description}
  \item \label{UT-UA3}
    \begin{description}
    \item[Initial State] The system initializes the "Auth" component within a mock routing environment.
    \item[Input/Condition]  The user clicks on the Practitioner button.
    \item[Expected Output] The user should be navigated to the '/practitioner' path.
    \item[Actual Output] The user was navigated to the '/practitioner' path.
    \item[Result] Pass
    \end{description}
  \item \label{UT-UA4}
    \begin{description}
    \item[Initial State]  The system initializes the "Auth" component within a mock routing environment.
    \item[Input/Condition] N/A
    \item[Expected Output] The slogan should be visible within the component with the correct text.
    \item[Actual Output] The slogan was displayed within the component with the correct text.
    \item[Result] Pass
    \end{description}
\end{enumerate}

\subsection{Instructor View Module}
\begin{enumerate}[UT-M1]
  \item \label{UT-M1}
    \begin{description}
    \item[Initial State] The system is initialized with default state settings. The "Instructor" component is about to be rendered within a mock routing environment.
    \item[Input/Condition] N/A
    \item[Expected Output] The "Instructor" component should be rendered with the title "Motion Mingle".
    \item[Actual Output] The "Instructor" component was rendered with the title "Motion Mingle".
    \item[Result] Pass
    \end{description}
  \item \label{UT-M2}
    \begin{description}
    \item[Initial State] The system has loaded the Instructor component, and the "SelectAnnotation" component is ready to receive user input.
    \item[Input/Condition] A user selects "Skeleton" annotation from the dropdown menu.
    \item[Expected Output] The dropdown menu should reflect that "Annotation 1" has been selected.
    \item[Actual Output] "Skeleton" has been selected from the dropdown menu.
    \item[Result] Pass
    \end{description}
  \item \label{UT-M3}
    \begin{description}
    \item[Initial State] The Instructor component is rendered and operational, presumed to be in a state where broadcasting is occurring (mocked condition).
    \item[Input/Condition] The user clicks the "Broadcast" button to simulate starting a broadcast and then clicks "Stop" to simulate ending the broadcast.
    \item[Expected Output] The "MessageModal" mock should be displayed, indicating that the stopping of the video is to be confirmed.
    \item[Actual Output] The "MessageModal" mock was displayed.
    \item[Result] Pass
    \end{description}
  \item \label{UT-M4}
    \begin{description}
    \item[Initial State] The Instructor component is rendered and operational. The self-video is initially off.
    \item[Input/Condition] The user clicks "Start self video" to initiate the video stream.
    \item[Expected Output] The system should transition to a state where the self-video is active.
    \item[Actual Output] The system transitioned to a state where the self-video is active.
    \item[Result] Pass
    \end{description}
  \item \label{UT-M5}
    \begin{description}
    \item[Initial State] The Instructor component is rendered and operational. The self-video is active.
    \item[Input/Condition] The user clicks "Close self video".
    \item[Expected Output] The system should transition to a state where the self-video is inactive.
    \item[Actual Output] The system transitioned to a state where the self-video is inactive.
    \item[Result] Pass
    \end{description}
\end{enumerate}

\subsection{Practitioner View Module}
\begin{enumerate}[UT-PV1]
  \item \label{UT-PV1}
    \begin{description}
    \item[Initial State] The system is initialized with default state settings. The "Practitioner" component is about to be rendered within a mock routing environment.
    \item[Input/Condition] N/A
    \item[Expected Output] The "Practitioner" component should be rendered with the title "Motion Mingle", a dropdown for selecting annotations initialized to "None", and a "Connect" button.
    \item[Actual Output] The "Practitioner" component was rendered with the title "Motion Mingle".
    \item[Result] Pass
    \end{description}
  \item \label{UT-PV2}
    \begin{description}
    \item[Initial State] The "Practitioner" component is rendered and displayed with the default selected annotation as "None".
    \item[Input/Condition] The user changes the annotation selection from "None" to "Skeleton" using the dropdown.
    \item[Expected Output] The dropdown menu for selecting annotations should reflect the change by displaying "Skeleton" as the current selection.
    \item[Actual Output] "Skeleton" was displayed as selected in the dropdown menu.
    \item[Result] Pass
    \end{description}
  \item \label{UT-PV3}
    \begin{description}
    \item[Initial State] The "Practitioner" component is rendered, displaying the "Connect" button while the video stream is not connected.
    \item[Input/Condition] The "Connect" button was clicked to start the video stream.
    \item[Expected Output] The "Connect" button should change to "Disconnect".
    \item[Actual Output]  The "Connect" button changed to "Disconnect".
    \item[Result] Pass
    \end{description}
  \item \label{UT-PV4}
    \begin{description}
    \item[Initial State] The "Practitioner" component is rendered, displaying the "Disconnect" button while the video stream is connected.
    \item[Input/Condition] The "Disconnect" button was clicked to stop the video stream.
    \item[Expected Output] The "Disconnect" button should change to "Connect".
    \item[Actual Output] The "Disconnect" button changed to "Connect".
    \item[Result] Pass
    \end{description}
\end{enumerate}

\subsection{Annotation Configuration Module}
\begin{enumerate}[UT-AC1]
  \item \label{UT-AC1}
    \begin{description}
    \item[Initial State] The "SelectAnnotation" component is initialized with default state settings.
    \item[Input/Condition] The component is rendered without any preselected value.
    \item[Expected Output] The component should display with the label "Annotation" and a dropdown button showing the default value "None".
    \item[Actual Output] The component was rendered with the label "Annotation" and a dropdown button showing the default value "None".
    \item[Result] Pass
    \end{description}
  \item \label{UT-AC2}
    \begin{description}
    \item[Initial State] The "SelectAnnotation" component is rendered with default settings.
    \item[Input/Condition] The user clicks on the dropdown menu to view available options.
    \item[Expected Output] The dropdown menu should display four options: "None", "Skeleton", "Edges", and "Cartoon".
    \item[Actual Output] The dropdown menu displayed four options: "None", "Skeleton", "Edges", and "Cartoon".
    \item[Result] Pass
    \end{description}
\end{enumerate}

\subsection{RTC Control Module}
\begin{enumerate}[UT-RC1]
  \item \label{UT-RC1}
    \begin{description}
    \item[Initial State] Before creating a new "RTCPeerConnection", no connection exists.
    \item[Input/Condition] The "createPeerConnection" function is called without any specific arguments since it uses predefined settings within its implementation.
    \item[Expected Output] A new "RTCPeerConnection" should be created with the configuration specified for unified-plan SDP semantics and the Google STUN server.
    \item[Actual Output] A new "RTCPeerConnection" was created with the configuration specified for unified-plan SDP semantics and the Google STUN server.
    \item[Result] Pass
    \end{description}
  \item \label{UT-RC2}
    \begin{description}
    \item[Initial State] A "RTCPeerConnection" instance exists but has not yet established a connection nor created any offers.
    \item[Input/Condition] The function "connectAsConsumer" is called with the created peer connection and "annotation" as arguments.
    \item[Expected Output] The function should asynchronously create an SDP offer, set it as the local description, send it to the server, and set the received SDP answer as the remote description.
    \item[Actual Output] Same as expected output.
    \item[Result] Pass
    \end{description}
  \item \label{UT-RC3}
    \begin{description}
    \item[Initial State]  A "RTCPeerConnection" instance is ready but not yet in a broadcasting state.
    \item[Input/Condition] The function "connectAsBroadcaster" is called with the peer connection object.
    \item[Expected Output] The function should asynchronously create an SDP offer for broadcasting, set it as the local description, send it to the server, and set the server's SDP answer as the remote description.
    \item[Actual Output] Same as expected.
    \item[Result] Pass
    \end{description}
\end{enumerate}

\subsection{Other Modules}
\begin{enumerate}[UT-OT1]
  \item \label{UT-OT1}
    \begin{description}
    \item[Initial State] The system initializes the "NotFound" component within a mock routing environment.
    \item[Input/Condition] The component is rendered without any user interaction.
    \item[Expected Output] The component should display the "Not Found" message.
    \item[Actual Output] The component displayed the "Not Found" message.
    \item[Result] Pass
    \end{description}
  \item \label{UT-OT2}
    \begin{description}
    \item[Initial State] The system initializes the "NotFound" component within a mock routing environment.
    \item[Input/Condition] The component is rendered, and the existence of a link is checked.
    \item[Expected Output] There should be a link that reads "GO HOME" and navigates to the root path '/' upon clicking.
    \item[Actual Output] Same as expected.
    \item[Result] Pass
    \end{description}
  \item \label{UT-OT3}
    \begin{description}
    \item[Initial State] The "MessageModal" component is rendered with "isModalOpen" set to true.
    \item[Input/Condition]  The "Cancel" button was clicked.
    \item[Expected Output] The "handleClose" mock function should be called once.
    \item[Actual Output] The "handleClose" mock function was called once.
    \item[Result] Pass
    \end{description}
  \item \label{UT-OT4}
    \begin{description}
    \item[Initial State] The "MessageModal" component is rendered with "isModalOpen" set to true.
    \item[Input/Condition] The "Stop Video" button was clicked.
    \item[Expected Output] The "handelStopVideo" mock function should be called once.
    \item[Actual Output] The "handelStopVideo" mock function was called once.
    \item[Result] Pass
    \end{description}
  \item \label{UT-OT5}
    \begin{description}
    \item[Initial State] The "MessageModal" component is initialized and rendered with "isModalOpen" set to true.
    \item[Input/Condition] N/A
    \item[Expected Output] The modal, including the warning message, should be visible.
    \item[Actual Output] The modal, including the warning message, was visible.
    \item[Result] Pass
    \end{description}
  \item \label{UT-OT6}
    \begin{description}
    \item[Initial State] The "MessageModal" component is initialized and rendered with "isModalOpen" set to false.
    \item[Input/Condition] N/A
    \item[Expected Output] None of the modal's content should be present in the document.
    \item[Actual Output] None of the modal's content was present in the document.
    \item[Result] Pass
    \end{description}
\end{enumerate}

\subsection{SFU Server Module}
\begin{enumerate}[UT-S1]
    \item \label{UT-S1}
    \begin{description}
    \item[Test Name] test\_broadcast\_options
    \item[Initial State] Server is initialized and running with /broadcast endpoint available.
    \item[Input/Condition] HTTP OPTIONS request to /broadcast.
    \item[Expected Output] HTTP status code 200.
    \item[Actual Output] HTTP status code 200.
    \item[Result] Pass
    \end{description}

    \item \label{UT-S2}
    \begin{description}
    \item[Test Name] test\_broadcast\_post
    \item[Initial State] Server is initialized with the /broadcast endpoint ready to accept POST requests.
    \item[Input/Condition] POST request to /broadcast with JSON body containing "sdp" and "type" fields.
    \item[Expected Output] HTTP status code 200 and a JSON response containing "sdp" and "type".
    \item[Actual Output] HTTP status code 200 and a JSON response containing "sdp" and "type".
    \item[Result] Pass
    \end{description}

    \item \label{UT-S3}
    \begin{description}
    \item[Test Name] test\_broadcast\_empty\_body\_post
    \item[Initial State] Server is initialized with the /broadcast endpoint ready to accept POST requests.
    \item[Input/Condition] POST request to /broadcast with an empty JSON body.
    \item[Expected Output] HTTP status code 400
    \item[Actual Output] HTTP status code 400
    \item[Result] Pass
    \end{description}

    \item \label{UT-S4}
    \begin{description}
    \item[Test Name] test\_broadcast\_post\_with\_invalid\_data\_types
    \item[Initial State] Server is initialized with the /broadcast endpoint ready.
    \item[Input/Condition] POST request to /broadcast with invalid data types for "sdp" and "type".
    \item[Expected Output] HTTP status code 400
    \item[Actual Output] HTTP status code 400
    \item[Result] Pass
    \end{description}

    \item \label{UT-S5}
    \begin{description}
    \item[Test Name] test\_broadcast\_post\_missing\_fields
    \item[Initial State] Server is initialized with the /broadcast endpoint ready.
    \item[Input/Condition] POST request to /broadcast with missing "type" field in the JSON body.
    \item[Expected Output] HTTP status code 400
    \item[Actual Output] HTTP status code 400
    \item[Result] Pass
    \end{description}

    \item \label{UT-S6}
    \begin{description}
    \item[Test Name] test\_invalid\_content\_type
    \item[Initial State] Server is initialized and ready to handle requests.
    \item[Input/Condition] POST request to /broadcast with a "Content-Type" of "text/plain".
    \item[Expected Output] HTTP status code 500
    \item[Actual Output] HTTP status code 500
    \item[Result] Pass
    \end{description}

    \item \label{UT-S7}
    \begin{description}
    \item[Test Name] test\_method\_not\_allowed
    \item[Initial State] Server is initialized with the /broadcast endpoint not configured to accept GET requests.
    \item[Input/Condition] GET request to /broadcast.
    \item[Expected Output] HTTP status code 405
    \item[Actual Output] HTTP status code 405
    \item[Result] Pass
    \end{description}

    \item \label{UT-S8}
    \begin{description}
    \item[Test Name] test\_response\_content\_type
    \item[Initial State] Server is initialized and ready to respond to POST requests.
    \item[Input/Condition] HTTP status code 200 and the "Content-Type" header is "application/json; charset=utf-8"
    \item[Actual Output] HTTP status code 200 and the "Content-Type" header is "application/json; charset=utf-8"
    \item[Result] Pass
    \end{description}

    \item \label{UT-S9}
    \begin{description}
    \item[Test Name] test\_consume\_options
    \item[Initial State] Server is initialized and running with /consumer endpoint available.
    \item[Input/Condition] HTTP OPTIONS request to /consumer.
    \item[Expected Output] HTTP status code 200
    \item[Actual Output] HTTP status code 200
    \item[Result] Pass
    \end{description}

    \item \label{UT-S10}
    \begin{description}
    \item[Test Name] test\_consume\_post
    \item[Initial State] Server is initialized with the /consumer endpoint ready to accept POST requests.
    \item[Input/Condition] POST request to /consumer with JSON body containing "sdp", "type", and "video\_transform".
    \item[Expected Output] HTTP status code 500
    \item[Actual Output] HTTP status code 500
    \item[Result] Pass
    \end{description}

    \item \label{UT-S11}
    \begin{description}
    \item[Test Name] test\_consume\_empty\_body\_post
    \item[Initial State] Server is initialized with the /consumer endpoint ready.
    \item[Input/Condition] POST request to /consumer with an empty JSON body.
    \item[Expected Output] HTTP status code 400
    \item[Actual Output] HTTP status code 400
    \item[Result] Pass
    \end{description}

    \item \label{UT-S12}
    \begin{description}
    \item[Test Name] test\_consume\_post\_unsupported\_video\_transform
    \item[Initial State] Server is initialized with the /consumer endpoint ready.
    \item[Input/Condition] POST request to /consumer with unsupported "video\_transform" value.
    \item[Expected Output] HTTP status code 400
    \item[Actual Output] HTTP status code 400
    \item[Result] Pass
    \end{description}
\end{enumerate}

\subsection{Video Transform Module}
Unit testing the video transform module is particularly challenging due to the nature of its operations which involve real-time video stream manipulation using complex image processing and machine learning models. 
These transformations, such as pose estimation and segmentation, rely on the input data's visual content, which can have near-infinite variability and require a visual context to determine correctness. 
The output is highly dependent on the specific content of the video frame being processed and the environmental conditions at the time of capture. Furthermore, such processes are typically non-deterministic due to the internal state management within the machine learning models (like MediaPipe's pose detection) and can be affected by the hardware acceleration capabilities of the environment where the tests are run. \\
Moreover, the transformations performed on video frames are designed to yield results that are subjectively evaluated visually, rather than through quantitative measures that could be programmatically asserted. 
While some aspects of the module could be unit tested by mocking dependencies and checking if the correct methods are called with expected arguments, the actual verification of the transformation's quality requires manual inspection or more sophisticated automated image comparison techniques that go beyond the scope of traditional unit testing. 
This inherent complexity and the need for a subjective assessment make the creation of automated unit tests for the video transform module both non-trivial and possibly less effective for ensuring the correctness of visual outputs.

\section{Changes Due to Testing}

\subsection{Change due to supervisor's feedback}
Andrew Mitchell, who is one of the supervisors of this project, has pointed out during the Revision 0 demo that the “check annotated video” button on the instructor page should appear after the instructor has started broadcasting instead of showing it directly, as clicking the check annotated video can’t show anything without input video stream. To follow the feedback from Andrew, the UI design will be changed so that the “check annotated video” button only shows up after the instructor starts broadcasting, and extra instructions will be added to the side of the instructor and practitioner view page.

\subsection{Change due to reviewing the question in the usability survey}
The VnV Plan, states that our group will review the questions in the usability survey as part of non-functional testing. One of the questions in the usability survey is: “Are there any specific content or learning materials you would like to see added to the application?” Some users state it is not convenient that they need to manually disconnect and reconnect to the server when selecting different annotations as a practitioner. The team decided to add an “auto-refresh” functionality to the system when an instructor or a practitioner changes their selected annotation would make the UI more intuitive and enhance user experiences.

\section{Automated Testing}
\subsection{Github Action}
Github Action provides a seamless and efficient way to ensure code integrity with each push. 
Tests are executed automatically on the latest commits to the main and develop branches, verifying that new changes do not break existing functionality. 
\subsection{Pytest}
PyTest is also used for the automatic testing. It is a robust and feature-rich Python library that streamlines the creation of test cases. 
By design, PyTest simplifies test authoring, allowing developers to write test functions that are concise and readable. 
It leverages the straightforward "assert" statement to check for expected outcomes, making test assertions both natural and easy to understand. 
Furthermore, PyTest's powerful parameterization capability, facilitated by the pytest.mark.parametrize decorator, permits the execution of a single test function against a variety of input sets. 
This not only enhances test coverage but also significantly boosts efficiency, as it allows for extensive testing without the need to write additional test cases.
 Overall, PyTest's intuitive approach and its ability to facilitate detailed and comprehensive testing with minimal code make it an invaluable tool for automated testing, contributing to faster development cycles and more reliable software.

\section{Trace to Requirements}

\setlength{\tabcolsep}{2pt}
\newgeometry{margin=2cm}
\begin{landscape}
  \begin{table}[h!]
    \centering
    \begin{tabular}{|c|c|c|c|c|c|c|c|c|c|c|c|c|c|c|c|c|} \hline
               & FR1 & FR2 & FR3 & FR4 & FR5 & FR6 & FR7 & FR8 & FR9 & FR10 & FR11 & FR12 & FR13 & FR14 & FR15 & FR16 \\ \hline
      \ref{FRT1}  & X   & X   &     &     &     &     &     &     &     &      &      &      &      &      &      &      \\ \hline
      \ref{FRT2}  &     &     & X   &     &     &     &     &     &     &      &      &      &      &      &      & X    \\ \hline
      \ref{FRT3}  &     &     &     & X   & X   &     &     &     &     &      &      &      &      &      &      &      \\ \hline
      \ref{FRT4}  &     &     &     &     &     & X   &     &     &     &      &      &      &      &      & X    &      \\ \hline
      \ref{FRT5}  &     &     &     &     &     &     &     & X   &     &      &      &      &      &      &      &      \\ \hline
      \ref{FRT6}  &     &     &     &     &     &     & X   &     &     &      &      &      &      &      &      &      \\ \hline
      \ref{FRT7}  &     &     &     &     &     &     &     &     & X   &      &      &      &      &      &      &      \\ \hline
      \ref{FRT8}  &     &     &     &     &     &     &     &     &     & X    &      &      &      &      &      &      \\ \hline
      \ref{FRT9}  &     &     &     &     &     &     &     &     &     &      & X    &      &      &      &      &      \\ \hline
      \ref{FRT10} &     &     &     &     &     &     &     &     &     &      &      & X    &      &      &      &      \\ \hline
      \ref{FRT11} &     &     &     &     &     &     &     &     &     &      &      &      & X    &      &      &      \\ \hline
      \ref{FRT12} &     &     &     &     &     &     &     &     &     &      &      &      &      & X    &      &      \\ \hline
    \end{tabular}
    \caption{Traceability matrix showing the connections between test cases
      and functional requirements}
    \label{tab:frt}
  \end{table}
  \begin{table}[h!]
    \centering
    \begin{tabular}{|c|c|c|c|c|c|c|c|c|c|c|c|c|c|c|c|c|} \hline
                & LF1 & UH1 & UH2 & PR1 & PR2 & PR3 & PR4 & PR5 & PR6 & PR7 & PR8 & PR9 & PR10 & PR11 & PR12 & PR13 \\ \hline
      \ref{NFRT1}  & X   & X   & X   &     &     &     &     &     &     &     &     &     &      &      &      &      \\ \hline
      \ref{NFRT2}  &     &     &     & X   & X   &     &     &     &     &     &     &     &      &      &      &      \\ \hline
      \ref{NFRT3}  &     &     &     &     & X   & X   &     &     &     &     &     &     &      &      &      &      \\ \hline
      \ref{NFRT4}  &     &     &     &     &     &     & X   &     &     &     &     &     &      &      &      &      \\ \hline
      \ref{NFRT5}  &     &     &     &     &     &     &     & X   &     &     &     &     &      &      &      &      \\ \hline
      \ref{NFRT6}  &     &     &     &     &     &     &     &     & X   &     &     &     &      &      &      &      \\ \hline
      \ref{NFRT7}  &     &     &     &     &     &     &     &     &     & X   &     &     &      &      &      &      \\ \hline
      \ref{NFRT8}  &     &     &     &     &     &     &     &     &     &     & X   &     &      &      &      &      \\ \hline
      \ref{NFRT9}  &     &     &     &     &     &     &     &     &     &     &     & X   &      &      &      &      \\ \hline
      \ref{NFRT10} &     &     &     &     &     &     &     &     &     &     &     &     & X    &      &      &      \\ \hline
      \ref{NFRT11} &     &     &     &     &     &     &     &     &     &     &     &     &      & X    &      &      \\ \hline
      \ref{NFRT12} &     &     &     &     &     &     &     &     &     &     &     &     &      &      & X    &      \\ \hline
      \ref{NFRT13} &     &     &     &     &     &     &     &     &     &     &     &     &      &      &      & X    \\ \hline
    \end{tabular}
    \caption{Traceability matrix showing the connections between test cases
      and non-functional requirements}
    \label{tab:nfrt1}
  \end{table}
  \begin{table}[h!]
    \centering
    \begin{tabular}{|c|c|c|c|c|c|c|c|c|c|c|c|c|c|c|c|c|c|} \hline
                & PR14 & PR15 & OE1 & OE2 & OE3 & MS1 & MS2 & SR1 & SR2 & SR3 & SR4 & SR5 & SR6 & CR1 & LR1 & HS1 & HS2 \\ \hline
      \ref{NFRT14} & X    &      &     &     &     &     &     &     &     &     &     &     &     &     &     &     &     \\ \hline
      \ref{NFRT15} &      & X    &     &     &     &     &     &     &     &     &     &     &     &     &     &     &     \\ \hline
      \ref{NFRT16} &      &      & X   &     &     &     &     &     &     &     &     &     &     &     &     &     &     \\ \hline
      \ref{NFRT17} &      &      &     & X   &     &     &     &     &     &     &     &     &     &     &     &     &     \\ \hline
      \ref{NFRT18} &      &      &     &     & X   &     &     &     &     &     &     &     &     &     &     &     &     \\ \hline
      \ref{NFRT19} &      &      &     &     &     & X   &     &     &     &     &     &     &     &     &     &     &     \\ \hline
      \ref{NFRT20} &      &      &     &     &     &     & X   &     &     &     &     &     &     &     &     &     &     \\ \hline
      \ref{NFRT21} &      &      &     &     &     &     &     & X   &     &     &     &     &     &     &     &     &     \\ \hline
      \ref{NFRT22} &      &      &     &     &     &     &     &     & X   &     &     &     &     &     &     &     &     \\ \hline
      \ref{NFRT23} &      &      &     &     &     &     &     &     &     & X   &     &     &     &     &     &     &     \\ \hline
      \ref{NFRT24} &      &      &     &     &     &     &     &     &     &     & X   &     &     &     &     &     &     \\ \hline
      \ref{NFRT25} &      &      &     &     &     &     &     &     &     &     &     & X   &     &     &     &     &     \\ \hline
      \ref{NFRT26} &      &      &     &     &     &     &     &     &     &     &     &     & X   &     &     &     &     \\ \hline
      \ref{NFRT27} &      &      &     &     &     &     &     &     &     &     &     &     &     & X   &     &     &     \\ \hline
      \ref{NFRT28} &      &      &     &     &     &     &     &     &     &     &     &     &     &     & X   &     &     \\ \hline
      \ref{NFRT29} &      &      &     &     &     &     &     &     &     &     &     &     &     &     &     & X   &     \\ \hline
      \ref{NFRT30} &      &      &     &     &     &     &     &     &     &     &     &     &     &     &     &     & X   \\ \hline
    \end{tabular}
    \caption{Traceability matrix showing the connections between test cases
      and non-functional requirements continued}
    \label{tab:nfrt2}
  \end{table}
\end{landscape}
\restoregeometry

\section{Trace to Modules}		

\section{Code Coverage Metrics}

\subsection*{Backend Code Coverage}
\bibliographystyle{plainnat}
\begin{table}[htbp]
  \centering
  \begin{tabular}{@{}lcccc@{}}
  \toprule
  \textbf{Module}                   & \textbf{statements} & \textbf{missing} & \textbf{excluded} & \textbf{coverage} \\ \midrule
  SFU Server Module                 & 118                 & 20               & 0                 & 83\%              \\
  Video Transform Module            & 86                  & 51               & 0                 & 41\%              \\ \midrule
  \textbf{Total}                    & 204                 & 71               & 0                 & 65\%              \\ \bottomrule
  \end{tabular}
  \caption{Backend Code coverage report}
  \label{tab:my-table}
\end{table}
The low code coverage for SFU Server Module can be attributed to the inherent complexity of testing asynchronous operations and external dependencies, such as WebRTC interactions(RTCPeerConnection and MediaStreamTrack), which are difficult to simulate in unit tests.\\
For the VideoTransformTrack module, the low coverage is due to the challenges in automatically testing image processing and machine learning functionalities that require subjective visual verification and can exhibit non-deterministic behavior.

\newpage{}
\section*{Appendix A --- Reflection}

The information in this section will be used to evaluate the team members on the
graduate attribute of Reflection.  Please answer the following question:

\begin{enumerate}
  \item In what ways was the Verification and Validation (VnV) Plan different
  from the activities that were actually conducted for VnV?  If there were
  differences, what changes required the modification in the plan?  Why did
  these changes occur?  Would you be able to anticipate these changes in future
  projects?  If there weren't any differences, how was your team able to clearly
  predict a feasible amount of effort and the right tasks needed to build the
  evidence that demonstrates the required quality?  (It is expected that most
  teams will have had to deviate from their original VnV Plan.)
\end{enumerate}

\section*{AppendixB --- Usability Survey}
The usability survey to MotionMingle can be found \href{https://docs.google.com/forms/d/e/1FAIpQLSchSdc_kHr98yHP8QPyWItHoP-dj_hJnuEtNByH0V2M_iDjWw/viewform}{here}.

\end{document}
\documentclass[12pt, titlepage]{article}

\usepackage{booktabs}
\usepackage{tabularx}
\newcolumntype{L}[1]{>{\raggedright\let\newline\\\arraybackslash\hspace{0pt}}p{#1}}
\usepackage{hyperref}
\hypersetup{
    colorlinks,
    citecolor=blue,
    filecolor=black,
    linkcolor=red,
    urlcolor=blue
}
\usepackage[round]{natbib}

\usepackage[shortlabels]{enumitem}
\usepackage{float}
\usepackage{geometry}
\usepackage{pdflscape}
\usepackage{siunitx}

\input{../Comments}
%% Common Parts

\newcommand{\progname}{SFWRENG 4G06 Capstone Design Project} % PUT YOUR PROGRAM NAME HERE
\newcommand{\projname}{MotionMingle} % project name
\newcommand{\authname}{Team \#18, InfiniView-AI
\\ Anhao Jiao
\\ Kehao Huang
\\ Qianlin Chen
\\ Shu Qi
\\ Xunzhou Ye
} % AUTHOR NAMES

\usepackage{hyperref}
    \hypersetup{colorlinks=true, linkcolor=blue, citecolor=blue, filecolor=blue,
                urlcolor=blue, unicode=false}
    \urlstyle{same}


\begin{document}

\title{Verification and Validation Report: \progname} 
\author{\authname}
\date{\today}
	
\maketitle

\pagenumbering{roman}

\section{Revision History}

\begin{tabularx}{\textwidth}{llX}
  \toprule {\bf Date} & {\bf Developer(s)} & {\bf Change} \\
  \midrule  
Mar 6 & Qi Shu & Add "Comparison to existing implementations section" and "Changes due to Testing" section.\\
\bottomrule
\end{tabularx}

~\newpage

\section{Symbols, Abbreviations and Acronyms}

\renewcommand{\arraystretch}{1.2}
\begin{tabular}{l l} 
  \toprule		
  \textbf{symbol} & \textbf{description}\\
  \midrule 
  T & Test\\
  \bottomrule
\end{tabular}\\

\wss{symbols, abbreviations or acronyms -- you can reference the SRS tables if needed}

\newpage

\tableofcontents

\listoftables %if appropriate

\listoffigures %if appropriate

\newpage

\pagenumbering{arabic}

This document ...

\section{Functional Requirements Evaluation}

\begin{enumerate}[FR-T1]
  \item \label{FRT1}
    \begin{description}
    \item[Type] Functional, Dynamic, Automated
    \item[Initial State] Client application is running on the user's device, but the
      user didn’t do any operations yet.
    \item[Input/Condition] User clicks on the applicable
      identity(instructor/practitioner) button
    \item[Expected Output] The live stream video window pops out on the user's screen.
    \item[Actual Output] The live stream video window shows on the instructor's screen, 
    it doesn't show on the practitioner's screen.
    \item[Result] Fail
    \end{description}
  \item \label{FRT2}
    \begin{description}
    \item[Type] Functional, Dynamic, Manual
    \item[Initial State] Application running on user’s computer, and the user has
      clicked on “the instructor identity button” to indicate they are a TaiChi
      instructor. A window asking for permission to use the camera on the
      instructor's device popped out.
    \item[Input/Condition] User allow/deny the webcam permission
    \item[Expected Output] The webcam on the instructor’s device is turned on
    \item[Actual Output] The webcam on the instructor’s device is turned on after 
    allowing webcam permission, and the webcam is not turned on after denying permission.
    \item[Result] Pass
    \end{description}
  \item \label{FRT3}
    \begin{description}
    \item[Type] Functional, Dynamic, Automatic
    \item[Initial State] both client applications and the server are running.
    \item[Input/Condition] The user clicks on the applicable identity button to
      indicate they are an instructor or a practitioner.
    \item[Expected Output] A log message indicates connection between the user’s device
      and the server has been established.
    \item[Actual Output] 
    \item[Result] 
    \end{description}
  \item \label{FRT4}
    \begin{description}
    \item[Type] Functional, Dynamic, Automated
    \item[Initial State] The live stream Window for practitioners.
    \item[Input/Condition] The user’s device has established a connection with the
      server as a practitioner device.
    \item[Expected Output] A request from the client device to the server for accessing
      the list of available annotation configuration.
    \item[Actual Output] 
    \item[Result] 
    \end{description}
  \item \label{FRT5}
    \begin{description}
    \item[Type] Functional, Dynamic, Automated
    \item[Initial State] The selectable list of the type of annotations is
      rendered on the user's screen.
    \item[Input/Condition] Practitioner’s selection on the list of types of
      annotations.
    \item[Expected Output] A request(that reflects user’s annotation selection) from
      the client device to the server for updating the annotation configuration,
      with a log indicating the request is sent.
    \item[Actual Output] 
    \item[Result] 
    \end{description}
  \item \label{FRT6}
    \begin{description}
    \item[Type] Functional, Dynamic, Automated
    \item[Initial State] The system is running and actively connected to
      practitioners.
    \item[Input/Condition] Practitioners initiate updates to annotation
      configurations.
    \item[Expected Output] The system receives and processes the updated annotation
      configurations.
    \item[Actual Output] The server gets the request and user sees the annotation they selected.
    \item[Result] Pass
    \end{description}
  \item \label{FRT7}
    \begin{description}
    \item[Type] Functional, Dynamic, Automated
    \item[Initial State] The server is running and actively receiving annotation
      configuration updates.
    \item[Input/Condition] In a controlled test environment, the practitioner-client
      initiates the update of an annotation configuration. The update is sent to
      the server for processing.
    \item[Expected Output] The expected result is that the server correctly processes
      the received annotation configuration from the practitioner-client.
    \item[Actual Output] The server is not able to receive annotation
    configuration updates while the connection has been established.
    \item[Result] Fail
    \end{description}
  \item \label{FRT8}
    \begin{description}
    \item[Type] Functional, Dynamic, Automated
    \item[Initial State] The server has received and processed the annotation
      configuration.
    \item[Input/Condition] The server uses the received annotation configuration to
      configure machine learning pipelines.
    \item[Expected Output] The machine learning pipelines are arranged and configured
      based on the annotation configuration.
    \item[Actual Output] 
    \item[Result] 
    \end{description}
  \item \label{FRT9}
    \begin{description}
    \item[Type] Functional, Dynamic, Automated
    \item[Initial State] The machine learning pipelines are configured and active.
    \item[Input/Condition] The instructor's video stream is processed with the
      annotation configuration.
    \item[Expected Output] The instructor's video stream is rendered with accurate
      annotations.
    \item[Actual Output] 
    \item[Result] 
    \end{description}
  \item \label{FRT10}
    \begin{description}
    \item[Type] Functional, Dynamic, Automated
    \item[Initial State] The server is actively connected to practitioner clients.
    \item[Input/Condition] The annotated video stream is generated and ready for
      transmission.
    \item[Expected Output] The annotated video stream is transmitted to each
      practitioner-client through their established connections.
    \item[Actual Output] 
    \item[Result] 
    \end{description}
  \item \label{FRT11}
    \begin{description}
    \item[Type] Functional, Dynamic, Automated
    \item[Initial State] The signaling server is running.
    \item[Input/Condition] Signaling requests for WebRTC connections are initiated.
    \item[Expected Output] The signaling server consistently responds to requests and
      establishes WebRTC connections.
    \item[Actual Output] 
    \item[Result] 
    \end{description}
  \item \label{FRT12}
    \begin{description}
    \item[Type] Functional, Dynamic, Automated
    \item[Initial State] The client application is running, but the user didn’t do
      any operations yet
    \item[Input/Condition] The user joining video stream session.
    \item[Expected Output] A button to identify if a user is an instructor or a
      practitioner is rendered.
    \item[Actual Output] 
    \item[Result] 
    \end{description}
  \end{enumerate}

\section{Nonfunctional Requirements Evaluation}

\subsection{Usability}
		
\subsection{Performance}

\subsection{etc.}
	
\section{Comparison to Existing Implementation}	

Our project shares similarities with two existing types of implementations. The first is a project developed by a group of researchers, focused on annotating Tai Chi videos with detailed features such as skeletal outlines and semantic segmentation. Like their summer Tai Chi annotation initiative, our project also annotates videos. However, ours operates in real-time, unlike the Tai Chi project which processes annotations offline, making ours more adept for live online learning contexts.

Additionally, our project parallels certain aspects of video conferencing tools like Zoom, which offer live annotation capabilities. While Zoom boasts a broader set of features due to its maturity as a platform, our project distinguishes itself in several key areas:

It is designed for one-directional streaming, in contrast to Zoom's multi-directional video conferencing.
Unlike Zoom, where annotations are generated client-side, our project renders annotations server-side, minimizing the hardware demands on viewers' devices.
Our system allows viewers the flexibility to choose their annotations, diverging from Zoom's model where annotations are unified for all viewers as determined by the video sender.
These distinctions underscore our project's unique contributions to video annotation technology, especially in educational and collaborative online environments.

\section{Unit Testing}
\subsection{User Authentication Module}
\begin{enumerate}[UT-UA1]
  \item \label{UT-UA1}
    \begin{description}
    \item[Initial State] The system initializes the Auth component within a mock routing environment.
    \item[Input/Condition] N/A
    \item[Expected Output] The component should display the Motion Mingle banner and a logo image.
    \item[Actual Output] The component displayed the Motion Mingle banner and a logo image.
    \item[Result] Pass
    \end{description}
  \item \label{UT-UA2}
    \begin{description}
    \item[Initial State] The system initializes the "Auth" component within a mock routing environment.
    \item[Input/Condition] The user clicks on the Instructor button.
    \item[Expected Output] The user should be navigated to the '/instructor' path.
    \item[Actual Output] The user was navigated to the '/instructor' path.
    \item[Result] Pass
    \end{description}
  \item \label{UT-UA3}
    \begin{description}
    \item[Initial State] The system initializes the "Auth" component within a mock routing environment.
    \item[Input/Condition]  The user clicks on the Practitioner button.
    \item[Expected Output] The user should be navigated to the '/practitioner' path.
    \item[Actual Output] The user was navigated to the '/practitioner' path.
    \item[Result] Pass
    \end{description}
  \item \label{UT-UA4}
    \begin{description}
    \item[Initial State]  The system initializes the "Auth" component within a mock routing environment.
    \item[Input/Condition] N/A
    \item[Expected Output] The slogan should be visible within the component with the correct text.
    \item[Actual Output] The slogan was displayed within the component with the correct text.
    \item[Result] Pass
    \end{description}
\end{enumerate}

\subsection{Instructor View Module}
\begin{enumerate}[UT-M1]
  \item \label{UT-M1}
    \begin{description}
    \item[Initial State] The system is initialized with default state settings. The "Instructor" component is about to be rendered within a mock routing environment.
    \item[Input/Condition] N/A
    \item[Expected Output] The "Instructor" component should be rendered with the title "Motion Mingle".
    \item[Actual Output] The "Instructor" component was rendered with the title "Motion Mingle".
    \item[Result] Pass
    \end{description}
  \item \label{UT-M2}
    \begin{description}
    \item[Initial State] The system has loaded the Instructor component, and the "SelectAnnotation" component is ready to receive user input.
    \item[Input/Condition] A user selects "Skeleton" annotation from the dropdown menu.
    \item[Expected Output] The dropdown menu should reflect that "Annotation 1" has been selected.
    \item[Actual Output] "Skeleton" has been selected from the dropdown menu.
    \item[Result] Pass
    \end{description}
  \item \label{UT-M3}
    \begin{description}
    \item[Initial State] The Instructor component is rendered and operational, presumed to be in a state where broadcasting is occurring (mocked condition).
    \item[Input/Condition] The user clicks the "Broadcast" button to simulate starting a broadcast and then clicks "Stop" to simulate ending the broadcast.
    \item[Expected Output] The "MessageModal" mock should be displayed, indicating that the stopping of the video is to be confirmed.
    \item[Actual Output] The "MessageModal" mock was displayed.
    \item[Result] Pass
    \end{description}
  \item \label{UT-M4}
    \begin{description}
    \item[Initial State] The Instructor component is rendered and operational. The self-video is initially off.
    \item[Input/Condition] The user clicks "Start self video" to initiate the video stream.
    \item[Expected Output] The system should transition to a state where the self-video is active.
    \item[Actual Output] The system transitioned to a state where the self-video is active.
    \item[Result] Pass
    \end{description}
  \item \label{UT-M5}
    \begin{description}
    \item[Initial State] The Instructor component is rendered and operational. The self-video is active.
    \item[Input/Condition] The user clicks "Close self video".
    \item[Expected Output] The system should transition to a state where the self-video is inactive.
    \item[Actual Output] The system transitioned to a state where the self-video is inactive.
    \item[Result] Pass
    \end{description}
\end{enumerate}

\subsection{Practitioner View Module}
\begin{enumerate}[UT-PV1]
  \item \label{UT-PV1}
    \begin{description}
    \item[Initial State] The system is initialized with default state settings. The "Practitioner" component is about to be rendered within a mock routing environment.
    \item[Input/Condition] N/A
    \item[Expected Output] The "Practitioner" component should be rendered with the title "Motion Mingle", a dropdown for selecting annotations initialized to "None", and a "Connect" button.
    \item[Actual Output] The "Practitioner" component was rendered with the title "Motion Mingle".
    \item[Result] Pass
    \end{description}
  \item \label{UT-PV2}
    \begin{description}
    \item[Initial State] The "Practitioner" component is rendered and displayed with the default selected annotation as "None".
    \item[Input/Condition] The user changes the annotation selection from "None" to "Skeleton" using the dropdown.
    \item[Expected Output] The dropdown menu for selecting annotations should reflect the change by displaying "Skeleton" as the current selection.
    \item[Actual Output] "Skeleton" was displayed as selected in the dropdown menu.
    \item[Result] Pass
    \end{description}
  \item \label{UT-PV3}
    \begin{description}
    \item[Initial State] The "Practitioner" component is rendered, displaying the "Connect" button while the video stream is not connected.
    \item[Input/Condition] The "Connect" button was clicked to start the video stream.
    \item[Expected Output] The "Connect" button should change to "Disconnect".
    \item[Actual Output]  The "Connect" button changed to "Disconnect".
    \item[Result] Pass
    \end{description}
  \item \label{UT-PV4}
    \begin{description}
    \item[Initial State] The "Practitioner" component is rendered, displaying the "Disconnect" button while the video stream is connected.
    \item[Input/Condition] The "Disconnect" button was clicked to stop the video stream.
    \item[Expected Output] The "Disconnect" button should change to "Connect".
    \item[Actual Output] The "Disconnect" button changed to "Connect".
    \item[Result] Pass
    \end{description}
\end{enumerate}

\subsection{Annotation Configuration Module}
\begin{enumerate}[UT-AC1]
  \item \label{UT-AC1}
    \begin{description}
    \item[Initial State] The "SelectAnnotation" component is initialized with default state settings.
    \item[Input/Condition] The component is rendered without any preselected value.
    \item[Expected Output] The component should display with the label "Annotation" and a dropdown button showing the default value "None".
    \item[Actual Output] The component was rendered with the label "Annotation" and a dropdown button showing the default value "None".
    \item[Result] Pass
    \end{description}
  \item \label{UT-AC2}
    \begin{description}
    \item[Initial State] The "SelectAnnotation" component is rendered with default settings.
    \item[Input/Condition] The user clicks on the dropdown menu to view available options.
    \item[Expected Output] The dropdown menu should display four options: "None", "Skeleton", "Edges", and "Cartoon".
    \item[Actual Output] The dropdown menu displayed four options: "None", "Skeleton", "Edges", and "Cartoon".
    \item[Result] Pass
    \end{description}
\end{enumerate}

\subsection{RTC Control Module}
\begin{enumerate}[UT-RC1]
  \item \label{UT-RC1}
    \begin{description}
    \item[Initial State] Before creating a new "RTCPeerConnection", no connection exists.
    \item[Input/Condition] The "createPeerConnection" function is called without any specific arguments since it uses predefined settings within its implementation.
    \item[Expected Output] A new "RTCPeerConnection" should be created with the configuration specified for unified-plan SDP semantics and the Google STUN server.
    \item[Actual Output] A new "RTCPeerConnection" was created with the configuration specified for unified-plan SDP semantics and the Google STUN server.
    \item[Result] Pass
    \end{description}
  \item \label{UT-RC2}
    \begin{description}
    \item[Initial State] A "RTCPeerConnection" instance exists but has not yet established a connection nor created any offers.
    \item[Input/Condition] The function "connectAsConsumer" is called with the created peer connection and "annotation" as arguments.
    \item[Expected Output] The function should asynchronously create an SDP offer, set it as the local description, send it to the server, and set the received SDP answer as the remote description.
    \item[Actual Output] Same as expected output.
    \item[Result] Pass
    \end{description}
  \item \label{UT-RC3}
    \begin{description}
    \item[Initial State]  A "RTCPeerConnection" instance is ready but not yet in a broadcasting state.
    \item[Input/Condition] The function "connectAsBroadcaster" is called with the peer connection object.
    \item[Expected Output] The function should asynchronously create an SDP offer for broadcasting, set it as the local description, send it to the server, and set the server's SDP answer as the remote description.
    \item[Actual Output] Same as expected.
    \item[Result] Pass
    \end{description}
\end{enumerate}

\subsection{Other Modules}
\begin{enumerate}[UT-OT1]
  \item \label{UT-OT1}
    \begin{description}
    \item[Initial State] The system initializes the "NotFound" component within a mock routing environment.
    \item[Input/Condition] The component is rendered without any user interaction.
    \item[Expected Output] The component should display the "Not Found" message.
    \item[Actual Output] The component displayed the "Not Found" message.
    \item[Result] Pass
    \end{description}
  \item \label{UT-OT2}
    \begin{description}
    \item[Initial State] The system initializes the "NotFound" component within a mock routing environment.
    \item[Input/Condition] The component is rendered, and the existence of a link is checked.
    \item[Expected Output] There should be a link that reads "GO HOME" and navigates to the root path '/' upon clicking.
    \item[Actual Output] Same as expected.
    \item[Result] Pass
    \end{description}
  \item \label{UT-OT3}
    \begin{description}
    \item[Initial State] The "MessageModal" component is rendered with "isModalOpen" set to true.
    \item[Input/Condition]  The "Cancel" button was clicked.
    \item[Expected Output] The "handleClose" mock function should be called once.
    \item[Actual Output] The "handleClose" mock function was called once.
    \item[Result] Pass
    \end{description}
  \item \label{UT-OT4}
    \begin{description}
    \item[Initial State] The "MessageModal" component is rendered with "isModalOpen" set to true.
    \item[Input/Condition] The "Stop Video" button was clicked.
    \item[Expected Output] The "handelStopVideo" mock function should be called once.
    \item[Actual Output] The "handelStopVideo" mock function was called once.
    \item[Result] Pass
    \end{description}
  \item \label{UT-OT5}
    \begin{description}
    \item[Initial State] The "MessageModal" component is initialized and rendered with "isModalOpen" set to true.
    \item[Input/Condition] N/A
    \item[Expected Output] The modal, including the warning message, should be visible.
    \item[Actual Output] The modal, including the warning message, was visible.
    \item[Result] Pass
    \end{description}
  \item \label{UT-OT6}
    \begin{description}
    \item[Initial State] The "MessageModal" component is initialized and rendered with "isModalOpen" set to false.
    \item[Input/Condition] N/A
    \item[Expected Output] None of the modal's content should be present in the document.
    \item[Actual Output] None of the modal's content was present in the document.
    \item[Result] Pass
    \end{description}
\end{enumerate}

\section{Changes Due to Testing}

\subsection{Change due to supervisor's feedback}
Andrew Mitchell, who is one of the supervisors of this project, has pointed out during the Revision 0 demo that the “check annotated video” button on the instructor page should appear after the instructor has started broadcasting instead of showing it directly, as clicking the check annotated video can’t show anything without input video stream. To follow the feedback from Andrew, the UI design will be changed so that the “check annotated video” button only shows up after the instructor starts broadcasting, and extra instructions will be added to the side of the instructor and practitioner view page.

\subsection{Change due to reviewing the question in the usability survey}
The VnV Plan, states that our group will review the questions in the usability survey as part of non-functional testing. One of the questions in the usability survey is: “Are there any specific content or learning materials you would like to see added to the application?” Some users state it is not convenient that they need to manually disconnect and reconnect to the server when selecting different annotations as a practitioner. The team decided to add an “auto-refresh” functionality to the system when an instructor or a practitioner changes their selected annotation would make the UI more intuitive and enhance user experiences.

\section{Automated Testing}
		
\section{Trace to Requirements}
		
\section{Trace to Modules}		

\section{Code Coverage Metrics}

\bibliographystyle{plainnat}
\bibliography{../../refs/References}

\newpage{}
\section*{Appendix --- Reflection}

The information in this section will be used to evaluate the team members on the
graduate attribute of Reflection.  Please answer the following question:

\begin{enumerate}
  \item In what ways was the Verification and Validation (VnV) Plan different
  from the activities that were actually conducted for VnV?  If there were
  differences, what changes required the modification in the plan?  Why did
  these changes occur?  Would you be able to anticipate these changes in future
  projects?  If there weren't any differences, how was your team able to clearly
  predict a feasible amount of effort and the right tasks needed to build the
  evidence that demonstrates the required quality?  (It is expected that most
  teams will have had to deviate from their original VnV Plan.)
\end{enumerate}

\end{document}
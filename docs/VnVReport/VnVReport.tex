\documentclass[12pt, titlepage]{article}

\usepackage{booktabs}
\usepackage{tabularx}
\newcolumntype{L}[1]{>{\raggedright\let\newline\\\arraybackslash\hspace{0pt}}p{#1}}
\usepackage{hyperref}
\hypersetup{
    colorlinks,
    citecolor=blue,
    filecolor=black,
    linkcolor=red,
    urlcolor=blue
}
\usepackage[round]{natbib}

\usepackage[shortlabels]{enumitem}
\usepackage{float}
\usepackage{geometry}
\usepackage{pdflscape}
\usepackage{siunitx}

%% Comments

\usepackage{color}

\newif\ifcomments\commentstrue %displays comments
%\newif\ifcomments\commentsfalse %so that comments do not display

\ifcomments
\newcommand{\authornote}[3]{\textcolor{#1}{[#3 ---#2]}}
\newcommand{\todo}[1]{\textcolor{red}{[TODO: #1]}}
\else
\newcommand{\authornote}[3]{}
\newcommand{\todo}[1]{}
\fi

\newcommand{\wss}[1]{\authornote{blue}{SS}{#1}} 
\newcommand{\plt}[1]{\authornote{magenta}{TPLT}{#1}} %For explanation of the template
\newcommand{\an}[1]{\authornote{cyan}{Author}{#1}}

%% Common Parts

\newcommand{\progname}{SFWRENG 4G06 Capstone Design Project} % PUT YOUR PROGRAM NAME HERE
\newcommand{\projname}{MotionMingle} % project name
\newcommand{\authname}{Team \#18, InfiniView-AI
\\ Anhao Jiao
\\ Kehao Huang
\\ Qianlin Chen
\\ Shu Qi
\\ Xunzhou Ye
} % AUTHOR NAMES

\usepackage{hyperref}
    \hypersetup{colorlinks=true, linkcolor=blue, citecolor=blue, filecolor=blue,
                urlcolor=blue, unicode=false}
    \urlstyle{same}


\begin{document}

\title{Verification and Validation Report: \progname} 
\author{\authname}
\date{\today}
	
\maketitle

\pagenumbering{roman}

\section{Revision History}

\begin{tabularx}{\textwidth}{p{3cm}p{2cm}X}
\toprule {\bf Date} & {\bf Version} & {\bf Notes}\\
\midrule
Date 1 & 1.0 & Notes\\
Date 2 & 1.1 & Notes\\
\bottomrule
\end{tabularx}

~\newpage

\section{Symbols, Abbreviations and Acronyms}

\renewcommand{\arraystretch}{1.2}
\begin{tabular}{l l} 
  \toprule		
  \textbf{symbol} & \textbf{description}\\
  \midrule 
  T & Test\\
  \bottomrule
\end{tabular}\\

\wss{symbols, abbreviations or acronyms -- you can reference the SRS tables if needed}

\newpage

\tableofcontents

\listoftables %if appropriate

\listoffigures %if appropriate

\newpage

\pagenumbering{arabic}

This document ...

\section{Functional Requirements Evaluation}

\begin{enumerate}[FR-T1]
  \item \label{FRT1}
    \begin{description}
    \item[Type] Functional, Dynamic, Automated
    \item[Initial State] Client application is running on the user's device, but the
      user didn’t do any operations yet.
    \item[Input/Condition] User clicks on the applicable
      identity(instructor/practitioner) button
    \item[Expected Output] The live stream video window pops out on the user's screen.
    \item[Actual Output] The live stream video window shows on the instructor's screen, 
    it doesn't show on the practitioner's screen.
    \item[Result] Fail
    \end{description}
  \item \label{FRT2}
    \begin{description}
    \item[Type] Functional, Dynamic, Manual
    \item[Initial State] Application running on user’s computer, and the user has
      clicked on “the instructor identity button” to indicate they are a TaiChi
      instructor. A window asking for permission to use the camera on the
      instructor's device popped out.
    \item[Input/Condition] User allow/deny the webcam permission
    \item[Expected Output] The webcam on the instructor’s device is turned on
    \item[Actual Output] The webcam on the instructor’s device is turned on after 
    allowing webcam permission, and the webcam is not turned on after denying permission.
    \item[Result] Pass
    \end{description}
  \item \label{FRT3}
    \begin{description}
    \item[Type] Functional, Dynamic, Automatic
    \item[Initial State] both client applications and the server are running.
    \item[Input/Condition] The user clicks on the applicable identity button to
      indicate they are an instructor or a practitioner.
    \item[Expected Output] A log message indicates connection between the user’s device
      and the server has been established.
    \item[Actual Output] 
    \item[Result] 
    \end{description}
  \item \label{FRT4}
    \begin{description}
    \item[Type] Functional, Dynamic, Automated
    \item[Initial State] The live stream Window for practitioners.
    \item[Input/Condition] The user’s device has established a connection with the
      server as a practitioner device.
    \item[Expected Output] A request from the client device to the server for accessing
      the list of available annotation configuration.
    \item[Actual Output] 
    \item[Result] 
    \end{description}
  \item \label{FRT5}
    \begin{description}
    \item[Type] Functional, Dynamic, Automated
    \item[Initial State] The selectable list of the type of annotations is
      rendered on the user's screen.
    \item[Input/Condition] Practitioner’s selection on the list of types of
      annotations.
    \item[Expected Output] A request(that reflects user’s annotation selection) from
      the client device to the server for updating the annotation configuration,
      with a log indicating the request is sent.
    \item[Actual Output] 
    \item[Result] 
    \end{description}
  \item \label{FRT6}
    \begin{description}
    \item[Type] Functional, Dynamic, Automated
    \item[Initial State] The system is running and actively connected to
      practitioners.
    \item[Input/Condition] Practitioners initiate updates to annotation
      configurations.
    \item[Expected Output] The system receives and processes the updated annotation
      configurations.
    \item[Actual Output] The server gets the request and user sees the annotation they selected.
    \item[Result] Pass
    \end{description}
  \item \label{FRT7}
    \begin{description}
    \item[Type] Functional, Dynamic, Automated
    \item[Initial State] The server is running and actively receiving annotation
      configuration updates.
    \item[Input/Condition] In a controlled test environment, the practitioner-client
      initiates the update of an annotation configuration. The update is sent to
      the server for processing.
    \item[Expected Output] The expected result is that the server correctly processes
      the received annotation configuration from the practitioner-client.
    \item[Actual Output] The server is not able to receive annotation
    configuration updates while the connection has been established.
    \item[Result] Fail
    \end{description}
  \item \label{FRT8}
    \begin{description}
    \item[Type] Functional, Dynamic, Automated
    \item[Initial State] The server has received and processed the annotation
      configuration.
    \item[Input/Condition] The server uses the received annotation configuration to
      configure machine learning pipelines.
    \item[Expected Output] The machine learning pipelines are arranged and configured
      based on the annotation configuration.
    \item[Actual Output] 
    \item[Result] 
    \end{description}
  \item \label{FRT9}
    \begin{description}
    \item[Type] Functional, Dynamic, Automated
    \item[Initial State] The machine learning pipelines are configured and active.
    \item[Input/Condition] The instructor's video stream is processed with the
      annotation configuration.
    \item[Expected Output] The instructor's video stream is rendered with accurate
      annotations.
    \item[Actual Output] 
    \item[Result] 
    \end{description}
  \item \label{FRT10}
    \begin{description}
    \item[Type] Functional, Dynamic, Automated
    \item[Initial State] The server is actively connected to practitioner clients.
    \item[Input/Condition] The annotated video stream is generated and ready for
      transmission.
    \item[Expected Output] The annotated video stream is transmitted to each
      practitioner-client through their established connections.
    \item[Actual Output] 
    \item[Result] 
    \end{description}
  \item \label{FRT11}
    \begin{description}
    \item[Type] Functional, Dynamic, Automated
    \item[Initial State] The signaling server is running.
    \item[Input/Condition] Signaling requests for WebRTC connections are initiated.
    \item[Expected Output] The signaling server consistently responds to requests and
      establishes WebRTC connections.
    \item[Actual Output] 
    \item[Result] 
    \end{description}
  \item \label{FRT12}
    \begin{description}
    \item[Type] Functional, Dynamic, Automated
    \item[Initial State] The client application is running, but the user didn’t do
      any operations yet
    \item[Input/Condition] The user joining video stream session.
    \item[Expected Output] A button to identify if a user is an instructor or a
      practitioner is rendered.
    \item[Actual Output] 
    \item[Result] 
    \end{description}
  \end{enumerate}

\section{Nonfunctional Requirements Evaluation}

\subsection{Usability}
		
\subsection{Performance}

\subsection{etc.}
	
\section{Comparison to Existing Implementation}	

This section will not be appropriate for every project.

\section{Unit Testing}

\section{Changes Due to Testing}

\wss{This section should highlight how feedback from the users and from 
the supervisor (when one exists) shaped the final product.  In particular 
the feedback from the Rev 0 demo to the supervisor (or to potential users) 
should be highlighted.}

\section{Automated Testing}
		
\section{Trace to Requirements}
		
\section{Trace to Modules}		

\section{Code Coverage Metrics}

\bibliographystyle{plainnat}
\bibliography{../../refs/References}

\newpage{}
\section*{Appendix --- Reflection}

The information in this section will be used to evaluate the team members on the
graduate attribute of Reflection.  Please answer the following question:

\begin{enumerate}
  \item In what ways was the Verification and Validation (VnV) Plan different
  from the activities that were actually conducted for VnV?  If there were
  differences, what changes required the modification in the plan?  Why did
  these changes occur?  Would you be able to anticipate these changes in future
  projects?  If there weren't any differences, how was your team able to clearly
  predict a feasible amount of effort and the right tasks needed to build the
  evidence that demonstrates the required quality?  (It is expected that most
  teams will have had to deviate from their original VnV Plan.)
\end{enumerate}

\end{document}
\documentclass{article}

\usepackage{tabularx}
\usepackage{booktabs}
\usepackage{ulem}

\title{Problem Statement and Goals\\\progname}

\author{\authname}

\date{25 September 2023}

%% Comments

\usepackage{color}

\newif\ifcomments\commentstrue %displays comments
%\newif\ifcomments\commentsfalse %so that comments do not display

\ifcomments
\newcommand{\authornote}[3]{\textcolor{#1}{[#3 ---#2]}}
\newcommand{\todo}[1]{\textcolor{red}{[TODO: #1]}}
\else
\newcommand{\authornote}[3]{}
\newcommand{\todo}[1]{}
\fi

\newcommand{\wss}[1]{\authornote{blue}{SS}{#1}} 
\newcommand{\plt}[1]{\authornote{magenta}{TPLT}{#1}} %For explanation of the template
\newcommand{\an}[1]{\authornote{cyan}{Author}{#1}}

%% Common Parts

\newcommand{\progname}{SFWRENG 4G06 Capstone Design Project} % PUT YOUR PROGRAM NAME HERE
\newcommand{\projname}{MotionMingle} % project name
\newcommand{\authname}{Team \#18, InfiniView-AI
\\ Anhao Jiao
\\ Kehao Huang
\\ Qianlin Chen
\\ Shu Qi
\\ Xunzhou Ye
} % AUTHOR NAMES

\usepackage{hyperref}
    \hypersetup{colorlinks=true, linkcolor=blue, citecolor=blue, filecolor=blue,
                urlcolor=blue, unicode=false}
    \urlstyle{same}


\begin{document}

\maketitle

\begin{table}[hp]
  \caption{Revision History} \label{TblRevisionHistory}
  \begin{tabularx}{\textwidth}{llX}
    \toprule
    \textbf{Date} & \textbf{Developer(s)} & \textbf{Change}\\
    \midrule
    25 September 2023 & AJ, KH, QC, QS, XY & Initial draft \\
    \midrule
    26 March 2024 & AJ, KH, QC, QS, XY & Rev1 \\
    \bottomrule
  \end{tabularx}
\end{table}

\section{Problem Statement}
\textcolor{red}{The following section outlines the primary challenges 
encountered by users of the Tai Chi instruction platform, particularly 
focusing on the transition from traditional in-person instruction to an online 
environment. It examines the specific limitations that currently exist within 
this new learning context and identifies the core issues that the platform seeks 
to address. Understanding these challenges is essential for developing solutions 
that enhance the user experience and learning outcomes.}

\subsection{Problem}

Tai Chi is a complete martial art system practiced chiefly by older individuals
for the purpose of health promotion and rehabilitation. It has been described as
“meditation in motion” due to its emphasis on slow, gentle movements that have a
low impact on the body and joints. During the COVID-19 pandemic, group exercise
classes like Tai Chi transitioned to the online format due to the enforcement of
social distancing. Therefore, online training for these group exercise lessons
is conducted in various forms including offering live sessions with webcams and
providing pre-recorded videos. However, it is widely perceived among both
practitioners and instructors that current models of teaching Tai Chi virtually
fall short compared to in-person instruction. One significant drawback of
learning Tai Chi Online is that practitioners may not be able to see all the
angles of the instructor’s body needed to understand and mimic the movements.
Also, the lack of real-time feedback from instructors and interaction among
participants is also limiting the learning outcomes and engagement for
practitioners. Therefore, a solution is desired to compensate for the
limitations of virtual Tai Chi lessons, and expectedly improve the learning
outcomes and engagement for practitioners to a similar or higher level of the
in-person teaching model.

\subsection{Inputs and Outputs}

\subsubsection{Inputs}

\begin{itemize}
\item Video stream of Tai Chi instructors
\item Video stream of practitioners
\item Audio from both instructors and practitioners, to enable communication during
  the session
\item Annotation preference of practitioners
\end{itemize}

\subsubsection{Outputs}

\begin{itemize}
\item Tai Chi virtual lessons with audio and real-time annotations
\item Personalized streaming interface
\end{itemize}

\subsection{Stakeholders}

\begin{itemize}
\item \sout{Professor of SFWRENG 4G06, Dr. Spencer Smith}
\item Supervisors of the project, Dr. Rong Zheng, Andrew Mitchell
\item Tai Chi instructors and practitioners
\end{itemize}

\subsection{Environment}

\begin{description}
\item[Software] Windows, Linux or Mac OS
\item[Hardware] Computers with a camera and optionally a microphone
\end{description}

\section{Goals}

\begin{description}
\item[Learning and teaching experience improvement] The product should provide at
  least three types of visual effects to show the instructor's motions in Tai
  Chi lectures, making it easier to see the instructor's motions in video calls.
\item[Minimizing the hardware requirements for end users] The product should be
  accessible to users as long as they have a device that supports video
  conferences.
\item[Real-time video post-processing] The speed of video processing and visual
  effect generation should not be noticeable. There should be no significant lag
  (\(< 3\) seconds) in the instructor's and practitioner's video calls.
\item[Safety of User Data] The application should secure user data. The application
  should not have any case of data leakage.
\item[Ease of use] The application should be intuitive to use and no further
  instruction is required. People of all age groups should be able to learn all
  the features of the application with ease.
\end{description}

\section{Stretch Goals}

\begin{description}
\item[Accurate annotations] The instructor's body annotation should give the best
  representation of the instructor's body motion.
\item[Form recognition and captioning] The application should be able to recognize
  the Tai Chi form based on the instructor's motion and display the name of the
  form as a caption on top of the video stream.
\end{description}

\end{document}